\documentclass[12pt]{article}

%INCLUDEONLY
\includeonly{sections/animace.tex}

%!!!!!!!!! Zapomeň odkomentovat attachfile !!!!!!!!!!
%!!!!!!!!! Nezapomeň opravit pracovní listy  !!!!!!!!!!

%DOČASNÉ
\usepackage{xcolor}
\usepackage{blindtext}
\usepackage{soul}
\hfuzz=10pt
\vfuzz=10pt
\hbadness=10000
\vbadness=10000

%Čeština
\usepackage[czech]{babel}
\usepackage[utf8]{inputenc}
\usepackage[T1]{fontenc}
\selectlanguage{czech}
\usepackage[a4paper,top=3cm,bottom=2.5cm,left=2.5cm,right=2.5cm]{geometry}
\usepackage{amsmath}
\usepackage[colorlinks=true, allcolors=black]{hyperref}

%Zdroje
\usepackage[backend=biber, style=iso-numeric, sorting=none, maxbibnames=3]{biblatex}
\addbibresource{zdroje.bib}
\usepackage{csquotes}

%Obrázky
\usepackage[draft]{graphicx} % pro rychlejší kompilaci lze přidat [draft]
\usepackage{float}
\usepackage{caption}
\usepackage{subcaption}

%Pracovní listy
\usepackage{titling}
\usepackage{enumitem}
\usepackage{dashrule}
\let\iint\relax
\usepackage{wasysym}
\usepackage{xfrac}
\usepackage{upgreek}
\usepackage{tcolorbox}
\usepackage{pdfpages}
\usepackage{fancybox}
\usepackage{framed}
\usepackage{attachfile2}
\newcommand{\pracovniList}[3]{
    \textbf{#2} {Dostupné ke stažení }\textattachfile[color=0 0 0]{#3}{\textit{zde}}\odst
    \begin{minipage}{0.5\textwidth}
      \centering
      \fbox{
        \includegraphics[scale = 0.3, page=1, clip]{#1}
      }
    \end{minipage}%
    \begin{minipage}{0.5\textwidth}
      \centering
      \fbox{
        \includegraphics[scale = 0.3, page=2, clip]{#1}
      }
    \end{minipage}\odst
}
\usepackage{tikz}
\usetikzlibrary{fadings}
\usepackage[normalem]{ulem}
\newcommand{\cara}[1]{\hspace{0.1cm}{\makebox[#1]{\dotfill}}}
\newcommand{\body}[1]{\begin{flushright}\rule{0.5cm}{0.4pt}{ / }{#1}\end{flushright}}
\newcommand{\tecky}[1]{\makebox[#1]{\dotfill}}
\newcommand{\nadtecky}[1]{\textcolor{red}{\raisebox{0.05cm}{\rlap{\hspace{0.25cm}#1}}}}
\newcommand{\nadteckyN}[1]{\textcolor{red}{\raisebox{0.05cm}{\rlap{#1}}}}
\newcommand{\redeq}[1]{{\setlength{\abovedisplayskip}{5pt}\setlength{\belowdisplayskip}{5pt}\textcolor{red}{\begin{equation*}#1\end{equation*}}}}
\newcommand{\dottext}[1]{\ensuremath{\dot{\text{#1}}}}

%Tabulky
\usepackage[table]{xcolor}
\usepackage{tabularx}
\usepackage{diagbox}

%Import grafu v souboru .pgf
\usepackage{pgf}

%Font
\usepackage{tgtermes}

\usetikzlibrary{patterns}
%DiagramMotoru #1 = offset #2 = piston height #3 = intake #4 = exhaust #5 = counterweight angle #6 = sincos
\newcommand{\makeEngineFourStrokeGasolineDiagram}[6]{%
  \makePiston{#1}{#2}

  \makeCase{#1}
  
  \makeSparkplug{#1}

  %crankshaft
  \makeCounterweight{#1}{#5}

  %connecting rod
  \makeConnectingRod{#1}{#2}{#6}

  \makeValve{#1}{#3/5} %intake
  \makeValve{#1 + 1}{#4/5} %exhaust
}

%DiagramMotoru #1 = offset #2 = piston height #3 = intake #4 = exhaust #5 = counterweight angle #6 = sincos
\newcommand{\makeEngineFourStrokeDieselDiagram}[6]{%
  \makePiston{#1}{#2}

  \makeCase{#1}

  \draw[thick]%sparker
  (0.925 + #1, 3) to (0.95 + #1, 2.9) to (1.05 + #1, 2.9) to (1.075 + #1, 3) to cycle
  (1.1 + #1, 3.1) to (1.1 + #1, 3.3) to (1.2 + #1, 3.4) to (1.2 + #1, 3.6) (0.8 + #1, 3.6) to (0.8 + #1, 3.4) to (0.9 + #1, 3.3) to (0.9 + #1, 3.1);

  %crankshaft
  \makeCounterweight{#1}{#5}

  %connecting rod
  \makeConnectingRod{#1}{#2}{#6}

  \makeValve{#1}{#3/5} %intake
  \makeValve{#1 + 1}{#4/5} %exhaust
}

\newcommand{\makeEngineTwoStrokeDiagram}[6]{%
  \makePiston{#1}{#2}

  %Case
  \draw[thick, pattern=north east lines]
  (0 + #1,0) to (0 + #1,1) to (-0.1 + #1, 1) to (-0.1 + #1, 0) -- cycle
  (0 + #1,-1) to (0 + #1,-0.5) to [in=-90, out=180](-0.5 + #1,0) to (-0.5 + #1,1) to [out=90,in=180](0 + #1,1.5) to (0 + #1,3) to (2 + #1,3) to (2 + #1,2) to (2.5 + #1,2) to (2.5 + #1, 2.1) to (2.1 + #1,2.1) to (2.1 + #1,3.1) to (-0.1 + #1, 3.1) to (-0.1 + #1,1.6) to [out=180,in=90](-0.6 + #1,1) to (-0.6 + #1,-0.1) to [out=-90,in=180](-0.1 + #1, -0.6) to (-0.1 + #1, -1) -- cycle
  (2 + #1,1.5) to (2.5 + #1,1.5) to (2.5 + #1, 1.4) to (2.1 + #1,1.4) to (2.1 + #1,-0.4) to (2.5 + #1,-0.4) to (2.5 + #1,-0.5) to (2 + #1, -0.5) to (2 + #1, 1.5) -- cycle
  (2 + #1,-0.9) to (2.5 + #1,-0.9) to (2.5 + #1, -1) to (2 + #1, -1) -- cycle;
  %bottom

  \makeSparkplug{#1}

  %crankshaft
  \makeCounterweight{#1}{#5}

  %connecting rod
  \makeConnectingRod{#1}{#2}{#6}
}

%Píst
\newcommand{\makePiston}[2]{%
  \draw[thick] %piston body
  (0.05 + #1,#2) rectangle (1.95 + #1,1.5 + #2)
  (1 + #1, 0.5 + #2) circle (0.25);

  \draw[thick] %rings
  (0.05 + #1,1.1 + #2) to (1.95 + #1, 1.1 + #2)
  (0.05 + #1,1 + + #2) to (1.95 + #1, 1 + + #2)
  (0.05 + #1,0.9 + + #2) to (1.95 + #1, 0.9 + + #2);
}

%Válec
\newcommand{\makeCase}[1]{%
  \draw[pattern=north east lines, thick]%left
  (0 + #1,-1) to (0 + #1,3) to (0.2 + #1,3) 
  to (0.25 + #1, 3.1) to (-0.1 + #1, 3.1) to (-0.1 + #1, -1) -- cycle;
  \draw[pattern=north east lines, thick]%middle
  (0.8 + #1,3) to (1.2 + #1,3)
  to (1.25 + #1, 3.1) to (0.75 + #1, 3.1) -- cycle;
  \draw[pattern=north east lines, thick]%right
  (2 + #1,-1) to (2 + #1,3) to (1.8 + #1,3)
  to (1.75 + #1, 3.1) to (2.1 + #1, 3.1) to (2.1 + #1, -1) -- cycle;
}

%Ventil
\newcommand{\makeValve}[2]{%
  \draw[thick] %valve
  (0.25 + #1,3 - #2) to (0.75 + #1,3 - #2) to (0.7 + #1,3.1 - #2) to (0.3 + #1,3.1 - #2) -- cycle;
  \draw[thick] %valve rod
  (0.46 + #1,3.1 - #2) to (0.46 + #1, 3.6)
  (0.54 + #1, 3.6) to (0.54 + #1,3.1 - #2);
}

%Counterweight
\newcommand{\makeCounterweight}[2]{%
  \draw[thick] (1 + #1, -2.5) circle (0.2);

  \draw[thick, rotate around={#2:(1 + #1, -2.5)}]
  (1.3 + #1, -1.5) to (1.3 + #1, -2.75) to (2 + #1, -3)
  (0 + #1, -3) to (0.7 + #1, -2.75) to (0.7 + #1, -1.5)
  ([shift=(180:1)]1 + #1,-3) arc (180:360:1);

  %arrow
  \draw[->, thick, rotate around={#2:(1 + #1, -2.5)}] (1.4 + #1,-3.1) to[out=-90, in=-90, looseness=1.75](0.6 + #1,-3.1);
}

%Connecting rod
\newcommand{\makeConnectingRod}[3]{%
  \draw[thick]
  (1 + cos{#3} + #1, sin{#3} -2.5) circle (0.3);

  \draw[dotted, thick]%lines
  (0.8 + #1,0.5 + #2) to (0.8 + cos{#3} + #1, sin{#3} -2.5)
  (1.2 + #1,0.5 + #2) to (1.2 + cos{#3} + #1, sin{#3} -2.5);
}

%Sparkplug
\newcommand{\makeSparkplug}[1]{%
  \draw[thick]%sparker
  (0.9 + #1, 3) to (0.95 + #1, 2.95) to (1.05 + #1, 2.95) to (1.1 + #1, 3) -- cycle
  (0.975 + #1, 2.9) to (1.025 + #1, 2.9) to (1.025 + #1, 2.95);

  \draw[thick]%top
  (0.9 + #1, 3.1) to (0.9 + #1, 3.2) to (1.1 + #1, 3.2) to (1.1 + #1, 3.1) -- cycle
  (0.95 + #1, 3.2) to (0.95 + #1, 3.5) to (1.05 + #1, 3.5) to (1.05 + #1, 3.2) -- cycle;

  \draw[thin]
  (0.95 + #1, 3.2) to (1.05 + #1, 3.25)
  (0.95 + #1, 3.25) to (1.05 + #1, 3.3)
  (0.95 + #1, 3.3) to (1.05 + #1, 3.35)
  (0.95 + #1, 3.35) to (1.05 + #1, 3.4)
  (0.95 + #1, 3.4) to (1.05 + #1, 3.45)
  (0.95 + #1, 3.45) to (1.05 + #1, 3.5);
}

\usepackage{pgffor}
%1 - offset 2 - radius
\newcommand{\makeBlades}[4]{%
  \draw[thick, fill=#4] (#2,#1) rectangle (#2+#3, -#1);
  \foreach \x in {0,...,32} {
      \draw (#2,\x/32*2*#1-#1) to (#2+#3,\x/32*2*#1-#1);
  }
}

\newcommand{\makeDot}[2]{%
  \draw[fill=#2, draw=#2] (#1) circle (0.05);
}

% 1 - from; 2 - to; 3 - position; 4 - text
\newcommand{\makePointer}[4]{%
  \draw[->, thick] (#1) to (#2);
  \node[#3] at (#1){#4};
}

%Jednotky
\newcommand{\mmm}{m\textsuperscript{3}}
\newcommand{\ms}{m\(\cdot\)s\textsuperscript{-1}}
\newcommand{\mss}{m\(\cdot\)s\textsuperscript{-2}}
\newcommand{\kms}{km\(\cdot\)s\textsuperscript{-1}}
\newcommand{\kgs}{kg\(\cdot\)s\textsuperscript{-1}}
\newcommand{\rads}{rad\(\cdot\)s\textsuperscript{-1}}
\newcommand{\otmin}{otáčky\(\cdot\)min\textsuperscript{-1}}
\newcommand{\ots}{otáčky\(\cdot\)s\textsuperscript{-1}}
\newcommand{\mertepkap}{J\(\cdot\)kg\textsuperscript{-1}\(\cdot\)K}
\newcommand{\jmolk}{J\(\cdot\)mol\textsuperscript{-1}\(\cdot\)K\textsuperscript{-1}}
\newcommand{\nm}{N\(\cdot\)m}

%Lepší rovnice
\usepackage{amsmath}
\makeatletter
\renewenvironment{subequations}{
  \refstepcounter{equation}
  \protected@edef\theparentequation{\theequation}
  \setcounter{parentequation}{\value{equation}}
  \setcounter{equation}{0}
  \def\theequation{\theparentequation.\arabic{equation}}
  \ignorespaces
}{
  \setcounter{equation}{\value{parentequation}}
  \ignorespacesafterend
}
\makeatother

%Vlastní příkaz \code
\def\code#1{\texttt{#1}}

%Citace JÁ
\newcommand{\jaObr}{[obrázek autor]}
\newcommand{\jaDiag}{[diagram autor]}
\newcommand{\jaFoto}{[foto autor]}
\newcommand{\jaGraf}{[graf autor]}
\newcommand{\jaTab}{[tabulka autor]}

%Vlastní příkaz \changemargin
\def\changemargin#1#2{\list{}{\rightmargin#2\leftmargin#1}\item[]}
\let\endchangemargin=\endlist 

\makeatletter
\newcommand{\arbitraryref}[2]{
  \phantomsection
  \def\@currentlabel{#2}
  \label{#1}
}
\makeatother

%Velikosti nadpisů
\usepackage{anyfontsize}
\usepackage{titlesec}
\titleformat{\section}
  {\fontsize{18}{18}\bfseries}{\thesection}{1em}{}

\titleformat{\subsection}
  {\fontsize{16}{16}\bfseries}{\thesubsection}{1em}{}

\titleformat{\subsubsection}
  {\fontsize{14}{14}\bfseries}{\thesubsubsection}{1em}{}

%České názvy override
\usepackage[labelsep=endash]{caption}
\addto\captionsczech{\renewcommand{\figurename}{Obr.}}
\addto\captionsczech{\renewcommand{\subfigurename}{Obr.}}
\addto\captionsczech{\renewcommand{\tablename}{Tab.}}

% Popis obrázků
\makeatletter
\renewcommand\thesubfigure{\thefigure.\arabic{subfigure}}
\captionsetup[subfigure]{labelformat=simple, labelsep=colon}
\renewcommand\p@subfigure{}
\makeatother

% Popist tabulek
\captionsetup[table]{labelformat=simple, labelsep=colon}

% Vlevo a vpravo
\begin{comment}
\begingroup
\makeatletter
\renewcommand\thesubfigure{\thefigure~--~\@nameuse{subfiglabel@\alph{subfigure}}}
\newcommand{\subfiglabel@a}{vlevo}
\newcommand{\subfiglabel@b}{vpravo}
\captionsetup[subfigure]{labelformat=simple, labelsep=colon}
\renewcommand\p@subfigure{}
\makeatother

...

\endgroup
\end{comment}
\newcommand{\obrref}[1]{Obr. \ref{#1}}
\newcommand{\rvref}[1]{(\ref{#1})}
\newcommand{\scref}[1]{\nameref{#1} (\ref{#1})}
\newcommand{\tabref}[1]{Tab. \ref{#1}}
\newcommand{\animprilohy}[1]{Animace je dostupná ke stažení v přílohách zde: \ref{#1}.}

% Odstavce a aby tam nebylo odsazení
\newcommand{\odst}{\vspace{\baselineskip}\\}
\setlength{\parindent}{0pt}

%soubory
\usepackage{subfiles}

\begin{document}

%TC:ignore

\pagestyle{empty}
{\centering
\fontsize{18}{0}\textbf{STŘEDOŠKOLSKÁ ODBORNÁ ČINNOST}\\
\vspace{0.25cm}
\fontsize{14}{0}\textbf{Obor č. 12: Tvorba učebních pomůcek, didaktická technologie}\\
\vspace{8cm}
\fontsize{20}{0}\textbf{3D modely a výukové pomůcky na téma tepelné motory}\\
\vfill{}

\begin{tabularx}{1\textwidth} { 
    >{\raggedright\arraybackslash}X 
    >{\centering\arraybackslash}X 
    >{\raggedleft\arraybackslash}X}
    \fontsize{16}{0}\textbf{Hlavní město Praha} & 
    \fontsize{16}{0}\textbf{Leon Kubota} & 
    \fontsize{16}{0}\textbf{Praha, 2025}
\end{tabularx}}

\newpage
{\centering
\fontsize{18}{0}\textbf{STŘEDOŠKOLSKÁ ODBORNÁ ČINNOST}\\
\vspace{0.25cm}
\fontsize{14}{0}\textbf{Obor č. 12: Tvorba učebních pomůcek, didaktická technologie}\\
\vspace{7.5cm}
\fontsize{20}{0}\textbf{3D modely a výukové pomůcky na téma tepelné motory}\\
\vspace{1.5cm}
\fontsize{20}{0}\textbf{3D Models and Educational Aids on the Topic of Heat Engines}\\
\vspace{0.25cm}
\vfill{}
\begin{minipage}{0.8\textwidth}
      \begin{flushleft}
        \fontsize{16}{20}\selectfont{\textbf{Autor:} Leon Kubota}\\
        \fontsize{16}{20}\selectfont{\textbf{Škola:} Gymnázium, Praha 6, Arabská 14}\\
        \fontsize{16}{20}\selectfont{\textbf{Kraj:} Hlavní město Praha}\\
        \fontsize{16}{20}\selectfont{\textbf{Konzultant:} Mgr. Jana Urzová, Ph.D.}\\
        \fontsize{12}{20}\selectfont{Praha, 2025}\\
    \end{flushleft}
\end{minipage}%
\begin{minipage}{0.2\textwidth}
    \begin{figure}[H]
        \includegraphics[scale = 0.4]{assets/images/logo.png}
    \end{figure}
\end{minipage}
}

\newpage

\section*{Prohlášení}
{Prohlašuji, že jsem svou práci SOČ vypracoval samostatně a použil jsem pouze prameny a literaturu uvedené v seznamu bibliografických záznamů.}\odst
{Prohlašuji, že tištěná verze a elektronická verze soutěžní práce SOČ jsou shodné.}\odst
{Nemám závažný důvod proti zpřístupňování této práce v souladu se zákonem č. 121/2000 Sb., o právu autorském, o právech souvisejících s právem autorským a o změně některých zákonů (autorský zákon) ve znění pozdějších předpisů.}\par
\vspace{2cm}
\noindent{V}
\noindent\rule{3cm}{0.4pt}
\noindent{dne}
\noindent\rule{3cm}{0.4pt}
\hfill
\noindent{Podpis:}
\noindent\rule{5cm}{0.4pt}

\newpage

\section*{Poděkování}
{Rád bych poděkoval Mgr. Janě Urzové za vedení práce, dobré rady a možnost praktického vyzkoušení mých materiálů. Dále bych chtěl poděkovat  Mgr. Janu Tuzarovi za vedení původní ročníkové práce. Za zapůjčení mikrofonu jsem vděčný Gymnáziu Arabská. Děkuji také mé rodině a přátelům za poskytnutí zpětné vazby k materiálům, které jsem jim příliš často ukazoval.}

\newpage

\section*{Anotace}
{Tato práce se zabývá tvorbou učebních pomůcek pro výuku tepelných motorů. Mezi tyto pomůcky patří zejména animované video obsahující 3D animace parních, spalovacích a reaktivních motorů. Modely těchto motorů jsou založeny na skutečných konstrukcích. V práci jsou také obsaženy prezentace určené pro základní školy a obsáhlejší prezentace pro 2. ročník středních škol. Znalosti žáků mohou učitelé vyzkoušet pomocí obsažených pracovních listů. Materiály byly vyzkoušeny ve výuce studentů druhého ročníku školy Gymnázium, Praha 6, Arabská 14.}

\section*{Klíčová slova}
{Tepelné motory; výukové video; 3D modely; 3D animace; učební pomůcky}

\section*{Abstract}
{This thesis is about the creation of educational tools for teaching about heat engines. Primarly, these tools include an animated video containing 3D animations of steam, combustion and reaction engines. The 3D models of these engines are based on real-life designs. The work also contains presentations made for primary schools and more advanced presentations for the second year of high schools. Teachers may assess the knowledge of students with the use of the included worksheets. The materials were piloted in a class of the second year students at Gymnázium,Prague 6, Arabská 16.}

\section*{Keywords}
{Heat engines; educational video; 3D models; 3D animation; educational aids}

\newpage

\tableofcontents

%TC:endignore

\newpage
\pagestyle{plain}

\section{Úvod}
{Tepelné motory jsou nedílnou součástí moderního světa: od sekaček a automobilů po obrovské jaderné elektrárny a vesmírné rakety. Narazíme na ně na každém rohu, prakticky ještě mnohem častěji. Proto výuku o nich považuji za důležitou. Touto prací navazuji na můj ročníkový projekt ,,Tepelné motory”. Po jeho dokončení jsem se rozhodl pokračovat do soutěže SOČ a vytvořit ještě lepší učební pomůcky.}\odst
{Výukové materiály na téma tepelných motorů již samozřejmě existují, a to anglické (\href{https://www.slideshare.net/slideshow/heat-engine-41517317/41517317}{Heat Engines}) i české (\href{https://zs-nucice.cz/UserFiles/File/eu_new_361-400/VY_32_INOVACE_374.pdf}{Tepelné motory - úvod - prezentace})}\odst
{V první kapitole je popsán ideální plyn, ten je pro tepelné motory velmi důležitý. Dále jsou v ní popsány základní informace o tepelných motorech. V následujících kapitolách jsou stručně popsány tepelné motory a to parní, spalovací a reaktivní. Krátce se zabývají historií, fungováním a využitím dané skupiny motorů. Na konci každé z těchto kapitol jsou uvedeny nejdůležitější rovnice týkající se těchto motorů. Tyto kapitoly slouží jako podklad pro praktickou část.}\odst
{Pomocí softwaru \code{Blender}, který používám již pátým rokem, jsem vytvořil 3D modely a ani\-ma\-ce různých tepelných motorů. Ty jsem využil v několikaminutovém animovaném videu, které zábavnou formou vykládá látku tepelných motorů. Dále práce obsahuje prezentace pro žáky základních i středních škol. Pro vyzkoušení znalostí žáků jsem vytvořil několik pracovních listů. Tyto pomůcky jsem vyzkoušel na studentech druhého ročníku školy Gymnázium Arabská.}\odst
{Tepelné motory jsou zajímavé ale také velice obsáhlé téma, které není možné vysvětlit v několika hodinách fyziky. Mým cílem bylo vytvořit pomůcky, které žákovy přiblíží základy těchto strojů a současně zaujmou natolik, že se někteří z nich začnou zajímat o tepelné motory ve svém volném čase.}

\section*{\textcolor{red}{Konkurence}}
\textcolor{red}{Dočasné zdroje co zakomponuju do úvodu.}
\subsubsection*{\textcolor{red}{Animace}}
\href{https://en.wikipedia.org/wiki/Steam_engine#/media/File:Steam_engine_in_action.gif}{\textcolor{red}{Parní stroj;}}
\href{https://www.tlv.com/sites/default/files/tlv_assets/ja/steam_story/images/0611syuruitoyouto/pafs-generator-turbine_EN.gif}{\textcolor{red}{Parní turbína;}}
\href{URL}{text}
\href{https://en.wikipedia.org/wiki/Internal_combustion_engine#/media/File:4StrokeEngine_Ortho_3D_Small.gif}{\textcolor{red}{Čtyřdobý zážehový motor;}}
\subsubsection*{\textcolor{red}{Videa}}
\href{https://www.youtube.com/watch?v=dR1pyp3q9Ko}{\textcolor{red}{Fyzika na dálku - 8. ročník - Tepelné motory}}\\
\href{https://www.youtube.com/watch?v=k9DhdvbmRiw}{\textcolor{red}{Tepelné motory - fyzika 8 ZŠ}}
\subsubsection*{\textcolor{red}{Prezentace}}
\href{https://zs-nucice.cz/UserFiles/File/eu_new_361-400/VY_32_INOVACE_374.pdf}{\textcolor{red}{Tepelné motory -- úvod -- prezentace}}\\
\href{https://view.officeapps.live.com/op/view.aspx?src=http://dumy.cz/nahled/73077}{\textcolor{red}{Tepelné motory - prezentace}}
\subsubsection*{\textcolor{red}{Pracovní listy}}
\href{https://www.google.com/url?sa=i&url=https\%3A\%2F\%2Fslideplayer.cz\%2Fslide\%2F12677857\%2F&psig=AOvVaw0fGM8NbBvM9WlZNO6TqO0o&ust=1741690312947000&source=images&cd=vfe&opi=89978449&ved=0CAMQjB1qFwoTCMCD2Zes_4sDFQAAAAAdAAAAABBP}{\textcolor{red}{Tepelné motory - opakování}}\\
\href{https://view.officeapps.live.com/op/view.aspx?src=http://dumy.cz/nahled/117673}{\textcolor{red}{Spalovací motory - pracovní list}}\\
\href{https://www.soshlinky.cz/documents/uploads/71\%20Motory.xlsx}{\textcolor{red}{Motory - pracovní list}}

\newpage

\section{Teorie tepelných motorů}
{Tepelný motor je stroj, který přeměňuje tepelnou energii na mechanickou práci. Tuto energii mu dodáváme plynnou pracovní látkou.}
\subsection{Ideální plyn}
{Pro porozumění tepelným motorům je nezbytné znát ideální plyn. Zákon ideálních plynů popisuje vztahy mezi tlakem, teplotou a objemem. Ve skutečnosti je tvořen třemi zákony: izochorickým Charlesovým zákonem \rvref{rv:Charles}, izotermickým Boylovým zákonem \rvref{rv:Boyle} a izobarickým Gay-Lussacovým zákonem \rvref{rv:GayLussac}.}
\cite{SP:IdealGasBehaviour}\odst
\begin{subequations}
    \begin{minipage}{0.3\textwidth}
        \begin{equation}\label{rv:Charles}
            \frac{p}{T}=konst.
        \end{equation}
    \end{minipage}
    \hfill
    \begin{minipage}{0.3\textwidth}
        \begin{equation}\label{rv:Boyle}
            pV=konst.
        \end{equation}
    \end{minipage}
    \hfill
    \begin{minipage}{0.3\textwidth}
        \begin{equation}\label{rv:GayLussac}
            \frac{V}{T}=konst.
        \end{equation}
    \end{minipage}\odst
{\(p\) je tlak [Pa]}\\
{\(V\) je objem [\mmm]}\\
{\(T\) je teplota [K]}\odst
{Spojením těchto tří zákonů získáme zákon ideálního plynu, ten lze zapsat stavovou rovnicí ideálního plynu \rvref{rv:stavovaRovnice}.}
\cite{SP:IdealGasBehaviour}\\
    \begin{equation}\label{rv:stavovaRovnice}
        pV=nRT
    \end{equation}
\end{subequations}
{\(n\) je látkové množství [mol]}\\
{\(R\) je univerzální plynová konstanta (8,31 \jmolk)[\jmolk]}\odst
{Ideální plyn se však velmi liší od skutečného plynu, velikost molekul považujeme za zanedbatelnou a stejnou u všech molekul, předpokládáme náhodný pohyb dle Newtonových zákonů a zanedbáváme ztráty energie při kolizích. Ze stavové rovnice však plyne velmi důležitý poznatek: při zahřátí se plyn rozpíná. Toho využívají tepelné motory pro konání práce.}
\cite{SP:IdealGasBehaviour}
\subsection{Účinnost a práce}\label{sc:SkutecnaUcinnost}
{Účinnost tepelných motorů je omezena druhým termodynamickým zákonem. Účinnost tepelného motoru je podíl užitečné mechanické práce a dodaného tepla, viz rovnice \rvref{rv:ucinnost1}.}
\cite{NCEPU:ThermalEfficiencyForHeatEngines}
\begin{subequations}
    \begin{equation}\label{rv:ucinnost1}
        \eta=\frac{W}{Q_d}
    \end{equation}
{\(\eta\) je účinnost [-]}\\
{\(W\) je práce [J]}\\
{\(Q_d\) je dodané teplo [J]}\odst
{Práci můžeme vypočítat pomocí rovnice \rvref{rv:prace}.}
\cite{NCEPU:ThermalEfficiencyForHeatEngines}
    \begin{equation}\label{rv:prace}
        W=Q_d-Q_o
    \end{equation}
{\(Q_o\) je odevzdané teplo [J]}\\
\end{subequations}

\newpage

\section{Parní motory}
{První skupinou motorů, které popíši, jsou parní motory. Jak již plyne z názvu, parní motory jsou poháněny párou. Parní stroje umožnily průmyslovou revoluci a celkově posunuly technologie lidstva kupředu, zatímco parní turbíny jsou v současnosti velmi důležité pro výrobu elektřiny.}

\subsection{Historie parních motorů}
{První praktický parní stroj vytvořil Denis Papin na konci 17. století, jeho parní stroj pracoval s pístem ve válci. Když se válec zahřál, voda v něm se přeměnila v páru a zdvihla píst, při schlazení pára kondenzovala a stáhla píst zpět. Tento parní stroj zdokonalili T. Savery a T. Newcomen přidáním ventilu a bojleru. Dalšího zdokonalení se parní stroj dočkal na konci 18. století, kdy James Watt přidal kondenzátor a klikový mechanismus, aby mohl tento motor vykonávat otáčivý pohyb (doposud vykonával pouze kmitavý pohyb). Dále zdokonalil parní stroj tím, že páru přiváděl střídavě na obě stranu válce. Tím se výrazně zvýšil výkon i účinnost. Parní stroje hrály klíčovou roli při průmyslové revoluci: pomáhaly v dolech a továrnách, poháněly lokomotivy a lodě. Díky parním motorům předíváme 19. století ,,Století páry".}
\cite{st:parniStroj}\odst
{První moderní parní turbínu vytvořil Charles Parsons v roce 1884, sedm let poté ji využil pro výrobu elektřiny. Později se začaly parní turbíny využívat i pro pohon lodí a nabyly obrovských výkonů, dodnes pohání největší lodě a vytváří elektřinu v tepelných elektrárnách.}
\cite{SA:SteamTurbines}

\subsection{Fungování parních motorů}
{Parní motory pracují s párou tvořenou vně motoru, řadíme je tedy mezi motory s vnějším spalováním. Obecně lze říci, že pára koná práci tlačením na nějaké součástky. Toho však docilují obě skupiny parních motorů zcela odlišně. Parní stroje pracují s písty, parní turbíny přeměňují energii páry v užitečnou práci pomocí lopatek.}

\subsubsection{Parní stroj}\label{sc:ParniStroj}
{Parní stroj má píst, který se posouvá a vrací uvnitř válce. Píst je posouván tlakem páry, tu do válce přivádí ventil, použitá pára je z válce odváděna výfukovou dírou. Na píst je přes pístovou tyč napojena ojnice, která přes klikový mechanismus roztáčí hnanou hřídel. Parní stroj však nedodává točivý moment spojitě, proti klepání je na hnané hřídeli připojeno velké závaží zvané setrvačník, které zajišťuje hladký chod parního stroje.}
\cite{st:parniStroj}\cite{vutb:parniStroj}\odst
{Existuje řada parních motorů, které lze dělit různými způsoby, například dělení dle polohy válců. Nejběžnější jsou ležaté motory, mají velký výkon a snadno se obsluhují, jejich nevýhodou je náročnost na prostor. Stojaté motory jsou poněkud skladnější a dosahují vysokých otáček (až 3500 za minutu) a jsou často víceválcové, zato jsou složitější. Dále můžeme parní stroje dělit na plnotlakové a expanzní, u plnotlakových motorů se válec naplno naplní párou, u expanzních se pára ve válci rozpíná. Většina parních strojů je expanzních, jelikož spotřebují méně páry.}
\cite{st:parniStroj}\cite{vutb:parniStroj}

\newpage

\subsubsection{Parní turbína}\label{sc:ParniTurbina}
{Parní turbíny se od parních strojů velmi liší. Energii páry převádí na mechanickou práci pomocí rotorů, které jsou připojeny na hřídel a kvůli páře která přes ně proudí se točí. Aby se pára jen netočila kolem hřídele, následuje za každým rotorem stator. Ten vede páru zpět správným směrem, aby následující rotor mohl opět účinně točit hřídelí.}
\cite{LESICS:WorkingOfSteamTurbine}\odst
{Teoreticky rozlišujeme dvě hlavní skupiny turbín: impulsní turbíny a reakční turbíny. Impulsní turbíny využívají pouze kinetickou energii páry, ta má tedy stejnou teplotu a tlak před i po průchodu rotorem. Naopak reakční turbína využívá pouze vnitřní energii a žádnou kinetickou energii. Skutečné turbíny jsou však kombinací těchto dvou, mohou být například 50 \% reakční a 50~\% impulsní. Na \obrref{fig:parniTurbina} je vyobrazena parní turbína.}
\cite{LESICS:WorkingOfSteamTurbine}

\vspace{1cm}
\begin{figure}[H]
    \centering
    \begin{tikzpicture}[scale=1.75]
        %hřídel
        \draw[fill=white,pattern=north east lines, thick] (0.5,0.05) to (3.9+2.2+0.4,0.05) to (3.9+2.2+0.4,-0.05) to (0.5,-0.05) to cycle;

        %proudy
        \draw[->, blue, path fading=east, postaction={draw, red, path fading=west}] (3.425-0.15, 2) to (3.425-0.15,0.8) to [out=-90,in=-10, draw=blue](2,0.5) to (0.5,0.8);
        \draw[->, blue, path fading=east, postaction={draw, red, path fading=west}] (3.425, 2) to (3.425, 0) to [out=-90, in=8](2,-0.65) to (0.5,-0.8);
        \draw[->, blue, path fading=west, postaction={draw, red, path fading=east}] (3.575, 2) to (3.575, 0) to [out=-90, in=172](5,-0.65) to (6.5,-0.8);
        \draw[->, blue, path fading=west, postaction={draw, red, path fading=east}] (3.575+0.15, 2) to (3.575+0.15,0.8) to [out=-90, in=190](5,0.5) to (6.5, 0.8);

        %rotory
        \makeBlades{1.5}{0.9}{0.16}{white}
        \makeBlades{1.3}{1.35}{0.15}{white}
        \makeBlades{1.15}{1.75}{0.14}{white}
        \makeBlades{1}{2.1}{0.13}{white}
        \makeBlades{0.9}{2.45}{0.12}{white}
        \makeBlades{0.85}{2.75}{0.11}{white}
        \makeBlades{0.8}{3}{0.1}{white}

        %statory
        \makeBlades{1.5}{1.25}{0.05}{white}
        \makeBlades{1.35}{1.65}{0.05}{white}
        \makeBlades{1.2}{2}{0.05}{white}
        \makeBlades{1.075}{2.35}{0.05}{white}
        \makeBlades{1}{2.65}{0.05}{white}
        \makeBlades{0.95}{2.9}{0.05}{white}

        \begin{scope}[xscale=-1, shift={(-7,0)}]
            %rotory
            \makeBlades{1.5}{0.9}{0.16}{white}
            \makeBlades{1.3}{1.35}{0.15}{white}
            \makeBlades{1.15}{1.75}{0.14}{white}
            \makeBlades{1}{2.1}{0.13}{white}
            \makeBlades{0.9}{2.45}{0.12}{white}
            \makeBlades{0.85}{2.75}{0.11}{white}
            \makeBlades{0.8}{3}{0.1}{white}

            %statory
            \makeBlades{1.5}{1.25}{0.05}{white}
            \makeBlades{1.35}{1.65}{0.05}{white}
            \makeBlades{1.2}{2}{0.05}{white}
            \makeBlades{1.075}{2.35}{0.05}{white}
            \makeBlades{1}{2.65}{0.05}{white}
            \makeBlades{0.95}{2.9}{0.05}{white}

            %skříň vrchní
            \draw[fill=white] (0.5,1.55) to (0.9+0.16, 1.55) to [out=-20, in=175](3.1,0.9) to [out=0,in=-90](3.2,1) to (3.2,1.8) to (3.1,1.8) to (3.1,1) to [out=175,in=-20](0.9+0.16,1.65) to (0.5,1.65) to cycle; 
            \draw[pattern=north east lines, thick] (0.5,1.55) to (0.9+0.16, 1.55) to [out=-20, in=175](3.1,0.9) to [out=0,in=-90](3.2,1) to (3.2,1.8) to (3.1,1.8) to (3.1,1) to [out=175,in=-20](0.9+0.16,1.65) to (0.5,1.65) to cycle; 
        \end{scope}

        %skříň vrchní
        \draw[fill=white] (0.5,1.55) to (0.9+0.16, 1.55) to [out=-20, in=175](3.1,0.9) to [out=0,in=-90](3.2,1) to (3.2,1.8) to (3.1,1.8) to (3.1,1) to [out=175,in=-20](0.9+0.16,1.65) to (0.5,1.65) to cycle;
        \draw[pattern=north east lines, thick] (0.5,1.55) to (0.9+0.16, 1.55) to [out=-20, in=175](3.1,0.9) to [out=0,in=-90](3.2,1) to (3.2,1.8) to (3.1,1.8) to (3.1,1) to [out=175,in=-20](0.9+0.16,1.65) to (0.5,1.65) to cycle;
        
        %skříň spodní
        \draw[fill=white] (0.5,-1.55) to (0.9+0.16, -1.55) to [out=20, in=-175](3.1,-0.9) to (3.9, -0.9) to [in=160, out=-5](3.9+2.2-0.16,-1.55) to (3.9+2.2+0.4,-1.55) to (3.9+2.2+0.4,-1.65) to (3.9+2.2-0.16,-1.65) to [out=160,in=-5](3.9,-1) to (3.1,-1) to [in=20, out=-175](0.9+0.16,-1.65) to (0.5, -1.65) to cycle;
        \draw[pattern=north east lines, thick] (0.5,-1.55) to (0.9+0.16, -1.55) to [out=20, in=-175](3.1,-0.9) to (3.9, -0.9) to [in=160, out=-5](3.9+2.2-0.16,-1.55) to (3.9+2.2+0.4,-1.55) to (3.9+2.2+0.4,-1.65) to (3.9+2.2-0.16,-1.65) to [out=160,in=-5](3.9,-1) to (3.1,-1) to [in=20, out=-175](0.9+0.16,-1.65) to (0.5, -1.65) to cycle;

        %popisky
        \makePointer{2.7,1.5}{3.1,1.4}{left}{skříň}
        \makePointer{6.5,0.3}{6.4,0.05}{above}{hřídel}
        \makePointer{4,1.5}{3.5,1.4}{right}{přívod páry}
        \makePointer{3.2,-1.3}{3.05,-0.8}{below}{rotor}
        \makePointer{4.2,-1.3}{4.95,-1.05}{below}{stator}
    \end{tikzpicture}
    \caption{Ukázka hlavních částí parní turbíny \jaDiag}
    \label{fig:parniTurbina}
\end{figure}

\newpage

\section{Spalovací motory}
{Spalovací motory jsou nejrozšířenějšími tepelnými motory, využívají se v nejrůznějších zařízeních od motorových pil přes běžné automobily až po obrovské nákladní lodě. Využívají rozpínání plynu při výbuchu, ten tlačí píst a tím koná práci.}

\subsection{Historie spalovacích motorů}\label{sc:HistorieSpalovacichMotoru}
{První zmínka o spalovacích motorech pochází z roku 1680, kdy fyzik Huygens navrhl spalovací motor poháněný střelným prachem. O 114 let později vyrobil dle plánů Johna Barbera Robert Street první spalovací motor, ten však byl prakticky nepoužitelný. Roku 1858 vytvořil inženýr Étinne Lenoir spalovací motor na svítiplyn, ten byl vůbec první úspěšný spalovací motor; byl užíván v průmyslu.}
\cite{vutb:vyzkumneSpalovaciMotory}\cite{ANNALS:TheHistoryOfTheInternalCombustionEngine}\odst
{Němec Nikolaus Otto po zdokonalení Lenoirova motoru roku 1876 vynalezl první čtyřdobý motor. V tuto dobu byl benzín považován za bezcennou složku ropy, Ognjen Stratonović ho však využil k pohonu prvního benzinového motoru. Později zdokonalil Gottlieb Daimler čtyřdobý motor přidáním karburátoru a opatřil jím automobil. První vznětový motor vyrobil Rudolf Diesel, ten se rychle ujal v těžkém průmyslu.}
\cite{vutb:vyzkumneSpalovaciMotory}\cite{ANNALS:TheHistoryOfTheInternalCombustionEngine}\odst
{Spalovací motory se od dob Otty a Daimlera nepřestaly vyvíjet, byli zdokonaleny a jejich výkon a účinnost stoupaly. V současnosti jsou prozkoumávány další paliva jako například zemní plyn či vodík s cílem snížení emisí.}
\cite{vutb:vyzkumneSpalovaciMotory}

\subsection{Fungování spalovacích motorů}
{Spalovací motory využívají rozpínání plynu při zahřátí. Aby plyn zahřály, spalují v něm palivo. Tuto energii převádí na mechanickou práci pomocí pístu, který přes klikové ústrojí točí hřídelí. Palivo je spalováno nespojitě v cyklech.}\odst
{Spalovací motory můžeme dělit dle způsobu zapálení plynu na zážehové a vznětové. Dále je lze dělit dle počtu dob, které se cyklicky opakují. V této kapitole popíši nejdůležitější tři druhy: čtyřdobé zážehové, čtyřdobé vznětové a dvoudobé zážehové.}
\cite{vutb:vyzkumneSpalovaciMotory}

\newpage

\subsubsection{Čtyřdobý zážehový motor}
{Čtyřdobý zážehový motor je v nejběžnější spalovací motor, pracuje ve čtyř dobách: sání, komprese, expanze a výfuk. Tyto doby proběhnou za dvě otáčky klikové hřídele. Tyto doby jsou popsány níže, graficky jsou znázorněny na \obrref{obr:PracovniCyklusCtyrdobehoZazehovehoMotoru}.}
\cite{ZUP:CtyrdobyAtmosferickyZazehovyMotor}
\begin{enumerate}
    \item {Sání: píst se pohybuje směrem dolů, tím zvětšuje pracovní prostor a tlak klesá. Skrz otevřený sací ventil do pracovního prostoru proudí vzduch smíchaný s palivem.}
    \item {Komprese: píst se pohybuje směrem nahoru, oba ventily jsou uzavřeny. Tlak a teplota rostou, tím se tvoří směs, která je velmi snadno zažehnutelná.}
    \item {Výbuch: elektrický výboj svíčky zažehne palivo, to se velice rychle spálí a s tím přichází vysoký tlak, který tlačí píst směrem dolů. V této době je konána užitečná práce.}
    \item {Výfuk: po otevření výfukového ventilu proudí spaliny ven. Zbylé spaliny vytlačí píst při jeho pohybu směrem nahoru. Cyklus se opakuje.}
\end{enumerate}
{V současnosti se velmi dbá na ekologii tepelných motorů. Výrobci se snaží vyrábět motory, které jsou účinné, tvoří méně škodlivých emisí ale zároveň dosahují velkých výkonů. K tomu může vést vyšší kompresní poměr, komprese vzduchu vně motoru či snížení tepelných a mechanických ztrát.}
\cite{ZUP:CtyrdobyAtmosferickyZazehovyMotor}\odst
{Čtyřdobé zážehové motory jsou hojně využívány nejen v automobilovém průmyslu, mezi jejich ostatní využití patří například letadla a motocykly.}

\begin{figure}[H]
    \hspace{-0.33cm}
    \begin{tikzpicture}[scale=0.75]

        %popisky
        \makePointer{-0.5,3}{0.2,2.85}{left}{sací ventil}
        \makePointer{-0.5,0.3}{0.4,0}{left}{píst}
        \makePointer{-0.5,-2.8}{0.2,-2.5}{left}{kliková hřídel}
        \makePointer{-0.5,2}{-0.1,1.9}{left}{stěna válce}
        \makePointer{11.5,-2.6}{12.6,-2}{left}{ojnice}
        \makePointer{14.5,3.2}{13.8,2.8}{right}{výfukový ventil}
        \makePointer{10.2,3.5}{9.1,3.3}{right}{svíčka}

        %sání
        \node[above] at (1,3.75 + 0.5){1. sání};
        \node[above] at (5,3.6 + 0.5){2. komprese};
        \node[above] at (9,3.6 + 0.5){3. výbuch};
        \node[above] at (13,3.6 + 0.5){4. výfuk};

        \makeEngineFourStrokeGasolineDiagram{0}{-0.5}{1}{0}{-135}{-45}
        \draw[->, thick] (2.5,1) -- (2.5,-0.5);

        %tečky
        \makeDot{0.675,3}{blue}
        \makeDot{0.8,2.5}{blue}
        \makeDot{0.2,2}{blue}
        \makeDot{1.4,2.7}{blue}
        \makeDot{1.5,1.3}{blue}
        \makeDot{0.6,1.45}{blue}
        \makeDot{1.8,2.35}{blue}
        \makeDot{1.3,1.9}{blue}
        
        %komprese
        \makeEngineFourStrokeGasolineDiagram{4}{0.5}{0}{0}{45}{135}
        \draw[->, thick]  (6.5,0.5) -- (6.5,2);

        %tečky
        \makeDot{4.5,2.9}{orange}
        \makeDot{4.7,2.2}{orange}
        \makeDot{4.2,2.3}{orange}
        \makeDot{5.7,2.7}{orange}
        \makeDot{5.4,2.25}{orange}
        \makeDot{4.8,2.6}{orange}
        \makeDot{5.8,2.35}{orange}
        \makeDot{5.3,2.8}{orange}

        %výbuch
        \makeEngineFourStrokeGasolineDiagram{8}{0.5}{0}{0}{-45}{45}
        \draw[->, thick]  (10.5,2) -- (10.5,0.5);

        %čáry
        \foreach \angle in {0,-15,...,-180} {
            \draw[thick, yellow] ({9 + cos(\angle)*0.1}, {2.9 + sin(\angle)*0.1}) -- ({9 + cos(\angle)*0.7}, {2.9 + sin(\angle)*0.7});
        }

        %tečky
        \makeDot{8.5,2.55}{red}
        \makeDot{9,2.2}{red}
        \makeDot{8.2,2.3}{red}
        \makeDot{9.7,2.4}{red}
        \makeDot{9.4,2.25}{red}
        \makeDot{8.7,2.3}{red}
        \makeDot{9.8,2.25}{red}
        \makeDot{9.3,2.5}{red}

        %výfuk
        \makeEngineFourStrokeGasolineDiagram{12}{-0.5}{0}{1}{135}{-135}
        \draw[->, thick]  (14.5,-0.5) -- (14.5,1);

        %tečky
        \makeDot{13.325,3}{black}
        \makeDot{13.2,2.5}{black}
        \makeDot{13.8,2}{black}
        \makeDot{12.6,2.7}{black}
        \makeDot{12.5,1.3}{black}
        \makeDot{13.4,1.45}{black}
        \makeDot{12.2,2.35}{black}
        \makeDot{12.7,1.9}{black}

    \end{tikzpicture}
    
    \caption{Pracovní cyklus čtyřdobého zážehového motoru \jaDiag}
    \label{obr:PracovniCyklusCtyrdobehoZazehovehoMotoru}
\end{figure}

\newpage

\subsubsection{Čtyřdobý vznětový motor}
{Čtyřdobý vznětový motor spaluje naftu. Stejně jako čtyřdobý zážehový motor pracuje ve čtyřech dobách, liší se v době sání a výbuchu. Tento motor je graficky znázorněn na \obrref{obr:PracovniCyklusCtyrdobehoVznetovehoMotoru}.}
\cite{CVUT:NávrhKonstrukceVznetovehoMotoru}
\begin{enumerate}
    \item {Sání: píst se pohybuje směrem dolů a pracovní prostor nad ním se plní čistým vzduchem bez paliva.}
    \item {Komprese: píst se pohybuje směrem nahoru a oba ventily jsou zavřeny. Tlak a teplota rostou, konečná teplota se pohybuje mezi 400--600 °C.}
    \item {Výbuch: do žhavého vzduchu se vstříkne nafta, ta okamžitě vznítí. Vlivem vzniklého vysokého tlaku se píst pohybuje směrem dolů a koná užitečnou práci}
    \item {Výfuk: píst se pohybuje směrem nahoru a otevřeným výfukovým ventilem tlačí ven spaliny, cyklus se opakuje.}
\end{enumerate}
{Vznětové motory se uplatňují v některých osobních automobilech, častěji je najdeme v nákladních automobilech, ve vlakové a lodní dopravě a jako pohon zemědělských, důlních a stavebních zařízení.}
\cite{CVUT:NávrhKonstrukceVznetovehoMotoru}

\begin{figure}[H]
    \hspace{-0.33cm}
    \begin{tikzpicture}[scale=0.75]

        %popisky
        \makePointer{-0.5,3}{0.2,2.85}{left}{sací ventil}
        \makePointer{-0.5,0.3}{0.4,0}{left}{píst}
        \makePointer{-0.5,-2.8}{0.2,-2.5}{left}{kliková hřídel}
        \makePointer{-0.5,2}{-0.1,1.9}{left}{stěna válce}
        \makePointer{11.5,-2.6}{12.6,-2}{left}{ojnice}
        \makePointer{14.5,3.2}{13.8,2.8}{right}{výfukový ventil}
        \makePointer{10.2,3.5}{9.15,3.2}{right}{tryska}


        %sání
        \node[above] at (1,3.75 + 0.5){1. sání};
        \node[above] at (5,3.6 + 0.5){2. komprese};
        \node[above] at (9,3.6 + 0.5){3. výbuch};
        \node[above] at (13,3.6 + 0.5){4. výfuk};

        \makeEngineFourStrokeDieselDiagram{0}{-0.5}{1}{0}{-135}{-45}
        \draw[->, thick] (2.5,1) -- (2.5,-0.5);

        %tečky
        \makeDot{0.675,3}{blue}
        \makeDot{0.8,2.5}{blue}
        \makeDot{0.2,2}{blue}
        \makeDot{1.4,2.7}{blue}
        \makeDot{1.5,1.3}{blue}
        \makeDot{0.6,1.45}{blue}
        \makeDot{1.8,2.35}{blue}
        \makeDot{1.3,1.9}{blue}
        
        %komprese
        \makeEngineFourStrokeDieselDiagram{4}{0.5}{0}{0}{45}{135}
        \draw[->, thick]  (6.5,0.5) -- (6.5,2);

        %tečky
        \makeDot{4.5,2.9}{orange}
        \makeDot{4.7,2.2}{orange}
        \makeDot{4.2,2.3}{orange}
        \makeDot{5.7,2.7}{orange}
        \makeDot{5.4,2.25}{orange}
        \makeDot{4.8,2.6}{orange}
        \makeDot{5.8,2.35}{orange}
        \makeDot{5.3,2.8}{orange}

        %výbuch
        \makeEngineFourStrokeDieselDiagram{8}{0.5}{0}{0}{-45}{45}
        \draw[->, thick]  (10.5,2) -- (10.5,0.5);

        %čáry
        \foreach \angle in {0,-15,...,-180} {
            \draw[thick, yellow] ({9 + cos(\angle)*0.1}, {2.9 + sin(\angle)*0.1}) -- ({9 + cos(\angle)*0.7}, {2.9 + sin(\angle)*0.7});
        }

        %tečky
        \makeDot{8.5,2.55}{red}
        \makeDot{9,2.2}{red}
        \makeDot{8.2,2.3}{red}
        \makeDot{9.7,2.4}{red}
        \makeDot{9.4,2.25}{red}
        \makeDot{8.7,2.3}{red}
        \makeDot{9.8,2.25}{red}
        \makeDot{9.3,2.5}{red}

        %výfuk
        \makeEngineFourStrokeDieselDiagram{12}{-0.5}{0}{1}{135}{-135}
        \draw[->, thick]  (14.5,-0.5) -- (14.5,1);

        %tečky
        \makeDot{13.325,3}{black}
        \makeDot{13.2,2.5}{black}
        \makeDot{13.8,2}{black}
        \makeDot{12.6,2.7}{black}
        \makeDot{12.5,1.3}{black}
        \makeDot{13.4,1.45}{black}
        \makeDot{12.2,2.35}{black}
        \makeDot{12.7,1.9}{black}

    \end{tikzpicture}
    
    \caption{Pracovní cyklus čtyřdobého vznětového motoru \jaDiag}
    \label{obr:PracovniCyklusCtyrdobehoVznetovehoMotoru}
\end{figure}

\newpage

\subsubsection{Dvoudobý zážehový motor}
{Ve dvoudobém zážehovém motoru probíhají stejné čtyři děje jako ve čtyřdobých motorech ve dvou dobách. K tomu se využívá se prostor pod pístem. Doby dvoudobého motoru jsou popsány níže, graficky jsou zobrazeny na \obrref{obr:PracovniCyklusDvoudobehoVznetovehoMotoru}.}
\cite{VUTB:DvoudobyMotorProParagliding}
\begin{enumerate}
    \item {Sání a komprese: píst se pohybuje nahoru, pod něj je přiváděna směs vzuchu a paliva. Mezitím se směs nad pístem stlačuje.}
    \item {Výbuch a výfuk: směs nad pístem se zažehne a tlačí píst dolů. Tím se odkryje výfukový kanál, kterým začnou unikat spaliny. Poté se odhalí přepouštěcí kanál, který z prostoru pod pístem přivede novou směs nad píst.}
\end{enumerate}
{Dvoudobé motory jsou oproti čtyřdobým motorům jednodušší, přesto disponují velkého měrného výkonu. Užívají se v menších motocyklech a malé mechanizaci jako jsou například motorové pily nebo sekačky. Jsou však také méně účinné, tvoří více emisí a spalují olej, také nejsou jednoduše škálovatelné. V současnosti jsou nahrazovány malými čtyřdobými zážehovými motory.}
\cite{VUTB:DvoudobyMotorProParagliding}

\begin{figure}[H]
    \hspace{1.75cm}
    \centering
    \begin{tikzpicture}[scale=0.75]
        \makeDot{12.5,0}{white}

        %popisky
        \begin{scope}[shift={(-1,0)}]
            \makePointer{-1,3}{-0.1,2.85}{left}{stěna válce}
            \makePointer{-1,1}{-0.3,0.5}{left}{přepouštěcí kanál}
            \makePointer{-1,-2.8}{0,-2.5}{left}{kliková hřídel}
            \node[above] at (1,3.6 + 0.5){1. sání a komprese};
        \end{scope}
        
        \begin{scope}[shift={(1,0)}]
            \makePointer{7,1}{5.5,0.8}{right}{píst}
            \makePointer{7,3.2}{5.1,3.4}{right}{svíčka}
            \makePointer{7,2}{6.2,1.7}{right}{výfukový kanál}
            \makePointer{7,-0.6}{6.2,-0.7}{right}{sací kanál}
            \makePointer{3.7,-2.6}{4.6,-2}{left}{ojnice}
            \node[above] at (5,3.6 + 0.5){2. výbuch a výfuk};
        \end{scope}

        \begin{scope}[shift={(-1,0)}]
            \makeEngineTwoStrokeDiagram{0}{0.5}{1}{0}{-45}{45}

            %tečky sání
            \makeDot{0.675,-0.1}{blue}
            \makeDot{0.8,-0.7}{blue}
            \makeDot{0.2,0.1}{blue}
            \makeDot{2.1,-0.65}{blue}
            \makeDot{1.5,-0.3}{blue}
            \makeDot{-0.2,-0.2}{blue}
            \makeDot{1.8,0.1}{blue}
            \makeDot{1.3,0.2}{blue}

            %tečky komprese
            \makeDot{0.5,2.75}{orange}
            \makeDot{0.7,2.2}{orange}
            \makeDot{0.2,2.3}{orange}
            \makeDot{1.7,2.7}{orange}
            \makeDot{1.4,2.25}{orange}
            \makeDot{0.8,2.6}{orange}
            \makeDot{1.8,2.35}{orange}
            \makeDot{1.3,2.8}{orange}
        \end{scope}

        \begin{scope}[shift={(1,0)}]
            %sluníčko
            \foreach \angle in {0,-15,...,-180} {
                \draw[thick, yellow] ({5 + cos(\angle)*0.1}, {2.9 + sin(\angle)*0.1}) -- ({5 + cos(\angle)*0.7}, {2.9 + sin(\angle)*0.7});
            }
            
            %komprese
            \makeEngineTwoStrokeDiagram{4}{-0.5}{0}{0}{135}{225}

            %tečky komprese
            \makeDot{4.5,2.2}{red}
            \makeDot{4.4,1.9}{red}
            \makeDot{4.2,2.5}{red}
            \makeDot{4.8,1.75}{red}
            \makeDot{5.8,2.65}{red}
            \makeDot{5.3,2.1}{red}

            %tečky výfuk
            \makeDot{5.75,2}{black}
            \makeDot{6.05,1.75}{black}
            \makeDot{5.45,1.45}{black}

            %tečky sání
            \makeDot{4.35,1.3}{blue}
            \makeDot{3.8,1.15}{blue}
            \makeDot{3.65,0.55}{blue}
            \makeDot{3.75,0.05}{blue}
            \makeDot{5.55,-0.75}{blue}
        \end{scope}

    \end{tikzpicture}
    
    \caption{Pracovní cyklus dvoudobého vznětového motoru \jaDiag}
    \label{obr:PracovniCyklusDvoudobehoVznetovehoMotoru}
\end{figure}

\newpage

\subsection{Rovnice spalovacích motorů}
{V této kapitole jsou popsány rovnice týkající se práce, výkonu a kroutícího momentu spalovacích motorů.}

\subsubsection{Práce}
{Jak již bylo zmíněno v kapitole \ref{sc:SkutecnaUcinnost}, práci vypočítáme pomocí rovnice \rvref{rv:prace2}.}
\cite{NCEPU:ThermalEfficiencyForHeatEngines}

\begin{equation}\label{rv:prace2}
    W=Q_d-Q_o
\end{equation}

{\(W\) je práce (J)}\\
{\(Q_d\) je dodané teplo (J)}\\
{\(Q_o\) je odevzdané teplo (J)}\\
\podst

{Teplo \(Q_d\) je teplo dodané spalováním paliva. Odevzdané teplo \(Q_o\) je nevyužité teplo, které uniká například s výfukovými plyny či do chladiče motoru.}

\subsubsection{Výkon}
{Výkon udává množství práce, kterou dokáže motor vytvořit za určitý čas. Vypočítáme jej pomocí rovnice \rvref{rv:vykonStandard}.}
\cite{NASA:IdealOttoCycle}

\begin{subequations}
    \begin{equation}\label{rv:vykonStandard}
        P=\frac{W}{t}
    \end{equation}

{$P$ je výkon [W] (někdy se používá kůň, 1 kW $\approx$ 1,35 k)}\\
{$W$ je práce [J]}\\
{$t$ je čas, za který je práce konána [s]}\\
\podst

{Rovnice \rvref{rv:vykonStandard} obecně popisuje výkon, pro spalovací motory je však vhodnější využít rovnici \rvref{rv:vykon}.}
\cite{NASA:IdealOttoCycle}

    \begin{equation}\label{rv:vykon}
        P=W\cdot{\frac{ot}{n}}
    \end{equation}

{\(ot\) je počet otáček za sekundu [\ots]}\\
{$n$ je počet pracovních dob za otáčku hřídele [-]}\\
\podst

{Počet pracovních dob za otáčku hřídele $n$ se liší dle druhu motoru. Ve dvoudobých se $n$ rovná 1 (pracovní doba proběhne každou otáčku), ve čtyřdobých je $n$ 2 (pracovní doba proběhne každou druhou otáčku).}
\end{subequations}

\subsubsection{Kroutící moment}
{Kroutící moment (často také točivý moment) je síla, kterou se točí hřídel. Tuto sílu vypočítáme rovnicí \rvref{rv:tocivyMoment}.}
\cite{TO:TechnologieOprav}

\begin{equation}\label{rv:tocivyMoment}
    M_k=\vec{F}\cdot{r}
\end{equation}

{\(M_k\) je kroutící moment [\nm]}\\
{\(F\) je síla, kterou píst tlačí [N]}\\
{\(r\) je poloměr kružnice otáčení klikového hřídele [m]}

\newpage

\section{Reaktivní motory}
{Reaktivní motory fungují na principu Newtonova třetího pohybového zákona, ten je vysvětlen v~kapitole \ref{sc:AkceReakce}. Tyto motory urychlují plyn na vysoké rychlosti a využívají reakci vyvolanou touto akcí. Tato reakce tlačí motor vpřed.}

\subsection{Historie reaktivních motorů}
{Historie reaktivních motorů začala roku 1232, kdy Číňané využili raketový motor na střelný prach k zapálení mongolského tábořiště. O několik století později zaujaly rakety britského politika Williama Congreve, jeho rakety byly nasazeny v napoleonských válkách a ve válce USA za~nezávislost. Vystupují i v americké státní hymně (\textit{,,And the rockets' red glare''}).}
\cite{VUTB:NavrhRaketovehoMotoru}\odst
{Roku 1930 získal angličan Frank Whittle patent na proudový motor a o sedm let později jej vyzkoušel, první proudové letadlo však nevyrobili angličané. Předběhli je němečtí inženýři, kteří roku 1939 opatřili letoun motorem HeS 3B s tahem 4,4 kN.}
\cite{VUTB:PrehledTechnickychAspektuVyvojeLeteckychProudovychMotoru}\odst
{Před druhou světovou válkou vytvořil Robert Goddard první raketový motor na kapalné palivo, za války jej použil jako zbraň. Němci využili raketové motory v raketě V2. Tato technologicky vyspělá zbraň velmi posunula raketovou vědu, stála však mnoho nevinných životů.}
\cite{VUTB:NavrhRaketovehoMotoru}\odst
{Po válce se reaktivní motory rychle vyvíjely, britská firma Vickers za pomoci vědomostí získaných od Němců vyrobila dvouproudové motory. Reaktivní motory si později našli cestu i~mimo vojenská využití, prvním civilním letadlem s proudovým motorem se stalo DH-106 Comet. V~současnosti jsou reaktivní motory hojně využívány v letectví a kosmonautice.}
\cite{VUTB:PrehledTechnickychAspektuVyvojeLeteckychProudovychMotoru}

\subsection{Fungování reaktivních motorů}\label{sc:AkceReakce}
{Abychom porozuměli reaktivním motorům, musíme znát Newtonův třetí pohybový zákon: zákon akce a reakce. Jeho české znění je následující:}
\cite{MIT:NewtonsLawsOfMotion}
%TC:ignore
\begin{changemargin}{50px}{50px}
    \begin{center}
        \textit{,,Zákon III: Pro každou akci je vždy opačná a rovná reakce; jinak: vzájemné působení dvou těles je vždy stejně velké a míří na opačné strany. Cokoliv co táhne či tlačí na něco je stejně taženo či tlačeno. Pokud prstem zatlačíte na kámen, kámen také tlačí na prst.''}
        \cite{MIT:NewtonsLawsOfMotion}
    \end{center}
\end{changemargin}
%TC:endignore
{Reaktivní motory tohoto využívají; velkou rychlostí vypuzují plyn, který působí silou \(\vec{F}_{1}\) a vyvolává opačnou a sobě rovnou reakci \(\vec{F}_{2}\), ta tlačí motor vpřed, nazýváme ji tah. Tento vztah můžeme zapsat rovnicí (\ref{rv:akcereakce}).}
\cite{MIT:NewtonsLawsOfMotion}

\begin{equation} \label{rv:akcereakce}
    \vec{F}_{2}=-\vec{F}_{1}
\end{equation}

{\(\vec{F}_{2}\) je síla tahu [N]}\\
{\(\vec{F}_{1}\) je síla plynu [N]}\odst

{Dle konstrukce dělíme reaktivní motory na dvě skupiny: raketové a proudové motory.}
\cite{VUTB:PrehledTechnickychAspektuVyvojeLeteckychProudovychMotoru}

\newpage

\subsection{Raketové motory}
{Raketové motory za vysokých teplot a tlaků spalují palivo ve spalovací komoře. Spaliny unikají tryskou z motoru, čímž tvoří tah.}
\cite{VUTB:NavrhRaketovehoMotoru}\odst
{Raketové motory dělíme na raketové motory na tuhé palivo a na kapalné palivo.}
\cite{VUTB:NavrhRaketovehoMotoru}

\subsubsection{Raketové motory na tuhé palivo}
{Raketový motor na tuhé palivo (viz \obrref{fig:raketovyMotorPevny}) je nejjednodušším raketovým motorem, nemá žádné pohyblivé části. Jeho nejdůležitější částí je samotné pevné palivo, to je ve spalovací komoře zažehnuto náloží a poté uniká skrze trysku. Z této jednoduchosti plyne jeho hlavní výhoda, kterou je cena a jednoduchost. Jeho nevýhodou je neuhasitelnost spalování a neovladatelnost tahu.}
\cite{VUTB:NavrhRaketovehoMotoru}


\begin{figure}[H]
    \hspace{0cm}
    \begin{tikzpicture}
        %směr proudu
        \draw[->, red] (4,0.8) to [out=-15,in=180](7.5,0.2) to [out=0,in=-160](10.5,0.8);
        \draw[->, red] (1,0.8) to [out=-15,in=180](8,0.1) to [out=0,in=-160](11,0.7);
        \draw[->, red] (0.5,0) to (11.5,0);
        \begin{scope}[yscale=-1]
            \draw[->, red] (4,0.8) to [out=-15,in=180](7.5,0.2) to [out=0,in=-160](10.5,0.8);
            \draw[->, red] (1,0.8) to [out=-15,in=180](8,0.1) to [out=0,in=-160](11,0.7);
        \end{scope}

        %skříň
        \draw[pattern=north east lines, thick] (0,0.1) to (0,1) to (7,1)
        %tryska horní
        to [out=-90,in=-170, looseness=1.25](10,1) to (10,1.1) to [in=-90,out=-170, looseness=1.3](7.1,1.1)
        %vrácení
        to (-0.1,1.1) to (-0.1,0.1) to cycle;

        \begin{scope}[yscale=-1] % Osová souměrnost
            %skříň
            \draw[pattern=north east lines, thick] (0,0.1) to (0,1) to (7,1)
            %tryska horní
            to [out=-90,in=-170, looseness=1.25](10,1) to (10,1.1) to [in=-90,out=-170, looseness=1.3](7.1,1.1)
            %vrácení
            to (-0.1,1.1) to (-0.1,0.1) to cycle;
        \end{scope}

        %Zážehová nálož
        \draw[pattern=north east lines, thick] (-0.4,0.1) to (0,0.1) to (0, -0.1) to (-0.4, -0.1) to cycle;

        %Palivo
        \fill[pattern=crosshatch dots] (0,1) to (7,1) to (7.05,0.6) to [out=-80, in=170, looseness=0.75](7.36,0.3) to (0,0.3) to cycle;
        \begin{scope}[yscale=-1]\fill[pattern=crosshatch dots] (0,1) to (7,1) to (7.05,0.6) to [out=-80, in=170, looseness=0.75](7.36,0.3) to (0,0.3) to cycle;\end{scope}

        \makePointer{5,1.5}{5.5,0.8}{left}{palivo}
        \makePointer{2,-1.5}{2.2,-1.1}{left}{plášť}
        \makePointer{8.5,-1.5}{9,-0.8}{left}{tryska}
        \makePointer{-0.5,0.5}{-0.2,0.1}{left}{zážehová nálož}

    \end{tikzpicture}
    \caption{Schéma raketového motoru na tuhé palivo \jaDiag}
    \label{fig:raketovyMotorPevny}
\end{figure}

\subsubsection{Raketové motory na kapalné palivo}
{Kapalné raketové motory využívají místo tuhého paliva palivo kapalné, to se za vysokého tlaku spolu s kapalným okysličovadlem spaluje ve spalovací komoře. Výhodou těchto motorů je dobré ovládání tahu a možnost opětného zážehu, nevýhodou je značně vyšší cena a komplexita. Tento motor je vyobrazen na \obrref{fig:raketovyMotorKapalny}.}
\cite{VUTB:NavrhRaketovehoMotoru}

\begin{figure}[H]
    \hspace{0.5cm}
    \begin{tikzpicture}
        %směr proudu
        \draw[->, red] (2.5,0.5) to [in=180,out=-30](3.5,0.2) to [out=0,in=-160](5,1) to [out=20,in=180](7.5,1.3);
        \draw[->, red] (2,0.4) to [in=180,out=-30](3.5,0.1) to [out=0,in=-160](5.5,0.5) to [out=20,in=180](8,0.7);
        \draw[->, red] (1.5,0) to (8.5,0);
        \begin{scope}[yscale=-1]
            \draw[->, red] (2.5,0.5) to [in=180,out=-30](3.5,0.2) to [out=0,in=-160](5,1) to [out=20,in=180](7.5,1.3);
            \draw[->, red] (2,0.4) to [in=180,out=-30](3.5,0.1) to [out=0,in=-160](5.5,0.5) to [out=20,in=180](8,0.7);
        \end{scope}

        %palivo
        \draw[rounded corners=10pt, pattern=north east lines, thick] (-3,0.25) rectangle (0,1.75);
        \fill[rounded corners=7pt, fill=white] (-2.9,0.35) rectangle (-0.1,1.65);
        \draw[rounded corners=7pt, pattern=crosshatch dots, pattern color=orange, thick] (-2.9,0.35) rectangle (-0.1,1.65);

        %vybělovadlo palivo
        \draw[thick] (0,1.05) to (-0.1,1.05)
        (0,0.95) to (-0.1,0.95);
        \fill[fill=white] (0.25,1.036) rectangle (-0.117,0.964);

        %šipka palivo
        \draw[->, orange] (-0.1,1) to (0.1,1) to [out=0,in=180,looseness=1.5](0.8,0.25) to (1.2,0.25);

        %okysličovadlo
        \draw[rounded corners=10pt, pattern=north east lines, thick] (-3,-0.25) rectangle (0,-1.75);
        \fill[rounded corners=7pt, fill=white] (-2.9,-0.35) rectangle (-0.1,-1.65);
        \draw[rounded corners=7pt, pattern=crosshatch dots, pattern color=blue, thick] (-2.9,-0.35) rectangle (-0.1,-1.65);

        %vybělovadlo okysličovadlo
        \draw[thick] (0,-1.05) to (-0.1,-1.05)
        (0,-0.95) to (-0.1,-0.95);
        \fill[fill=white] (0.25,-1.036) rectangle (-0.117,-0.964);

        %šipka okysličovadlo
        \draw[->, blue] (-0.1,-1) to (0.1,-1) to [out=0,in=180,looseness=1.5](0.8,-0.25) to (1.2,-0.25);

        %spalovací komora a tryska vršek
        \draw[pattern=north east lines, thick] (1,0.3) to (1,0.6) to [in=180,out=90](1.4,1) to (2.5,1) to [out=0,in=120](3.5,0.25) to [in=180,out=45](7,1.5) to (7,1.6) to [out=180,in=45](3.525,0.4) to [in=0,out=120](2.6,1.1) to (1.4,1.1) to [out=180,in=90](0.9,0.6) to (0.9,0.3) to cycle;

        %spalovací komora a tryska spodek
        \begin{scope}[yscale=-1]
            \draw[pattern=north east lines, thick] (1,0.3) to (1,0.6) to [in=180,out=90](1.4,1) to (2.5,1) to [out=0,in=120](3.5,0.25) to [in=180,out=45](7,1.5) to (7,1.6) to [out=180,in=45](3.525,0.4) to [in=0,out=120](2.6,1.1) to (1.4,1.1) to [out=180,in=90](0.9,0.6) to (0.9,0.3) to cycle;
            \draw[thick, dotted] (0,1.05) to (0.25,1.05) to [out=0,in=180](0.75,0.3) to (0.9,0.3)
            (0,0.95) to (0.15,0.95) to [out=0,in=180](0.65,0.2) to (0.9,0.2);
        \end{scope}

        %trubka palivo
        \draw[thick, dotted] (0,1.05) to (0.25,1.05) to [out=0,in=180](0.75,0.3) to (0.9,0.3)
        (0,0.95) to (0.15,0.95) to [out=0,in=180](0.65,0.2) to (0.9,0.2);

        %mezivěcička
        \draw[pattern=north east lines, thick] (0.9,0.2) rectangle (1,-0.2);

        \makePointer{-3.5,1.5}{-2,1}{left}{palivo}
        \makePointer{-3.5,-1.5}{-2,-1}{left}{okysličovadlo}
        \makePointer{2.5,1.5}{2,0.5}{above}{spalovací komora}
        \makePointer{3,-2}{4.25,-0.98}{left}{tryska}

    \end{tikzpicture}
    \caption{Schéma raketového motoru na kapalné palivo \jaDiag}
    \label{fig:raketovyMotorKapalny}
\end{figure}

\newpage

\subsection{Proudové motory}
{Proudové motory urychlují nasátý vzduch za pomoci spalování. Dělají to různými způsoby, většina proudových motorů má čtyři části. Nejprve kompresor stlačí nasátý vzduch, poté je tento vzduch ve spalovací komoře obohacen o palivo a spálen. Následně žhavé spaliny točí turbínou, která zpětně pohání kompresor. Poslední částí je tryska, skrze kterou je vysokou rychlostí vyfukován plyn, čímž motor tvoří tah. Proudové motory dělíme dle konstrukce na dvě hlavní skupiny: jednoproudové a dvouproudové motory.}
\cite{VUTB:PrehledTechnickychAspektuVyvojeLeteckychProudovychMotoru}

\subsubsection{Jednoproudové motory}
{Tyto motory jsou oproti dvouproudovým motorům jednoduché, jsou však méně účinné, a proto jsou jimi v současnosti mnohdy nahrazeny. Dříve se kvůli jejich jednoduchosti využívaly ve~veškerých proudových letadlech, pro jejich vyšší výkon se využívají převážně pro vojenské účely. Příklad jednoproudového motoru naleznete na \obrref{fig:jednoproudovyMotor}.}
\cite{VUTB:PrehledTechnickychAspektuVyvojeLeteckychProudovychMotoru}

\begin{figure}[H]
    \centering
    \begin{tikzpicture}[scale=0.85]

        %ukazování směru proudu
        \draw[->, blue] (-1,0.75) to [in=180, out=10](4.1,1.05);
        \draw[->, red] (6,1.1) to [in=180,out=10](7.2,1.2) to [in=170,out=-10](9,0.35) to [out=-10,in=180](9.5,0.25);
        \begin{scope}[yscale=-1]
            \draw[->, blue] (-1,0.75) to [in=180, out=10](4.1,1.05);
            \draw[->, red] (6,1.1) to [in=180,out=10](7.2,1.2) to [in=170,out=-10](9,0.35) to [out=-10,in=180](9.5,0.25);
        \end{scope}

        %výbuchy
        \makeDot{4.7,1}{orange}\makeDot{4.8,1.1}{orange}\makeDot{4.9,0.9}{orange}\makeDot{5,1.05}{orange}\makeDot{5.2,0.95}{orange}\makeDot{5.3,1.15}{orange}\makeDot{5.4,1.05}{orange}\makeDot{5.5,0.9}{orange}
        \begin{scope}[yscale=-1]\makeDot{4.7,1}{orange}\makeDot{4.8,1.1}{orange}\makeDot{4.9,0.9}{orange}\makeDot{5,1.05}{orange}\makeDot{5.2,0.95}{orange}\makeDot{5.3,1.15}{orange}\makeDot{5.4,1.05}{orange}\makeDot{5.5,0.9}{orange}\end{scope}

        %skříň
        \draw[pattern=north east lines, thick] (-0.5,1.5) to (0,1.5) to (4,1.25) to (6.4,1.35) to (7.1,1.6) to (8.25,1.2) to (9,1.2) to (9,1.25) to (8.25,1.25) to (7.1,1.65) to (6.4,1.4) to (4,1.3) to (0,1.55) to (-0.5,1.55) to cycle;

        %vnitřek
        \draw[pattern=north east lines, thick] (-1,0) to [out=80,in=180](-0.25,0.5) to (1,0.5) to [out=15,in=180](3.75,0.9) to (5,0.75) to (6.5,0.8) to (7,0.9) to (8.5,0);

        %Hřídel
        \draw[fill=white, thick] (3.8,0) to (3.8, 0.5) to (4, 0.1) to (4.3, 0.1) to (4.3, 0.2) to (4.4, 0.2) to (4.4, 0.1) to (4.5, 0.1) to (4.5,0.15) to (4.6,0.15) to (4.6, 0.1) to (4.7,0.1) to (4.8,0.15) to (4.9,0.15) to (5,0.25) to (5.9,0.25) to (6,0.35) to (7,0.35) to (7,0);
        \draw (4,0.1) to (4,0) (4.3,0.1) to (4.3,0) (4.4,0.2) to (4.4,0) (4.5,0.15) to (4.5,0) (4.6,0.15) to (4.6,0) (4.7,0.1) to (4.7,0) (4.8,0.15) to (4.8,0) (4.9,0.15) to (4.9,0) (5,0.25) to (5,0) (5.9,0.25) to (5.9,0) (6,0.35) to (6,0) (7,0.35) to (7,0);

        \begin{scope}[yscale=-1]
            %skříň
            \draw[pattern=north east lines, thick] (-0.5,1.5) to (0,1.5) to (4,1.25) to (6.4,1.35) to (7.1,1.6) to (8.25,1.2) to (9,1.2) to (9,1.25) to (8.25,1.25) to (7.1,1.65) to (6.4,1.4) to (4,1.3) to (0,1.55) to (-0.5,1.55) to cycle;

            %vnitřek
            \draw[pattern=north east lines, thick] (-1,0) to [out=80,in=180](-0.25,0.5) to (1,0.5) to [out=15,in=180](3.75,0.9) to (5,0.75) to (6.5,0.8) to (7,0.9) to (8.5,0);

            %hřídel
            \draw[fill=white, thick] (3.8,0) to (3.8, 0.5) to (4, 0.1) to (4.3, 0.1) to (4.3, 0.2) to (4.4, 0.2) to (4.4, 0.1) to (4.5, 0.1) to (4.5,0.15) to (4.6,0.15) to (4.6, 0.1) to (4.7,0.1) to (4.8,0.15) to (4.9,0.15) to (5,0.25) to (5.9,0.25) to (6,0.35) to (7,0.35) to (7,0);
            \draw (4,0.1) to (4,0) (4.3,0.1) to (4.3,0) (4.4,0.2) to (4.4,0) (4.5,0.15) to (4.5,0) (4.6,0.15) to (4.6,0) (4.7,0.1) to (4.7,0) (4.8,0.15) to (4.8,0) (4.9,0.15) to (4.9,0) (5,0.25) to (5,0) (5.9,0.25) to (5.9,0) (6,0.35) to (6,0) (7,0.35) to (7,0);
        \end{scope}

        %Kompresor
        \makeBlades{1.43}{0}{0.1}{white}
        \makeBlades{1.415}{0.3}{0.1}{white}
        \makeBlades{1.4}{0.6}{0.1}{white}
        \makeBlades{1.385}{0.9}{0.1}{white}
        \makeBlades{1.37}{1.2}{0.1}{white}
        \makeBlades{1.355}{1.5}{0.1}{white}
        \makeBlades{1.335}{1.8}{0.0975}{white}
        \makeBlades{1.32}{2.1}{0.095}{white}
        \makeBlades{1.285}{2.4}{0.0925}{white}
        \makeBlades{1.27}{2.7}{0.09}{white}
        \makeBlades{1.245}{3}{0.0875}{white}
        \makeBlades{1.22}{3.25}{0.085}{white}
        \makeBlades{1.205}{3.5}{0.0825}{white}
        \makeBlades{1.17}{3.75}{0.08}{white}

        %turbína
        \makeBlades{1.3}{6.5}{0.125}{white}
        \makeBlades{1.4}{6.75}{0.15}{white}
        \makeBlades{1.5}{7}{0.175}{white}

        %vysvětlivky
        \makePointer{1,-2}{2.1,-1.2}{left}{kompresor}
        \makePointer{5,-2}{6.5,-1.2}{left}{turbína}
        \makePointer{6.5,2}{6,0}{right}{hřídel}
        \makePointer{9.75,1}{8.75,0.25}{right}{tryska}
        \makePointer{3.5,2}{5.3,1}{left}{spalovací komora}
        
    \end{tikzpicture}
    \caption{Schéma jednoproudového motoru \jaDiag}
    \label{fig:jednoproudovyMotor}
\end{figure}

\subsubsection{Dvouproudové motory}
{Dvouproudové motory rozdělují proud ve dva. Oba proudy se nasají dmychadlem a poté se rozdělí na vnitřní a obtokový. Vnitřní proud pokračuje jádrem motoru do kompresoru, postupuje skrze spalovací komoru a turbínu až do trysky, kde se spojí s obtokovým proudem a tvoří tah. Vzniká také nový parametr: obtokový poměr. Obecně lze říci, že čím vyšší je obtokový poměr, tím účinnější je motor. Zároveň platí, že čím nižší je, tím vyšší má výkon. Protože jsou dvouproudové motory účinnější, jsou využívány ve valné většině civilních i vojenských letounů. Jeho konstrukce je popsána na \obrref{obr:dvouproudovyMotor}.}
\cite{VUTB:PrehledTechnickychAspektuVyvojeLeteckychProudovychMotoru}

\begin{figure}[H]
    \centering
    \begin{tikzpicture}[scale=0.85]
        %ukazování směru proudu
        \draw[->, blue] (-1,1.75) to (4,1.75) to [out=0,in=180](7,2);
        \draw[->,blue] (-1,1) to [out=-10,in=180](3,0.75);
        \draw[->,red] (4,0.75) to [out=0,in=180](5.9,1) to [out=0,in=180](8,0.5);

        \begin{scope}[yscale=-1]
            \draw[->, blue] (-1,1.75) to (4,1.75) to [out=0,in=180](7,2);
            \draw[->,blue] (-1,1) to [out=-10,in=180](3,0.75);
            \draw[->,red] (4,0.75) to [out=0,in=180](5.9,1) to [out=0,in=180](8,0.5);
        \end{scope}

        %výbuchy
        \makeDot{3.7,0.75}{orange}
        \makeDot{3.5,0.7}{orange}
        \makeDot{3.6,0.8}{orange}
        \makeDot{3.4,0.85}{orange}
        \makeDot{3.7,0.95}{orange}

        \begin{scope}[yscale=-1]
            %výbuchy
            \makeDot{3.7,0.75}{orange}
            \makeDot{3.5,0.7}{orange}
            \makeDot{3.6,0.8}{orange}
            \makeDot{3.4,0.85}{orange}
            \makeDot{3.7,0.95}{orange}
        \end{scope}

        %skříň vnější
        \draw[pattern=north east lines, thick] (-0.5,2.25) to (4,2.2) to [out=170,in=0](1,2.5) to [out=180,in=30]cycle;
        \begin{scope}[yscale=-1]\draw[pattern=north east lines, thick] (-0.5,2.25) to (4,2.2) to [out=170,in=0](1,2.5) to [out=180,in=30]cycle;\end{scope}

        %skříň vnitřní
        \draw[pattern=north east lines, thick] (0.75,1.25) to [out=0,in=170](3,0.95) to [out=30,in=190](4.5,1.1) to [out=15,in=180](5.7,1.35) to [out=0, in=170](7,1.1)to (7,1.2) to [out=170,in=0](5.7,1.45) to [out=180,in=20](4.5,1.2) to [out=180,in=0](0.75,1.35) to cycle;
        \begin{scope}[yscale=-1]\draw[pattern=north east lines, thick] (0.75,1.25) to [out=0,in=170](3,0.95) to [out=30,in=190](4.5,1.1) to [out=15,in=180](5.7,1.35) to [out=0, in=170](7,1.1)to (7,1.2) to [out=170,in=0](5.7,1.45) to [out=180,in=20](4.5,1.2) to [out=180,in=0](0.75,1.35) to cycle;\end{scope}

        %vnitřek
        \draw[pattern=north east lines, thick] (0.5,0) to (0.5,0.4) to [out=0,in=185](3,0.65) to [out=-15,in=180](4.5,0.5) to [out=0,in=190](5.75,0.7) to (6.9,0);
        \begin{scope}[yscale=-1]\draw[pattern=north east lines, thick] (0.5,0) to (0.5,0.4) to [out=0,in=185](3,0.65) to [out=-15,in=180](4.5,0.5) to [out=0,in=190](5.75,0.7) to (6.9,0);\end{scope}

        %Hřídel
        \draw[fill=white!75!blue, thick] (-0.3,0) to [out=50,in=180](0.5,0.3) to (0.5, 0.2) to (1.7, 0.2) to (1.8,0.1) to (5,0.1) to (5.1,0.2) to (5.5,0.2) to (5.5,-0.1);
        \draw[fill=white!75!red, thick] (1.8,0.1) to (1.7, 0.2) to (5,0.2) to (5.1,0.1) to (5,0.1) to (1.8,0.1);
        \begin{scope}[yscale=-1]\draw[fill=white!75!blue, thick] (-0.3,0) to [out=50,in=180](0.5,0.3) to (0.5,0) to (0.5, 0.2) to (1.7, 0.2) to (1.8,0.1) to (5,0.1) to (5.1,0.2) to (5.5,0.2) to (5.5,0);
        \draw[fill=white!75!red, thick] (1.8,0.1) to (1.7, 0.2) to (5,0.2) to (5.1,0.1) to (5,0.1) to (1.8,0.1);\end{scope}

        %Dmychadlo
        \makeBlades{2.2}{0.5}{0.2}{white!75!blue}

        %Kompresor nízkotlaký
        \makeBlades{1.2}{0.9}{0.1}{white!75!blue}
        \makeBlades{1.175}{1.2}{0.1}{white!75!blue}
        \makeBlades{1.15}{1.5}{0.1}{white!75!blue}

        %Kompresor vysokotlaký
        \makeBlades{1}{1.9}{0.0925}{white!75!red}
        \makeBlades{0.975}{2.2}{0.09}{white!75!red}
        \makeBlades{0.95}{2.5}{0.0875}{white!75!red}
        \makeBlades{0.925}{2.8}{0.085}{white!75!red}

        %Turbína vysokotlaká
        \makeBlades{1.05}{4.5}{0.125}{white!75!red}
        \makeBlades{1.1}{4.75}{0.15}{white!75!red}

        %Turbína nízkotlaká
        \makeBlades{1.2}{5}{0.175}{white!75!blue}
        \makeBlades{1.25}{5.25}{0.175}{white!75!blue}
        \makeBlades{1.3}{5.5}{0.175}{white!75!blue}
    
        %Popisky
        \makePointer{-1,2}{0.5,1.8}{left}{dmychadlo}
        \makePointer{2,-3}{1,-1}{right}{nízkotlaký kompresor}
        \makePointer{4,-2.3}{2,-1}{right}{vysokotlaký kompresor}
        \makePointer{7,-1.6}{4.9,-0.8}{right}{vysokotlaká turbína}
        \makePointer{8,0.1}{7,0.3}{right}{tryska}
        \makePointer{7.6,1}{5.7,0.9}{right}{nízkotlaká turbína}
        \makePointer{4.3,2.6}{3.8,0.9}{right}{spalovací komora}

    \end{tikzpicture}
    \caption{Schéma dvouproudového motoru \jaObr}
    \label{obr:dvouproudovyMotor}
\end{figure}

\newpage

\subsection{Rovnice reaktivních motorů}
{V této kapitole vysvětlím rovnice o reaktivních motorech, přesněji rovnice pro výpočet tahu raketových a proudových motorů. Také popíši rovnici pro specifický impuls.}

\subsubsection{Tah raketových motorů}
{Tah raketového motoru můžeme vypočítat pomocí rovnice \rvref{rv:RaketovyMotorTah}.}
\cite{NASA:propulsionIndex}

\begin{equation}\label{rv:RaketovyMotorTah}
    \vec{F}=\dot{m}\cdot{v}
\end{equation}

{\(\vec{F}\) je tah [N]}\\
{\(\dot{m}\) je hmotnostní průtok [\kgs]}\\
{\(v\) je výstupní rychlost [\ms]}\\

\subsubsection{Tah proudových motorů}
{Tah je jedním z nejdůležitějších parametrů proudového motoru, vypočítáme jej rovnicí \rvref{rv:ProudovyMotorTah}.}
\cite{NASA:propulsionIndex}

\begin{equation}\label{rv:ProudovyMotorTah}
    \vec{F}=\dot{m}_{1}\cdot{v_1}-\dot{m}_{0}\cdot{v_0}
\end{equation}

{\(\vec{F}\) je tah [N]}\\
{\(\dot{m}_{1}\) je výstupní hmotnostní průtok [\kgs]}\\
{\(\dot{v}_{1}\) je výstupní rychlost [\ms]}\\
{\(\dot{m}_{0}\) je vstupní hmotnostní průtok [\kgs]}\\
{\(\dot{v}_{0}\) je vstupní rychlost [\ms]}

\subsubsection{Specifický impuls}
{Specifický impuls je parametr, díky kterému lze snadno porovnávat různé proudové motory. Motor s vyšším specifickým impulsem je účinnější, jelikož na daný hmotnostní průtok vyvine větší tah. K výpočtu specifického impulsu raketového motoru využijeme rovnici \rvref{rv:SpecifickyImpuls}.}
\cite{NASA:propulsionIndex}

\begin{subequations}
    \begin{equation}\label{rv:SpecifickyImpuls}
        I_s=\frac{\vec{F}}{\dot{m}\cdot{\vec{g}_0}}
    \end{equation}

    {\(I_s\) je specifický impuls [s]}\\
    {\(\vec{F}\) je tah [N]}\\
    {\(\dot{m}\) je hmotnostní průtok [\kgs]}\\
    {\(\vec{g}_0\) je gravitační zrychlení na zemi (9,81 \mss)[\mss]}\odst

    {Pokud chceme vypočítat specifický impuls proudového motoru, místo celkového hmotnostního průtoku použijeme pouze hmotnostní průtok paliva, viz rovnice \rvref{rv:SpecifickyImpulsProudovy}.}

    \begin{equation}\label{rv:SpecifickyImpulsProudovy}
        I_s=\frac{\vec{F}}{\dot{m}_f\cdot{\vec{g}_0}}
    \end{equation}

    {$\dot{m}_f$ je hmotnostní průtok paliva [\kgs]}
\end{subequations}

\newpage

\addtocontents{toc}{\newpage}
\section{Praktická část}
{V této kapitole je popsána tvorba učebních pomůcek, mým cílem bylo vytvořit materiály, které učitelé mohou snadno využít při výuce tepelných motorů a které budou studenty bavit. Toho jsem se pokusil docílit použitím 3D grafiky a animace. Věřím, že žák, kterého výuka baví, si více zapamatuje a více se naučí.}\odst
{Vzhledem k mým zkušenostem s open source softwarem \code{Blender} pro 3D tvorbu jsem se rozhodl využít právě tento program; nabízí mnoho nástrojů v okruhu 3D modelování, (procedurální) animace a renderování. Nejprve jsem vytvořil 3D animace jednotlivých tepelných motorů, ty jsou samostatně dostupné ke stažení v příloze, slouží však především jako podklad pro další výukové materiály, nikoliv jako samostatné výukové pomůcky.}\odst
{Vytvořil jsem celkem 4 prezentace, jedna je určena pro obecné seznámení s tepelnými motory, zbylé tři podrobněji popisují každý druh těchto strojů. Pro snadné sdílení jsem využil služby \code{Google Sheets}, díky které mají učitelé a žáci snadný přístup k prezentacím odkudkoliv. Tyto prezentace obsahují již zmíněné animace a také mnou vytvořené ilustrační obrázky.}\odst
{Další výukovou pomůckou kterou jsem vytvořil je výukové video, to popisuje všechny tepelné motory zmíněné v prezentacích pomocí dynamické a svižné 3D animace s hlasovým komentářem. Opět jsem využil software \code{Blender} spolu s programem \code{Audacity} pro nahrání hlasu a zvukové knihovny \code{Soundly}. Toto video může učitel promítat o hodině nebo zaslat žákům přes internet. Žák na něj může žák sám narazit na platformě \code{Youtube}.}\odst
{Poslední výukovou pomůcku, kterou jsem vytvořil, jsou pracovní listy; ty jsem vytvořil pomocí sázecího softwaru \LaTeX. První z nich je určen pro žáky základních škol, druhý pro žáky druhého stupně středních škol a gymnázií; druhý jmenovaný má dvě varianty. Součástí jsou samozřejmě pracovní listy se správnými odpověďmi a návrh na hodnocení. Pracovní listy mohou sloužit pro procvičení či jako písemné práce.}\odst
{Dohromady by tyto pomůcky měli pokrýt výuku jedné až tří vyučovacích hodin v závislosti na způsobu vyučování. Mým cílem bylo nejen vytvořit obsahově hodnotné materiály, ale také díky zajímavé grafické stránce zaujmout žákovu pozornost a vzbudit v něm zájem nejen o tepelné motory, ale o celou fyziku. V následujících kapitolách popíši tvorbu a návrh na použití mých materiálů. Na závěr jsem vyzkoušel své materiály v praxi na žácích druhého ročníku gymnázia Arabská.}

\newpage

\section{Animace}\label{sc:animace}
{Pro tvorbu animací jsem využil software \code{Blender}, a to kvůli mým předešlým zkušenostem. Tvorbu animací popíši na příkladu čtyřdobého vznětového motoru 1.9 TDI, ostatní animace jsem tvořil obdobně.}\odst
{Nejprve jsem za pomoci technických výkresů vytvořil geometrii jednotlivých součástek daného motoru, v tomto případě nejprve pístu. Tvorba z výkresu (na \obrref{obr:pistVykresy}) je standardním postupem, je však nutno používat i jiné zdroje a domýšlet si třetí rozměr dané součástky.\\ Na \obrref{obr:pistHotovy} je hotový model pístu.}

\begin{figure}[H]
    \centering
    \begin{subfigure}{.5\textwidth}
        \centering
        \includegraphics[scale=.19]{assets/images/pistBok.png}
        \caption{Tvorba pístu motoru 1.9 TDI s použitím technických výkresů \jaObr}
        \label{obr:pistVykresy}
    \end{subfigure}%
    \begin{subfigure}{.5\textwidth}
        \centering
        \includegraphics[scale=.2]{assets/images/pist.png}
        \caption{Hotový píst \jaObr}
        \label{obr:pistHotovy}
    \end{subfigure}
\end{figure}

{Po vytvoření všech součástek motoru je nutno přidat materiály, rozhodl jsem se je tvořit procedurálně. Tento postup, vyobrazen na \obrref{obr:NodeUkazka}, je tvořen různě spojenými operacmi, které určují jak se má materiál chovat.}

\begin{figure}[H]
    \centering
    \includegraphics[scale=.25]{assets/images/NodeUkazka.png}
    \caption{Ukázka procedurální tvorby materiálů \jaObr}
    \label{obr:NodeUkazka}
\end{figure}

\newpage

{Na \obrref{obr:predMaterialy} je motor 1.9 TDI před přidáním materiálů, na \obrref{obr:poMaterialy} po přidání materiálů. Plochy, kde byl motor rozříznut, jsem obarvil červeně.}

\begin{figure}[H]
    \centering
    \begin{subfigure}[t]{.45\textwidth}
        \centering
        \includegraphics[scale=.2]{assets/images/1.9TDImodelUkazka.png}
        \caption{Motor 1.9 TDI před přidáním materiálů \jaObr}
        \label{obr:predMaterialy}
    \end{subfigure}
    \hfill
    \begin{subfigure}[t]{.45\textwidth}
        \centering
        \includegraphics[scale=.2]{assets/images/1.9TDImodelMaterialy.png}
        \caption{Motor 1.9 TDI po přidání materiálů \jaObr}
        \label{obr:poMaterialy}
    \end{subfigure}
\end{figure}

{Model motoru vypadá jako hotový, je však statický. Animace jsem přidal procedurálně, k tomu jsem využit \code{geometry nodes}. Ty dokáží nedestruktivně; tedy způsobem, při kterém nedochází k nenávratným změnám, upravovat geometrii modelů. Parní turbíny, proudové a raketové motory byli vcelku jednoduché na naanimování, nejsložitější byla animace čtyřdobých motorů a parního stroje, a to kvůli jejich klikovým mechanismům.}\odst
{Vše se odvíjí od pohybu klikové hřídele, na ni jsou napojeny ojnice a na ně písty. Spodní část ojnice se pohybuje po kružnici s poloměrem vzdálenosti od středu klikové hřídele. Ojnici jsem poté otočil tak, aby se střed vrchní části pohyboval pouze vertikálně. Úhel, o který jsem ji musel otočit jsem zjistil pomocí sinové a kosinové větě. Vše dohromady vypadá v \code{Blenderu} takto: \obrref{obr:geometryNodeUkazka}.}

\begin{figure}[H]
    \centering
    \includegraphics[scale=.35]{assets/images/geometryNodeUkazka.png}
    \caption{Ukázka procedurální animace ojnice motoru 1.9 TDI \jaObr}
    \label{obr:geometryNodeUkazka}
\end{figure}

{Při animaci pístů jsem lehce podváděl, písty se pohybují narozdíl od skutečných motorů po sinusoidě, díky lehkým úpravám je však tento nedostatek nepostřehnutelný.}

\newpage

{Podobným způsobem jsem animoval i otevírání a zavírání ventilů. Nakonec jsem přidal barevná ,,vlákna'', která zobrazují pohyb plynu v motoru (\obrref{obr:1.9TDICary}).}

\begin{figure}[H]
    \centering
    \includegraphics[scale=.35]{assets/images/cary.png}
    \caption{,,Vlákna'' zobrazující pohyb plynu \jaObr}
    \label{obr:1.9TDICary}
\end{figure}

{Posledním krokem je převedení 3D modelu na obrázek, tento proces je hlouběji popsán v kapitole \scref{sc:renderovani}. K tomu jsem použil zabudovaný render engine \code{Cycles}, který věrně mimikuje chování skutečného světla. Produkt tohoto postupu je na \obrref{obr:1.9TDIUkazkaHotovo}.}

\begin{figure}[H]
    \begin{minipage}[b]{0.5\linewidth}
        \centering
        \includegraphics[scale=.11]{assets/images/1.9TDI1.png}\\
        {Zážeh ve čtvrtém válci}
        \vspace{4ex}
    \end{minipage}
    \begin{minipage}[b]{0.5\linewidth}
        \centering
        \includegraphics[scale=.11]{assets/images/1.9TDI2.png}\\
        {Zážeh ve třetím válci}
        \vspace{4ex}
    \end{minipage}
    \begin{minipage}[b]{0.5\linewidth}
        \centering
        \includegraphics[scale=.11]{assets/images/1.9TDI3.png}\\
        {Zážeh v prvním válci}
        \vspace{4ex}
    \end{minipage}
    \begin{minipage}[b]{0.5\linewidth}
        \centering
        \includegraphics[scale=.11]{assets/images/1.9TDI4.png}\\
        {Zážeh ve druhém válci}
        \vspace{4ex}
    \end{minipage}
    \caption{Hotový motor 1.9 TDI zachycen v různých částech svého cyklu \jaObr}
    \label{obr:1.9TDIUkazkaHotovo}
\end{figure}

\newpage

\subsection{Animace parního stroje}
{Jako příklad parního stroje jsem vytvořil ležatý dvojčinný parní stroj s šoupátkovým rozvodem, ten můžete vidět na \obrref{obr:ParniStrojRender}. Tento motor není na rozdíl od ostatních modelů založen na žádném skutečném motoru.}

\begin{figure}[H]
    \centering
    \includegraphics[scale=.2]{assets/images/ParniStrojRender.png}
    \caption{Model parního stroje \jaObr}
    \label{obr:ParniStrojRender}
\end{figure}

{Animace je dostupná ke stažení v přílohách zde: \ref{pr:animaceParniStroj}.}

\newpage

\subsection{Animace parní turbíny}
{Vytvořil jsem 3D model parní turbíny MTD-50 společnosti Doosan Škoda Power. Pro tvorbu modelu jsem využil technického výkresu \obrref{obr:DoosanSkodaPowerMTD50}.}
\cite{SP:ApplicationAspectsOfSteamTurbinesForCombinedHeatAndPowerGeneration}

\begin{figure}[H]
    \centering
    \includegraphics[scale=2]{assets/images/DoosanSkodaPowerMTD50.png}
    \caption{Technické výkresy parní turbíny MTD-50 \cite{SP:ApplicationAspectsOfSteamTurbinesForCombinedHeatAndPowerGeneration}}
    \label{obr:DoosanSkodaPowerMTD50}
\end{figure}

{Hotový model turbíny je na \obrref{obr:ParniTurbinaRender}. Animace je dostupná ke stažení v přílohách zde: \ref{pr:animaceParniTurbina}.}

\begin{figure}[H]
    \centering
    \includegraphics[scale=.2]{assets/images/ParniTurbinaRender.png}
    \caption{Model parní turbíny MDT-50 \jaObr}
    \label{obr:ParniTurbinaRender}
\end{figure}

\newpage

\subsection{Animace čtyřdobého spalovacího motoru}
{Jako příklad čtyřdobého spalovacího motoru jsem vybral motor Škoda 1000 MB. Použil jsem výkresy \obrref{obr:1000MBvykresyPrurez} a \obrref{obr:1000MBvykresyBok}.}
\cite{AUTOMOBIL:Skoda1000MBLegendaSlavi60Let}

\begin{figure}[H]
    \centering
    \begin{subfigure}{.5\textwidth}
        \centering
        \includegraphics[scale=.2]{assets/images/Skoda1000MBPredek.jpg}
        \caption{Průřez}
        \label{obr:1000MBvykresyPrurez}
    \end{subfigure}%
    \begin{subfigure}{.5\textwidth}
        \centering
        \includegraphics[scale=.2]{assets/images/Skoda1000MBBok.jpg}
        \caption{Bok}
        \label{obr:1000MBvykresyBok}
    \end{subfigure}
\end{figure}

{Výsledek mé práce je na \obrref{obr:1000MBRender}, animaci můžete stáhnout v přílohách zde: \ref{pr:animaceCtyrdobyZazehovyMotor}.}

\begin{figure}[H]
    \centering
    \includegraphics[scale=0.6]{assets/images/1000MBRender.png}
    \caption{Model čtyřdobého zážehového motoru Škoda 1000 MB \jaObr}
    \label{obr:1000MBRender}
\end{figure}
\newpage

\subsection{Animace čtyřdobého vznětového motoru}
{Za předlohu pro čtyřdobý vznětový motor jsem vybral motor od společnosti Volkswagen, a to motor VW 1.9 R4 8v TDI. Pro tvorbu 3D modelu jsem využil technické výkresy \obrref{obr:1.9TDIvykresyPrurez} a \obrref{obr:1.9TDIvykresyBok}.}
\cite{VWGAG:RealizingFutureTrendsInDieselEngineDevelopment}

\begin{figure}[H]
    \centering
    \begin{subfigure}{.5\textwidth}
        \centering
        \includegraphics[scale=.3]{assets/images/1.9TDICrossSection.png}
        \caption{Průřez}
        \label{obr:1.9TDIvykresyPrurez}
    \end{subfigure}%
    \begin{subfigure}{.5\textwidth}
        \centering
        \includegraphics[scale=.3]{assets/images/1.9TDILongitudinalSection.png}
        \caption{Bok}
        \label{obr:1.9TDIvykresyBok}
    \end{subfigure}
\end{figure}

{Hotový model motoru 1.9 TDI je na \obrref{obr:1.9TDImodel}, animace je dostupná ke stažení zde: \ref{pr:animaceCtyrdobyVznetovyMotor}.}

\begin{figure}[H]
    \centering
    \includegraphics[scale=.2]{assets/images/1.9TDI3.png}
    \caption{Model motoru 1.9 TDI \jaObr}
    \label{obr:1.9TDImodel}
\end{figure}

\newpage

\subsection{Animace dvoudobého zážehového motoru}
{Jako příklad dvoudobého zážehového motoru jsem vybral motor motocyklu ČZ 175, přesněji jeho sportovní verzi se dvěma svíčkami. Jako podklady pro tvorbu modelu jsem využil výkresy \obrref{obr:CZ175Vykres}.}
\cite{MZ:CZ175}

\begin{figure}[H]
    \centering
    \includegraphics[scale=.3]{assets/images/motor175.jpg}
    \caption{Výkresy motoru ČZ 175}
    \label{obr:CZ175Vykres}
\end{figure}

{Hotový model je na \obrref{obr:CZ175Render}, animaci lze stáhnout zde: \ref{pr:animaceDvoudobyZazehovyMotor}.}

\begin{figure}[H]
    \centering
    \includegraphics[scale=.6]{assets/images/CZ175Render.png}
    \caption{Model dvoudobého zážehového motoru ČZ 175 \jaObr}
    \label{obr:CZ175Render}
\end{figure}

\newpage

\subsection{Animace raketového motoru na tuhé palivo}
{Jako příklad raketového motoru na tuhé palivo jsem vybral Hercules/Bermite Mk. 36 z naváděné rakety AIM-9 Sidewinder. Jako podklady pro modely jsem použil výkres \obrref{obr:AIM9vykres}. Pro tuto animaci jsem se rozhodl změnit barvu pozadí na tmavě šedou, aby lépe vynikl plamen, také jsem zobrazil průřez palivem.}
\cite{TB:AIM9Sidewinder}

\begin{figure}[H]
    \centering
    \includegraphics[scale=.5]{assets/images/aim-9-sidewinder.png}
    \caption{Výkres rakety AIM-9 Sidewinder}
    \label{obr:AIM9vykres}
\end{figure}

{Dokončený model je na \ref{obr:AIM9Render}, animaci můžete stáhnout zde: \ref{pr:animacePevnyRaketovyMotor}.}

\begin{figure}[H]
    \centering
    \includegraphics[scale=0.5]{assets/images/AIM9Render.png}
    \caption{Model raketového motoru na tuhé palivo AIM-9 Sidewinder \jaObr}
    \label{obr:AIM9Render}
\end{figure}

\newpage

\subsection{Animace raketového motoru na kapalné palivo}
{Vytvořil jsem model motoru Rocketdyne J-2. Jako podklad pro tvorbu jsem využil technický výkres \obrref{obr:J2vykres}. Ze stejného důvodu jako u raketového motoru na pevné palivo jsem zvolil tmavě šedé pozadí.}
\cite{HR:F1RocketEngine}

\begin{figure}[H]
    \centering
    \includegraphics[scale=.2]{assets/images/RocketdyneJ2vykres.png}
    \caption{Výkres raketového motoru J-2}
    \label{obr:J2vykres}
\end{figure}

{Hotový model naleznete na \obrref{obr:J2Render}, animace je dostupná ke stažení zde: \ref{pr:animaceKapalnyRaketovyMotor}.}

\begin{figure}[H]
    \centering
    \includegraphics[scale=.6]{assets/images/J2Render.png}
    \caption{Model raketového motoru na kapalné palivo J-2 \jaObr}
    \label{obr:J2Render}
\end{figure}

\newpage

\subsection{Animace proudového motoru}
{Vytvořil jsem model malého proudového motoru TJ100 vyráběného v PBS Velký Bíteš. Pro tvorbu modelu jsem využil výkresy \obrref{obr:PBSTJ100Vykres}.}
\cite{PBS:Minijets}

\begin{figure}[H]
    \centering
    \includegraphics[scale=.3]{assets/images/TJ100Vykres.png}
    \caption{Výkresy motoru TJ100}
    \label{obr:PBSTJ100Vykres}
\end{figure}

{Model motoru TJ100 je na \obrref{obr:TJ100Render}. Animace je dostupná ke stažení v přílohách zde: \ref{pr:animaceProudovyMotor}.}

\begin{figure}[H]
    \centering
    \includegraphics[scale=0.5]{assets/images/TJ100Render.png}
    \caption{Model proudového motoru TJ100 \jaObr}
    \label{obr:TJ100Render}
\end{figure}

\newpage

\subsection{Animace dvouproudového motoru}
{Na popsání fungování dvouproudového motoru jsem vybral motor CFM International CFM56, který je jedním z nejrozšířenějších proudových motorů vůbec. Ke tvorbě jsem využil výkresy \obrref{obr:CFM56vykres1} a \obrref{obr:CFM56vykres2}.}
\cite{RG:ReducedOrderModel}\cite{TL:OffDesignPerformancePrediction}

\begin{figure}[H]
    \centering
    \begin{subfigure}{.5\textwidth}
        \centering
        \includegraphics[scale=.4]{assets/images/CFM56vykres1.png}
        \caption{Výkres motoru CFM56}
        \label{obr:CFM56vykres1}
    \end{subfigure}%
    \begin{subfigure}{.5\textwidth}
        \centering
        \includegraphics[scale=.3]{assets/images/CFM56vykres2.png}
        \caption{Výkres motoru CFM56}
        \label{obr:CFM56vykres2}
    \end{subfigure}
\end{figure}

{Hotový model je na \obrref{obr:CFM56Render}, animace je dostupná ke stažení zde: \ref{pr:animaceDvouproudovyMotor}.}

\begin{figure}[H]
    \centering
    \includegraphics[scale=0.6]{assets/images/CFM56Render.png}
    \caption{Model dvouproudového motoru CFM56 \jaObr}
    \label{obr:CFM56Render}
\end{figure}

\newpage

\section{Video}
{Jedním z nejdůležitějších výstupů mé práce je krátké výukové video; v této kapitole popíši a odůvodním svůj postup při jeho tvorbě. \textcolor{red}{Video je dostupné na platformě YouTube a také v plné kvalitě v přílohách zde: \ref{pr:videoTepelneMotory}.}

\subsection{Předprodukce}
{Proces tvorby videa začal již před začátkem samotné 3D tvorby, nejprve jsem si stanovil cíle a vytvořil přibližný plán videa, který jsem poté rozšířil do scénáře.}\odst

\subsubsection{Cíl videa}
{Nejprve jsem si musel rozmyslet čeho se vlastně snažím dosáhnout; cílem mého videa je zábavně a dynamicky vysvětlit žákům látku tepelných motorů, video by zárověň nemělo být příliš dlouhé a mělo by být lehce vstřebatelné.}\odst
{Abych dosáhl tohoto cíle, musel jsem vytvořit rychlý, ale také dostatečně obsáhlý scénář, pomocí kterého lze účinně vysvětlit látka tepelných motorů. Samotný výklad by mohl působit nudně a nezajímavě, proto jsem pro zpestření využil 3D animaci. Snažil jsem se aby bylo video srozumitelné a zároveň žáky zaujalo.}

\subsubsection{Plánování a scénář}
{Video je celé synchronizované s hlasovým komentářem, toto načasování by mohla rozhodit i drobná změna intonace či tempa výkladu. 3D animace je velmi složitý a časově náročný proces, abych předešel opakování práce, vytvořil jsem nejprce detailní scénář, kterého se později budu muset pevně držet.}\odst
{Na tomto videu jsem s nikým nespolupracoval, můj scénář proto nemusel být psán ve formátu skutečných scénářů pro film či reklamu. Důležité bylo, abych mu dokázal porozumět já sám, proto mi postačil jednoduchý textový dokument s barevným zvýrazněním. K jeho vytvoření jsem použil \code{Google docs}.}

\newpage

\subsection{Produkce}
{V této kapitole je popsána samotná tvorba videa, ta probíhala téměř výhradně v prostředí softwaru \code{Blenderu}. Nejprve jsem nahrál hlasový doprovod, poté jsem vytvořil modely, které jsem následně rozhýbal a obarvil. Nakonec jsem vše vyrenderoval a připravil na postprodukci.}

\subsubsection{Hlasový doprovod}
{Hlasový doprovod je základem celého videa, řídí se dle něj veškeré vizuály. Logicky byl tedy první částí, na které jsem začal pracovat.}\odst
{Zvuk jsem nahrával se zapůjčeným mikrofonem RØDE VideoMic Rycote. Tento mikrofon nelze přímo připojit k počítači, proto jsem jako mezičlánek využil kameru JVS GC-PX100 BE. Vše jsem umístil na stativ, aby byl mikrofon v přívětivé pozici. Hotová sestava je na \obrref{obr:nahravaciSestava}.}

\begin{figure}[H]
    \centering
    \includegraphics[scale=0.08]{images/MikrofonovaSestava.png}
    \caption{Nahrávací sestava \jaFoto}
    \label{obr:nahravaciSestava}
\end{figure}

{Dohromady jsem nahrál přibližně hodinové video, ze kterého jsem ponechal pouze zvukovou stopu, z té jsem odebral části kde bylo ticho nebo kde jsem udělal chybu při čtení scénáře; nakonec zbylo pouchý třiadvacet minut použitelného pro účely videa. V této části procesu jsem odhadoval, že video bude mít přibližně 8 minut. Scénář jsem tedy namluvil přibližně třikrát.}\odst
{Jelikož jsem nahrával v domácím prostředí, ve zvukové stopě šel slyšet šum. Tento šum jsem odebral pomocí softwaru \code{Audacity}, konkrétně jsem využil funkci \code{noise gate}.}\odst
{Naopak pro část kde odpočítávám start rakety jsem potřeboval zvuk degradovat, využil jsem opět software \code{Audacity}: přidal jsem šum, fázový posun a ozvěnu; omezil jsem vysoké a nízké frekvence a na závěr jsem snížil vzorkovací frekvenci na 8000.}

\newpage

\subsubsection{Modelování}
{Mimo 3D modely samotných motorů, jejichž tvorbu jsem vysvětlil v kapitole \scref{sc:animace}, jsem vytvořil i mnoho modelů pro dovysvětlení či vizuální zpestření, většinou jde o motory vozidel. V této kapitole vysvětlím 3D modelování ještě podrobněji.}\odst
{3D modelování je proces tvoření geometrie objektu, tedy modelu. Ty můžeme dělit na dva druhy: \textit{hard--surface} modely jsou charakterizovány ostrými hranami a pravidelnými tvary, zatímco \textit{soft--surface} modely obsahují organické a nepravidelné tvary. Většina mých modelů spadá do skupiny \textit{hard--surface}.}\odst
{Tvořit modely lze dvěma způsoby: destruktivní a nedestruktivní. Při destruktivním modelování měníme geometrii objektu přímo, posouváme body ve 3D prostoru a tím upravujeme tvar. Výhodou tohoto přístupu je jednoduchost a rychlost, nelze však změnit každý parametr. Nedestruktivní modelování spočívá ve tvorbě systémů které vytvoří geometrii, můžeme se proto kdykoliv vrátit a upravit jakoukoliv hodnotu, tento způsob je však velmi zdlouhavý a pro některé modely příliš nepraktický. Oba tyto způsoby mají své místo, pro většinu modelů jsem však zvolil destruktivní či kombinovaný přístup.}\odst
{Při tvorbě modelu velice pomůže předloha, ať už v podobě fotografií či technických plánků. Na \obrref{obr:predlohaFotky} je soubor referenčních obrázků pro tvorbu startovní plošiny pro raketu Saturn V organitovaný programem \code{PureRef}, \ref{obr:predlohaVykres} je technický výkres této plošiny.}

\begin{figure}[H]
    \centering
    \begin{subfigure}{.7\textwidth}
        \centering
        \includegraphics[scale=.15]{images/PureRef.png}
        \caption{Referenční obrázky pro tvorbu startovací plošiny, \\vytvořeno programem \code{PureRef}}
        \label{obr:predlohaFotky}
    \end{subfigure}%
    \begin{subfigure}{.3\textwidth}
        \centering
        \includegraphics[scale=.2]{images/technickyVykresStartovaciPlosina.jpg}
        \caption{Technický výkres startovací plošiny}
        \label{obr:predlohaVykres}
    \end{subfigure}
\end{figure}

{Po shromáždění obrázků začneme proces modelování, ten vysvětlím na modelu startovací plošiny. Nejprve si představíme, který základní tvar je nejbližší tvořenému objektu; v tomto případě je to krychle. Tuto krychli upravíme do přibližného tvaru startovací plošiny, pracujeme ve skutečném měřítku, aby nebyl objekt po \textit{importu} do hlavní scény správně velký v porovnání s ostatními objekty. K této krychli přidáme různé detaily, různé přístroje vytvoříme z menších krychlí a potrubí vytvoříme z válců.}

\newpage

{Abychom si usnadnili práci, využijeme osovou souměrnost plošiny a vytvoříme pouze půlku. Také díky scénáři víme, že bude tato plošina daleko od kamery a jakým úhlem bude natočena, nemusíme tedy ztrácet čas modelováním detailů, které nepůjdou vidět. Hotový model z pohledu kamery je na \obrref{obr:plosinaModel}, model ze zadní strany je na \obrref{obr:plosinaModelZezadu}.}

\begin{figure}[H]
    \centering

    \begingroup
    \makeatletter
    \renewcommand\thesubfigure{\thefigure~--~\@nameuse{subfiglabel@\alph{subfigure}}}
    \newcommand{\subfiglabel@a}{vlevo}
    \newcommand{\subfiglabel@b}{vpravo}
    \captionsetup[subfigure]{labelformat=simple, labelsep=colon}
    \renewcommand\p@subfigure{}
    \makeatother

    \begin{subfigure}{.48\textwidth}
        \centering
        \includegraphics[scale=.3]{images/raketaPlatformaModel.png}
        \caption{Model startovací plošiny z \\pohledu kamery \jaObr}
        \label{obr:plosinaModel}
    \end{subfigure}
    \hfill
    \begin{subfigure}{.48\textwidth}
        \centering
        \includegraphics[scale=.3]{images/raketaPlatformaModelZezadu.png}
        \caption{Model startovací plošiny \\zezadu \jaObr}
        \label{obr:plosinaModelZezadu}
    \end{subfigure}

    \endgroup

\end{figure}

{Model je dokončen, jeho kvalita je samozřejmě subjektivní, můžeme však hodnotit několik objektivních kritérií:}\odst 
{Měli bychom se snažit o ,,správnou'' topologii, obecně jde o to aby byl objekt tvořen přibližně stejně velkými čtyřúhelníky, důležitější je však aby nevznikaly žádné chyby při výpočtu stínů a aby se s modelem lehce pracovalo. Topologie startovací plošiny nevyjímečná.}\odst
{Další chybou jsou dvě plochy ve stejném místě, to vede k velice viditelnému artefaktu v podobě úplně černého místa. Tento problém nastal na věži startovací plošiny, kdy se některé tyče překrývaly. Jednoduchým řešením bylo posunout tyče o několik milimetrů, aby se plochy nepřekrývaly.}\odst
{Dále by měl model použít co nejméně bodů za zachování věrnosti předloze, velké množství bodů by mohlo vést ke zpomalení či naplnění paměti. Tato startovací plošina je tvořena dvěma sty tisíci body, celé video se skládá z \hl{X} bodů.}\odst
{Nejdůležitější je však jak model vypadá a zdali působí přesvědčivě jako předmět, kterým se snaží být.}

\newpage

\subsubsection{Materiály}
{Materiály ovládají, jak se světlo chová při interakci s objektem. Bez materiálu je startovací plošina šedá.}\odst
{Lze je tvořit dvěma způsoby: můžeme použít \textit{textury} (obrázky) a pokrýt jimi náš model. Tento způsob velice rychle docílí realistických výsledků, není však přizpůsobivý danému využití. Procedurální způsob naopak nabízí plnou kontrolu, tvoří ho jednoduché operace které dohromady říkají \code{Blenderu}, jak se má světlo chovat při interakci s objektem.}\odst
{Jelikož cílem tohoto projektu není věrná replikace skutečnosti a procedurální materiály jsou mi bližší, rozhodl jsem se je využít i pro tento projekt. Procedurální materiál se skládá z \textit{nodů} (uzly, které upravují parametry), které jsou do sebe vzájemně zapojeny. Existuje přes 80 \textit{nodů}, zde představím nejdůležitější z nich:}

\begin{itemize}
    \item \code{Principled BSDF}\par
        {Tento \textit{node} je srdcem většiny materiálů, patří do kategorie \textit{shaderů}, ty udávají jak se má světlo chovat. Umožňuje nastavit například barvu, kovovost, drsnost, průhlednost či normály (ty vytváří iluzi trojrozměrného povrchu bez přidání geometrie).}
    \item \code{Principled Volume}\par
        {Pokud jde o materiál plynu, využijeme tento \textit{shader}. Můžeme nastavit hustotu, barvu, anizotropie či záření plynu.}
    \item \code{Noise Texture}\par
        {Tento uzel patří mezi generující \textit{nody}. Generuje náhodný šum, můžeme upravit jeho velikost, detail, drsnost a vlnivost.}
    \item \code{Math Node}\par
        {Patří mezi upravující \textit{nody}, upravuje vstupní parametry pomocí různých aritmetických, trigonometrických a logických operací.}
    \item \code{Texture Coordinate}\par
        {Umožňuje přístup k datům geometrie; můžeme získat například souřadnice, normály či UV mapu (ta se používá pro mapování 2D obrázků na 3D povrch).}
\end{itemize}

{Materiály se tvoří v \textit{shader editoru}, v tomto okně přidáváme \textit{nody}, viz \obrref{obr:shaderEditor}.}

\begin{figure}[H]
    \centering
    \includegraphics[scale=0.45]{images/shaderEditor.png}
    \caption{Tvorba materiálu v \textit{shader editoru} \jaObr}
    \label{obr:shaderEditor}
\end{figure}

\newpage

\subsubsection{Animace}
{Modely jsou hotové, ale statické. Pro jejich pohyb je třeba je animovat. Využil jsem klíčovou animaci, ta je založena na přídávání klíčových snímků. Klíčový snímek, neboli klíčová hodnota, obsahuje informace o tom, jakou hodnotu má mít daný parametr v daný čas. Můžeme tak nastavit pozici, rotaci a velikost objektů, upravit parametry jejich materiálů a další.}\odst
{Pro detailnější ovládání klíčových snímků jsem využil \code{graph editor}, ten umožňuje přesněji měnit časování a nuance pohybu, ukázka \code{graph editoru} je na \obrref{obr:grafEditor}.}

\begin{figure}[H]
    \centering
    \includegraphics[scale=0.5]{images/graphEditor.png}
    \caption{Zobrazení animace v \code{graph editoru} \jaObr}
    \label{obr:grafEditor}
\end{figure}

{Animace musela být synchronizována s komentářem, ten byl nastaven tak, aby se automaticky přehrával spolu s ní. Mimo klíčovou animaci jsem využil i procedurální animaci pomocí \textit{geometry nodes}. Tento přístup jsem zvolil pro jednoduché animace, které mohou být vyjádřeny předpisy. Například animace pístů, ojnic, hřídelí, ventilů atd. Tímto způsobem jsem vytvořil i povrch vody v části ,,Parní motory''.}\odst
{Posledním způsobem animace je simulace, tu jsem využil pro vytvoření fyzikálně založených objektů jako jsou například tuhá tělesa, částice, látky či plyny. Při tvorbě simulací je nutné správně nastavit vstupní hodnoty a poté nechat počítač simulaci vypočítat. Simulace často nevypadá tak jak bychom si ji představovali, proto upravíme parametry a proces opakujeme, dokud nejsou výsledky uspokojivé. Příklad simulace a parametrů naleznete na \obrref{obr:simulace}.}

\begin{figure}[H]
    \centering
    \begin{subfigure}{0.35\textwidth}
        \centering
        \includegraphics[scale=0.3]{images/simulaceParametry.png}
        \caption{Parametry simulace\\\jaObr}
    \end{subfigure}%
    \begin{subfigure}{0.6\textwidth}
        \centering
        \includegraphics[scale=0.6]{images/simulaceVysledek.png}
        \caption{Výsledný simulace \jaObr}
    \end{subfigure}
    \caption{Proces simulace v softwaru \code{Blender}}
    \label{obr:simulace}
\end{figure}

\newpage

\subsubsection{Skutečné záběry}
\textcolor{red}{Tato část ještě nebyla nahrána.}\odst
{Abych propojil obsah videa se skutečností a tím poukázal na to, proč jsou tepelné motory důležité, rozhodl jsem se do videa přidat skutečné záběry. Záběry jsou pořízeny za jízdy z automobilu pomocí kamery \code{GoPro HERO12 Black}.}\odst
{Pro plynulý přehod mezi animací a živou částí jsem vytvořil v \code{Blenderu} plátno, na které se začnou promítat skutečné záběry. \code{Blender} camera se přiblíží a plynule se obraz změní na záběry ze skutečné kamery.}

\newpage

\subsubsection{Renderování}\label{sc:renderovani}
{Trojrozměrná scéna je pouze sbírka nejrůznějších hodnot jako jsou pozice objektů, jijch geometrie, textury, síla světel a nesmírně mnoho dalších. Tyto data se pomocí procesu zvaného renderování převedou na obrázky (což jsou vlastně také jen data). Počítač simuluje odrazy paprsků světla a jejich dopady na kameru, poté zaznamená barvu a intenzitu do hodnoty pixelu, na který světlo dopadlo. Konkrétně jsem využil \textit{path tracing} \textit{render engine} \code{Cycles}, který věrně mimikuje skutečné světlo.}\odst
{Pro správné renderování bylo zapotřebí nastavit hodnoty, se kterými bude počítač pracovat. Pokud bych je nastavil mírně, byl by výsledný obraz zrnitý nebo by obsahovat různé artefakty; naopak kdybych je nastavil příliš přísně, renderování by trvalo velice dlouho. Na základě vlastních zkušeností a experimentace jsem zvolil tyto hodnoty (jen nejdůležitější): }

\begin{itemize}
    \item {\code{Resolution} -- 1920x1080 (HD)}\par
        {Rozlišení výsledného videa, 1920x1080 je v současnosti jedno z nejrozšířenějších rozlišení. Pokud bych zvolil 2560x1440 či vyšší, čas renderování by se velice zvýšil.}
    \item {\code{Frame Rate} -- 24}\par
        {Počet snímků za sekundu videa, 24 je nejnižší standard který se používá.}
    \item {\code{Max Samples} -- 1024}\par
        {Nejvyšší počet paprsků, které dopadnou na jeden pixel. Výchozí hodnota je 4096, ta je však přemrštěná a pro mé účely prakticky nedosažitelná.}
    \item {\code{Persistent Data} -- Zapnuto}\par
        {Zachovává data v operační paměti, razantně zrychlí renderování na úkor zvýšeného využívání operační paměť (desítky GB).}
    \item {\code{View Transform} -- \code{AgX}}\par
        {Způsob, kterým se vyhodnotí barva pixelu. Hlavní předností \code{AgX} je vyblednutí světlých objektů, stejně jako u skutečných kamer.}
\end{itemize}

{Dobu renderování však nejvíce ovlivňuje použitý \textit{hardware}. Toto video jsem renderoval na velice výkonném pracovním počítači opatřeným grafickou kartou Nvidia RTX 4080, procesorem AMD 9 7900X a 32 GB operační paměti.}\odst
{Produktem renderování v tomto případě není video, ale tisíce obrázků formátu \code{exr}. Tento způsob ukládání přináší mnoho výhod: render může být přerušen, části videa mohou být zpětně upraveny bez nutnosti opakovaného renderování a je zachována vysoká kvalita.}

\newpage

{Do metadat obrázků jsem ukládal čas a paměť potřebnou k vykreslení snímku, pomocí vlastního programu napsaného v jazyce \code{Python} jsem vytvořil grafy, ty naleznete na \obrref{obr:grafRender}. Dohromady se video renderovalo \hl{X} hodin, což je \hl{X} dní nepřetržitého renderování. I přes obrovskou časovou náročnost nebylo renderování omezujícím faktorem, protože převážně v době, kdy nemohla být prováděna jiná práce (například v noci).}

\begin{figure}[H]
    \centering
    \begin{subfigure}{\textwidth}
        \centering
        \includegraphics[width=\textwidth]{images/grafUvod.png}
        \caption{Graf času a paměti potřebné k renderování části ,,Úvod'' \jaGraf}\vspace{0.25cm}
    \end{subfigure}
    \begin{subfigure}{\textwidth}
        \includegraphics[width=\textwidth]{images/grafUvod.png}
        \caption{Graf času a paměti potřebné k renderování části ,,Parní motory'' \jaGraf}\vspace{0.25cm}
    \end{subfigure}
    \begin{subfigure}{\textwidth}
        \includegraphics[width=\textwidth]{images/grafUvod.png}
        \caption{Graf času a paměti potřebné k renderování části ,,Spalovací motory'' \jaGraf}\vspace{0.25cm}
    \end{subfigure}
    \begin{subfigure}{\textwidth}
        \includegraphics[width=\textwidth]{images/grafUvod.png}
        \caption{Graf času a paměti potřebné k renderování části ,,Reaktivní motory'' \jaGraf}\vspace{0.25cm}
    \end{subfigure}
    \begin{subfigure}{\textwidth}
        \includegraphics[width=\textwidth]{images/grafUvod.png}
        \caption{Graf času a paměti potřebné k renderování části ,,Závěr'' \jaGraf}
    \end{subfigure}
    \caption{Grafy času a paměti potřebné k renderování}
    \label{obr:grafRender}
\end{figure}

{Tyto grafy jsou užitečné pro určení výpočetně náročných částí, obecně je nejnáročnější renderovat plyny, velké množství bodů a složité geometrie (řádově statisíce bodů).}

\newpage

\subsection{Postprodukce}
\textcolor{red}{Tato část ještě nebyla dokončena, v této kapitole popíši dokončování videa}

\subsubsection{Střih a kompozice}
\textcolor{red}{Tato část ještě nebyla dokončena}

\subsubsection{Zvukový design}
\textcolor{red}{Tato část ještě nebyla dokončena.}

\subsubsection{Export a distribuce}
\textcolor{red}{Tato část ještě nebyla dokončena.}

\newpage

\section{Prezentace}
{Prezentace jsem tvořil v softwaru \code{Google Slides}, jsou dostupné v přílohách (\ref{pr:prezentaceTepelneMotory}, \ref{pr:prezentaceParniMotory}, \ref{pr:prezentaceSpalovaciMotory} a \ref{pr:prezentaceReaktivniMotory}). Prezentace ,,Tepelné motor'' zabere přibližně jednu vyučovací hodinu, a to i s příklady. Ostatní prezentace trvají přibližně 20 minut.}\odst
{Cílem prezentací bylo účinně popsat látku tepelných motorů, toho jsem docílil stručnými body. Historie v prezentacích není příliš zastoupena, zaměřil jsem se na fungování a využití daných motorů. K vysvětlení fungování velmi pomohly animace.}

\subsection{Vizuální stránka prezentací}
{Pro vizuální stránku prezentací jsem zvolil tmavě červené detaily doplněné ilustračními obrázky vozidel, ve kterých se daný typ motoru používá. Na \obrref{obr:tatra813} jsou popsány jednotlivé části tvorby takového obrázku.}

\begin{figure}[H]
    \begin{subfigure}{0.5\textwidth}
        \centering
        \includegraphics[scale=.11]{images/tatra813_1.png}\\
        \caption{První hrubý model}
    \end{subfigure}
    \begin{subfigure}{0.5\linewidth}
        \centering
        \includegraphics[scale=.11]{images/tatra813_5.png}\\
        \caption{Hotový model}
    \end{subfigure}
    \begin{subfigure}{0.5\linewidth}
        \centering
        \includegraphics[scale=.11]{images/tatra813_6.png}\\
        \caption{Model s texturamy}
    \end{subfigure}
    \begin{subfigure}{0.5\linewidth}
        \centering
        \includegraphics[scale=.22]{images/tepelneMotoryObecneTatra.png}\\
        \caption{Hotový snímek z prezentace}
    \end{subfigure}
    \caption{Proces tvorby ilustračního obrázku nákladního automobilu Tatra 813 \jaObr}
    \label{obr:tatra813}
\end{figure}

{Dále jsem vytvořil obrázky, na kterých jde názorně vidět princip fungování daného motoru: pro parní motory jsem vytvořil obrázek tlakového hrnce, pro spalovací motory píst tlačen výbuchem a pro reaktivní motory židli tlačenou hasicími přístroji.}

\newpage

\section{Pracovní listy}
{Vytvořil jsem 3 pracovní listy, jeden určený pro základní školy a dva určeny pro střední školy. Vytvořil jsem je pomocí sázecího softwaru \LaTeX, stejně jako celý tento dokument. Snažil jsem se, aby byly obsahově vyhovující a nebyly příliš obtížné, žáci by měli být schopni získat dobrou známku pouze ze studování materiálů, které jsem vytvořil. Také mi šlo o dobrou grafickou úpravu a použití obrázků, které jsou rozluštitelné i když jsou vytisknuty černobíle.}\odst
{K těmto pracovním listům jsem samozřejmě vytvořil i řešení, podle kterého je může učitel snadno opravit.}
\subsection{Návrh na hodnocení}
\textcolor{red}{Návrh na hodnocení se odvíjí od výuky, která proběhne 27. a 28. února, není tedy ještě hotová.}

\newpage

\section{Použití pomůcek v praxi}\label{sc:pouzitiVPraxi}
{Pracovní list pro základní školy a prezentaci ,,Tepelné motory'' jsem vyzkoušel na žácích druhého ročníku Gymnázia Arabská. Učil jsem dvě hodiny, první hodinu jsem učil třídu 2.B. Nejprve jsem ukázal prezentaci a poté rozdal pracovní papíry jako domácí úkol. Ve druhé hodině jsem učil třídu 2.A, rozdal jsem pracovní listy před hodinou, aby je žáci vyplňovali současně s prezentací a více se soustředili, termín odevzdání byl o několik hodin později.}\odst
{Mé materiály samozřejmě nebyly perfektní. Prezentace potřebovala drobné úpravy, její následnost byla místy neohrabaná a nelogická, tyto problémy jsem opravil upravením pořadí snímků a změnou či přidáním obrázků.}\odst
{Žáci měli problémy s identifikací svíčky, přidal jsem tedy obrázek svíčky k vysvětlování zážehových motorů, aby na něj učitel mohl upozornit. Dále bylo složité vysvětlit počet pracovních dob za otáčku hřídele. Přidal jsem popisný text ke vzorci výkonu spalovacích motorů. Upravené snímky jsou na \obrref{obr:svickaSlide} a \obrref{obr:vzorceSlide}.}
\begin{figure}[H]
    \begingroup
    \makeatletter
    \renewcommand\thesubfigure{\thefigure~--~\@nameuse{subfiglabel@\alph{subfigure}}}
    \newcommand{\subfiglabel@a}{vlevo}
    \newcommand{\subfiglabel@b}{vpravo}
    \captionsetup[subfigure]{labelformat=simple, labelsep=colon}
    \renewcommand\p@subfigure{}
    \makeatother
    \begin{subfigure}{0.47\textwidth}
        \centering
        \setlength{\fboxsep}{0pt}
        \fbox{\includegraphics[scale=0.2]{assets/images/SvickaPrezentace.png}}
        \caption{Obrázek svíčky u vysvětlení\\čtyřdobých zážehových motorů \jaObr}
        \label{obr:svickaSlide}
    \end{subfigure}\hfill
    \begin{subfigure}{0.47\textwidth}
        \centering
        \setlength{\fboxsep}{0pt}
        \fbox{\includegraphics[scale=0.2]{assets/images/PracovniDoby1.png}}
        \caption{Přidání vysvětlení počtu dob za otáčku \jaObr}
        \label{obr:vzorceSlide}
    \end{subfigure}
    \endgroup
\end{figure}
{Také jsem vytvořil nový obrázek pro znázornění akce a reakce, jelikož starý obrázek židle (viz \obrref{obr:zidlePrezentace}) tlačené hasicími přístroji byl nevyhovující. Nový obrázek \obrref{obr:balonekPrezentace} odlétajícího nafukujícího balónku je mnohem názornější a pochopitelnější.}
\begin{figure}[H]
    \begingroup
    \makeatletter
    \renewcommand\thesubfigure{\thefigure~--~\@nameuse{subfiglabel@\alph{subfigure}}}
    \newcommand{\subfiglabel@a}{vlevo}
    \newcommand{\subfiglabel@b}{vpravo}
    \captionsetup[subfigure]{labelformat=simple, labelsep=colon}
    \renewcommand\p@subfigure{}
    \makeatother
    \begin{subfigure}{0.47\textwidth}
        \centering
        \setlength{\fboxsep}{0pt}
        \fbox{\includegraphics[scale=0.2]{assets/images/zidlePrezentace.png}}
        \caption{Obrázek židle s hasicími přístroji u vysvětlení zákona akce a reakce \jaObr}
        \label{obr:zidlePrezentace}
    \end{subfigure}\hfill
    \begin{subfigure}{0.47\textwidth}
        \centering
        \setlength{\fboxsep}{0pt}
        \fbox{\includegraphics[scale=0.2]{assets/images/balonekPrezentace.png}}
        \caption{Obrázek nafukovacího balónku u vysvětlení zákona akce a reakce \jaObr}
        \label{obr:balonekPrezentace}
    \end{subfigure}
    \endgroup
\end{figure}
{Dále jsem musel opravit chyby v příkladech. Jelikož jsem příklady mnohokrát předělával, výsledky nebyli vždy správné. V hotové verzi je toto samozřejmě opraveno. Některým žákům dělaly příklady potíže, především kvůli jejich problémům s převody jednotek. Příklady však s drobnou asistencí žáci vyřešili správně.}
\newpage
\subsection{Nedostatky a úprava pracovních listů}
{Pracovní listy jsem zadal jako dobrovolný domácí úkol, termín na odevzdání byl stanoven na tentýž den. Žáci pracovali ve dvojicích. Výsledky jsou shrnuty v \tabref{tab:vysledkyPracovnichListu}.}
\begin{table}[H]
    \centering
    \begin{tabular}{|c|c|c|c|}
        \hline
        Cvičení & úspěšnost třídy 2.B & úspěšnost třídy 2.A & celková úspěšnost\\
        \hline
        Celkem & 73,1 \% & 80,0 \% & 76,5 \%\\
        Cvičení 1 & 76,9 \% & 77,7 \% & 77,3 \%\\
        Cvičení 2 & 97,4 \% & 97,2 \% & 97,3 \%\\
        Cvičení 3 & 94,9 \% & 100 \% & 97,3 \%\\
        Cvičení 4 & 69,2 \% & 100 \% & 84,0 \%\\
        Cvičení 5 & 46,2 \% & 91,7 \% & 68,0 \%\\
        Cvičení 6 & 76,9 \% & 66,7 \% & 72,0 \%\\
        Cvičení 7 & 38,4 \% & 87,5 \% & 62,0 \%\\
        Cvičení 8 & 64,1 \% & 52,8 \% & 58,7 \%\\
        Cvičení 9 & 56,4 \% & 36,1 \% & 46,7 \%\\
        Cvičení 10 & 84,6 \% & 86,1 \% & 85,3 \%\\
        \hline
    \end{tabular}
    \caption{Hodnocení pracovních listů \jaTab}
    \label{tab:vysledkyPracovnichListu}
\end{table}
{Zkušenosti z výuky jsem využil ke zdokonalení mých materiálů. Snížil jsem počet bodů za jednoduchá cvičení 2 a 3. Ve cvičení 4 žáci doplňovali hodnoty do tabulky, někteří žáci (převážně ze třídy 2.B) toto cvičení přeskočili, pravděpodobně bylo příliš dlouhé. Ve finální verzi jsem ho zkrátil. Cvičení 5 a 6 navazují na cvičení 4, což je jasně řečeno v zadání. Žáci zadání často nečetli a tak jsem se rozhodl upravit rozvržení pracovního papíru abych zdůraznil jejich návaznost. Původní rozvržení je znázorněno na \obrref{obr:puvodniRozvrzeni}, upravené rozvržení na \obrref{obr:upraveneRozvrzeni}.}
\begin{figure}[H]
    \begingroup
    \makeatletter
    \renewcommand\thesubfigure{\thefigure~--~\@nameuse{subfiglabel@\alph{subfigure}}}
    \newcommand{\subfiglabel@a}{vlevo}
    \newcommand{\subfiglabel@b}{vpravo}
    \captionsetup[subfigure]{labelformat=simple, labelsep=colon}
    \renewcommand\p@subfigure{}
    \makeatother
    \begin{subfigure}{0.45\textwidth}
        \centering
        \begin{tikzpicture}[scale=0.23]
            \draw[thick] (0,0) rectangle (21, 29.7);
            \draw[thick](0,27.5) to (21,27.5) (0,22) to (21,22) (0,18) to (21,18) (11,18) to (11,0) (0,9) to (11,9) (11,14.5) to (21,14.5) (11,11) to (21,11);
            \node[right] at (0,28.6) {Jméno a hodnocení};
            \node[right] at (0,26.5) {Cvičení 1};
            \node[right] at (0,21) {Cvičení 2};
            \node[right] at (0,17) {Cvičení 3};
            \node[right] at (0,8) {Cvičení 4};
            \node[right] at (11,17) {Cvičení 5};
            \node[right] at (11,13.5) {Cvičení 6};
            \node[right] at (11,10) {Cvičení 7};
        \end{tikzpicture}
        \caption{Původní rozvržení první strany pracovního listu \jaDiag}
        \label{obr:puvodniRozvrzeni}
    \end{subfigure}\hfill
    \begin{subfigure}{0.45\textwidth}
        \centering
        \begin{tikzpicture}[scale=0.23]
            \draw[thick] (0,0) rectangle (21, 29.7);
            \draw[thick](0,27.5) to (21,27.5) (0,22) to (21,22) (0,18) to (21,18) (10.5,18) to (10.5,11) (0,11) to (10.5,11) (10.5,14.5) to (21,14.5) (10.5,11) to (21,11) (0, 10.75) to (21, 10.75) (11,10.75) to (11,0);
            \node[right] at (0,28.6) {Jméno a hodnocení};
            \node[right] at (0,26.5) {Cvičení 1};
            \node[right] at (0,21) {Cvičení 2};
            \node[right] at (0,17) {Cvičení 4};
            \node[right] at (0,9.5) {Cvičení 3};
            \node[right] at (11,17) {Cvičení 5};
            \node[right] at (11,13.5) {Cvičení 6};
            \node[right] at (11,9.5) {Cvičení 7};
        \end{tikzpicture}
        \caption{Upravené rozvržení první strany pracovního listu \jaDiag}
        \label{obr:upraveneRozvrzeni}
    \end{subfigure}
    \endgroup
\end{figure}
{Dále jsem upravil obrázky pro lepší tisk a změnil jsem obrázek ze cvičení 8, jelikož měli žáci potíže rozeznat některé malé součástky. Nakonec jsem vyměnil pořadí posledních dvou cvičení, aby byl lehčí příklad první.}
\newpage
\subsection{Vyhodnocení praktického použití pomůcek}
{Při výuce se ukázalo mnoho chyb prezentace, pracovních listů i mého samotného výstupu. Tyto zkušenosti jsem využil ke zdokonalení svých výukových pomůcek, a to nejen těch, které jsem přímo v hodinách využil.}\odst
{Prezentaci jsem upravil nejméně, přidal jsem několik obrázků a lehce upravil text. Pracovní listy pro základní školy jsem zdokonalil a zpřehlednil; principy, které jsem se naučil, jsem aplikoval i na pracovní listy pro střední školy. Nejvíce jsem změnil svůj výklad, ten byl ve druhé hodině výrazně lepší.}\odst
{Z těchto dvou hodin jsem se naučil mnoho, a doufám, že žáci také. V budoucnosti bych rád vyzkoušel další materiály na různorodějších třídách, například i na základní škole.}

\newpage

\section{Závěr}
{Po dokončení původního ročníkového projektu jsem se rozhodl pokračovat do soutěže SOČ s tím, že výukové pomůcky jen lehce vylepším a přepracuji. Nakonec jsem svou původní práci podstatně rozšířil: vylepšil jsem modely, předělal prezentace a pracovní listy a samozřejmě jsem vytvořil i animované výukové video.}\odst
{Dále jsem tyto materiály vyzkoušel v praxi, tato zkušenost mi přinesla důležité zkušenosti a informace, které jsem využil k dalšímu zdokonalení mých materiálů. Jsem spokojen s dosavadním postupem mé práce, budu pracovat na dokončení animovaného videa, které považuji za jeden z nejdůležitějších výstupů mé práce.}\odst
{Přestože se 3D modelováním v \code{Blenderu} věnuji již pátým rokem, naučil jsem se díky tomuto projektu nové vědomosti, převážně v organizaci velkých projektů. Doufám, že mé materiály budou využívány pro výuku tepelných motorů nejen na Gymnáziu Arabská a že je budou mít rádi učitelé i žáci.}

\newpage

\section{Seznam použitých zdrojů}

\printbibliography[heading=none]

\newpage

\section{Přílohy}
\subsection*{Animace}
\begin{enumerate}[align=left, labelwidth=0cm, label={Příloha č. }\arabic*{: }]
    \item \label{pr:animaceParniStroj} {Animace parního stroje: formát }\textattachfile[color=0 0 0]{assets/renders/animations/parniStroj.mp4}{\textit{mp4}}{ nebo }\textattachfile[color=0 0 0]{assets/renders/gifs/parniStroj.gif}{\textit{gif}}{.}
    \item \label{pr:animaceParniTurbina} {Animace parní turbíny: formát }\textattachfile[color=0 0 0]{assets/renders/animations/parniTurbina.mp4}{\textit{mp4}}{ nebo }\textattachfile[color=0 0 0]{assets/renders/gifs/parniTurbina.gif}{\textit{gif}}{.}
    \item \label{pr:animaceCtyrdobyZazehovyMotor} {Animace čtyřdobého zážehového motoru: formát }\textattachfile[color=0 0 0]{assets/renders/animations/ctyrdobyZazehovyMotor.mp4}{\textit{mp4}}{ nebo }\textattachfile[color=0 0 0]{assets/renders/gifs/ctyrdobyZazehovyMotor.gif}{\textit{gif}}{.}
    \item \label{pr:animaceCtyrdobyVznetovyMotor} {Animace čtyřdobého vznětového motoru: formát }\textattachfile[color=0 0 0]{assets/renders/animations/ctyrdobyVznetovyMotor.mp4}{\textit{mp4}}{ nebo }\textattachfile[color=0 0 0]{assets/renders/gifs/ctyrdobyVznetovyMotor.gif}{\textit{gif}}{.}
    \item \label{pr:animaceDvoudobyZazehovyMotor} {Animace dvoudobého zážehového motoru: formát }\textattachfile[color=0 0 0]{assets/renders/animations/dvoudobyZazehovyMotor.mp4}{\textit{mp4}}{ nebo }\textattachfile[color=0 0 0]{assets/renders/gifs/dvoudobyZazehovyMotor.gif}{\textit{gif}}{.}
    \item \label{pr:animacePevnyRaketovyMotor} {Animace raketového motoru na tuhé palivo: formát }\textattachfile[color=0 0 0]{assets/renders/animations/raketovyMotorPevny.mp4}{\textit{mp4}}{ nebo }\textattachfile[color=0 0 0]{assets/renders/gifs/raketovyMotorPevny.gif}{\textit{gif}}{.}
    \item \label{pr:animaceKapalnyRaketovyMotor} {Animace raketového motoru na kapalné palivo: formát }\textattachfile[color=0 0 0]{assets/renders/animations/raketovyMotorKapalny.mp4}{\textit{mp4}}{ nebo }\textattachfile[color=0 0 0]{assets/renders/gifs/raketovyMotorKapalny.gif}{\textit{gif}}{.}
    \item \label{pr:animaceProudovyMotor} {Animace jednoproudového motoru: formát }\textattachfile[color=0 0 0]{assets/renders/animations/proudovyMotor.mp4}{\textit{mp4}}{ nebo }\textattachfile[color=0 0 0]{assets/renders/gifs/proudovyMotor.gif}{\textit{gif}}{.}
    \item \label{pr:animaceDvouproudovyMotor} {Animace dvouproudového motoru: formát }\textattachfile[color=0 0 0]{assets/renders/animations/dvouproudovyMotor.mp4}{\textit{mp4}}{ nebo }\textattachfile[color=0 0 0]{assets/renders/gifs/dvouproudovyMotor.gif}{\textit{gif}}{.}
\end{enumerate}

\subsection*{Video}
\begin{enumerate}[align=left, labelwidth=0cm, label={Příloha č. }\arabic*{: }, resume]
    \item \label{pr:videoTepelneMotory} {Animované výukové video: }\textattachfile[color=0 0 0]{assets/renders/video/TepelneMotory-3DAnimovaneVideo.mp4}{\textit{mp4}}{ nebo \href{https://www.youtube.com/watch?v=vvIy9x1V8KU}{\textit{online}}.}
\end{enumerate}

\subsection*{Prezentace}
\begin{enumerate}[align=left, labelwidth=0cm, label={Příloha č. }\arabic*{: }, resume]
    \item \label{pr:prezentaceTepelneMotory} {Prezentace ,,Tepelné motory'': }\textattachfile[color=0 0 0]{assets/renders/presentations/TepelneMotory.pdf}{\textit{pdf}}{ nebo \href{https://docs.google.com/presentation/d/1xBrIPiTZJOHme3T_spH1y9jdtNlOpabc-S7u2acBesY/edit?usp=sharing}{\textit{online}}.}
    \item \label{pr:prezentaceParniMotory} {Prezentace ,,Parní motory'': }\textattachfile[color=0 0 0]{assets/renders/presentations/TepelneMotory.pdf}{\textit{pdf}}{ nebo \href{https://docs.google.com/presentation/d/1jMvo6rMvbJpfOmJiKJR7ShKW5ZFC9MJw8l2pMkFMbv0/edit?usp=sharing}{\textit{online}}.}
    \item \label{pr:prezentaceSpalovaciMotory} {Prezentace ,,Spalovací motory'': }\textattachfile[color=0 0 0]{assets/renders/presentations/TepelneMotory.pdf}{\textit{pdf}}{ nebo \href{https://docs.google.com/presentation/d/147L6ILcjG5YR67yuBgRODnFNM2z2Ke75k0EHgOf31J4/edit?usp=sharing}{\textit{online}}.}
    \item \label{pr:prezentaceReaktivniMotory} {Prezentace ,,Reaktivní motory'': }\textattachfile[color=0 0 0]{assets/renders/presentations/TepelneMotory.pdf}{\textit{pdf}}{ nebo \href{https://docs.google.com/presentation/d/1W9alCUAWTDiDOqy4tcJGXQX2FINbPNP93r4UaerzI_g/edit?usp=sharing}{\textit{online}}.}
\end{enumerate}

\subsection*{Pracovní listy}
\begin{enumerate}[align=left, labelwidth=0cm, label={Příloha č. }\arabic*{: }, resume]
    \item \label{pr:pracovniListZS} {Pracovní list pro ZŠ: }\textattachfile[color=0 0 0]{pracovniListy/pracovniListTepelneMotoryZS.pdf}{\textit{pdf}}{.}
    \item \label{pr:pracovniListZSreseni} {Řešení pracovního listu pro ZŠ: }\textattachfile[color=0 0 0]{pracovniListy/pracovniListTepelneMotoryZSreseni.pdf}{\textit{pdf}}{.}
    \item \label{pr:pracovniListSSvarA} {Pracovní list pro SŠ (Var. A): }\textattachfile[color=0 0 0]{pracovniListy/pracovniListTepelneMotorySSVarA.pdf}{\textit{pdf}}{.}
    \item \label{pr:pracovniListSSvarAreseni} {Řešení pracovního listu pro SŠ (Var. A): }\textattachfile[color=0 0 0]{pracovniListy/pracovniListTepelneMotorySSVarAreseni.pdf}{\textit{pdf}}{.}
    \item \label{pr:pracovniListSSvarB} {Pracovní list pro SŠ (Var. B): }\textattachfile[color=0 0 0]{pracovniListy/pracovniListTepelneMotorySSVarB.pdf}{\textit{pdf}}{.}
    \item \label{pr:pracovniListSSvarBreseni} {Řešení pracovního listu pro SŠ (Var. B): }\textattachfile[color=0 0 0]{pracovniListy/pracovniListTepelneMotorySSVarBreseni.pdf}{\textit{pdf}}{.}
\end{enumerate}

\newpage

% \section{Přílohy}

%Animace
%\arbitraryref{pr:animaceParniStroj}{Animace parního stroje}
{Animace \textbf{parního stroje} je dostupná ke stažení ve formátu }\textattachfile[color=0 0 0]{assets/renders/animations/parniStroj.mp4}{\textit{mp4}}{ nebo }\textattachfile[color=0 0 0]{assets/renders/gifs/parniStroj.gif}{\textit{gif}}{.}

\arbitraryref{pr:animaceParniTurbina}{Animace parní turbíny}
{Animace \textbf{parní turbíny} je dostupná ke stažení ve formátu }\textattachfile[color=0 0 0]{assets/renders/animations/parniTurbina.mp4}{\textit{mp4}}{ nebo }\textattachfile[color=0 0 0]{assets/renders/gifs/parniTurbina.gif}{\textit{gif}}{.}

\arbitraryref{pr:animaceCtyrdobyZazehovyMotor}{Animace čtyřdobého zážehového motoru}
{Animace \textbf{čtyřdobého zážehového motoru} je dostupná ke stažení ve formátu }\textattachfile[color=0 0 0]{assets/renders/animations/ctyrdobyZazehovyMotor.mp4}{\textit{mp4}}{ nebo }\textattachfile[color=0 0 0]{assets/renders/gifs/ctyrdobyZazehovyMotor.gif}{\textit{gif}}{.}

\arbitraryref{pr:animaceCtyrdobyVznetovyMotor}{Animace čtyřdobého vznětového motoru}
{Animace \textbf{čtyřdobého vznětového motoru} je dostupná ke stažení ve formátu }\textattachfile[color=0 0 0]{assets/renders/animations/ctyrdobyVznetovyMotor.mp4}{\textit{mp4}}{ nebo }\textattachfile[color=0 0 0]{assets/renders/gifs/ctyrdobyVznetovyMotor.gif}{\textit{gif}}{.}

\arbitraryref{pr:animaceDvoudobyZazehovyMotor}{Animace dvoudobého zážehového motoru}
{Animace \textbf{dvoudobého zážehového motoru} je dostupná ke stažení ve formátu }\textattachfile[color=0 0 0]{assets/renders/animations/dvoudobyZazehovyMotor.mp4}{\textit{mp4}}{ nebo }\textattachfile[color=0 0 0]{assets/renders/gifs/dvoudobyZazehovyMotor.gif}{\textit{gif}}{.}

\arbitraryref{pr:animacePevnyRaketovyMotor}{Animace raketového motoru na tuhé palivo}
{Animace \textbf{raketového motoru na tuhé palivo} je dostupná ke stažení ve formátu }\textattachfile[color=0 0 0]{assets/renders/animations/raketovyMotorPevny.mp4}{\textit{mp4}}{ nebo }\textattachfile[color=0 0 0]{assets/renders/gifs/raketovyMotorPevny.gif}{\textit{gif}}{.}

\arbitraryref{pr:animaceKapalnyRaketovyMotor}{Animace raketového motoru na kapalné palivo}
{Animace \textbf{raketového motoru na kapalné palivo} je dostupná ke stažení ve formátu }\textattachfile[color=0 0 0]{assets/renders/animations/raketovyMotorKapalny.mp4}{\textit{mp4}}{ nebo }\textattachfile[color=0 0 0]{assets/renders/gifs/raketovyMotorKapalny.gif}{\textit{gif}}{.}

\arbitraryref{pr:animaceProudovyMotor}{Animace proudového motoru}
{Animace \textbf{proudového motoru} je dostupná ke stažení ve formátu }\textattachfile[color=0 0 0]{assets/renders/animations/proudovyMotor.mp4}{\textit{mp4}}{ nebo }\textattachfile[color=0 0 0]{assets/renders/gifs/proudovyMotor.gif}{\textit{gif}}{.}

\arbitraryref{pr:animaceDvouproudovyMotor}{Animace dvouproudového motoru}
{Animace \textbf{dvouproudového motoru} je dostupná ke stažení ve formátu }\textattachfile[color=0 0 0]{assets/renders/animations/dvouproudovyMotor.mp4}{\textit{mp4}}{ nebo }\textattachfile[color=0 0 0]{assets/renders/gifs/dvouproudovyMotor.gif}{\textit{gif}}{.}

%Videa
\arbitraryref{pr:videoTepelneMotory}{Video ,,Tepelné motory''}

%Prezentace
\arbitraryref{pr:prezentaceTepelneMotory}{Prezentace ,,Tepelné motory''}
{Prezentace ,,Tepelné motory'' je dostupná online \href{https://docs.google.com/presentation/d/1xBrIPiTZJOHme3T_spH1y9jdtNlOpabc-S7u2acBesY/edit?usp=sharing}{zde}.}
\arbitraryref{pr:prezentaceParniMotory}{Prezentace ,,Parní motory''}
\arbitraryref{pr:prezentaceSpalovaciMotory}{Prezentace ,,Spalovací motory''}
\arbitraryref{pr:prezentaceReaktivniMotory}{Prezentace ,,Reaktivní motory''}

\newpage
%Pracovní listy
\arbitraryref{pr:pracovniListy}{Pracovní listy}
\arbitraryref{pr:pracovniListZS}{Pracovní list pro základní školy}
\pracovniList{pracovniListy/pracovniListTepelneMotoryZS}{Pracovní list pro základní školy}{pracovniListy/pracovniListTepelneMotoryZS.pdf}

\arbitraryref{pr:pracovniListZSreseni}{Řešení pracovního listu pro základní školy}
\pracovniList{pracovniListy/pracovniListTepelneMotoryZSreseni}{Řešení pracovního listu pro základní školy}{pracovniListy/pracovniListTepelneMotoryZSreseni.pdf}

\newpage
\arbitraryref{pr:pracovniListSSvarA}{Pracovní list pro střední školy (varianta A)}
\pracovniList{pracovniListy/pracovniListTepelneMotorySSvarA}{Pracovní list pro střední školy (varianta A)}{pracovniListy/pracovniListTepelneMotorySSvarA.pdf}

\arbitraryref{pr:pracovniListSSvarAreseni}{Řešení pracovního listu pro střední školy (varianta A)}
\pracovniList{pracovniListy/pracovniListTepelneMotorySSvarAreseni}{Řešení pracovního listu pro střední školy (varianta A)}{pracovniListy/pracovniListTepelneMotorySSvarAreseni.pdf}

\newpage
\arbitraryref{pr:pracovniListSSvarB}{Pracovní list pro střední školy (varianta B)}
\pracovniList{pracovniListy/pracovniListTepelneMotorySSvarB}{Pracovní list pro střední školy (varianta B)}{pracovniListy/pracovniListTepelneMotorySSvarB.pdf}

\arbitraryref{pr:pracovniListSSvarBreseni}{Řešení pracovního listu pro střední školy (varianta B)}
\pracovniList{pracovniListy/pracovniListTepelneMotorySSvarBreseni}{Řešení pracovního listu pro střední školy (varianta B)}{pracovniListy/pracovniListTepelneMotorySSvarBreseni.pdf}

%Vyplněné pracovní listy
\newpage
\arbitraryref{pr:vyplnenePracovniListy}{Vyplněné pracovní listy}
\textbf{Úspěšné řešení pracovního listu} {Dostupné ke stažení }\textattachfile[color=0 0 0]{assets/images/ScanPracovniListy.pdf}{\textit{zde}}\odst
\begin{minipage}{0.5\textwidth}
    \centering
    \fbox{
    \includegraphics[scale = 0.3, page=2, clip]{assets/images/ScanPracovniListy.pdf}
    }
\end{minipage}%
\begin{minipage}{0.5\textwidth}
    \centering
    \fbox{
    \includegraphics[scale = 0.3, page=1, clip]{assets/images/ScanPracovniListy.pdf}
    }
\end{minipage}\odst

\textbf{Neúspěšné řešení pracovního listu} {Dostupné ke stažení }\textattachfile[color=0 0 0]{assets/images/ScanPracovniListy.pdf}{\textit{zde}}\odst
\begin{minipage}{0.5\textwidth}
    \centering
    \fbox{
    \includegraphics[scale = 0.3, page=1, clip]{assets/images/ScanPracovniListy.pdf}
    }
\end{minipage}%
\begin{minipage}{0.5\textwidth}
    \centering
    \fbox{
    \includegraphics[scale = 0.3, page=2, clip]{assets/images/ScanPracovniListy.pdf}
    }
\end{minipage}\odst

\end{document}