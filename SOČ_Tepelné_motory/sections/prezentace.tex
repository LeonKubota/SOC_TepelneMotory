\section{Prezentace}\label{sc:prezentace}
{Prezentace jsem tvořil v softwaru \code{Google Slides}, jsou dostupné v přílohách. Obecná prezentace ,,Tepelné motory'' zabere přibližně jednu vyučovací hodinu, a to i s příklady. Ostatní prezentace trvají přibližně 20 minut.}\odst
{Cílem prezentací bylo účinně popsat látku tepelných motorů, toho jsem docílil stručnými body. Historie v prezentacích není příliš zastoupena, zaměřil jsem se na fungování a využití daných motorů. K vysvětlení fungování velmi pomohly animace.}
\subsection{Obsah prezentací}
{Prezentace obecně seznamují s tepelnými motory (nebo konkrétními druhy). Nejprve přiblíží fungování, které učitel dále vysvětlí na animacích. Následují příklady využití. Ke každé skupině motorů jsou vysvětleny rovnice.}\odst
{Na závěr jsou příklady. Učitel vždy vyvolá dva žáky, kteří proti sobě budou soupeřit; žák, který jako první správně spočítá příklad, získá dobrou známku či body. Nejprve se odkryjí dvě zadání, pro každého žáka jedno (viz \obrref{obr:prikladyZadani}), až ho žáci spočítají, učitel ukáže postup a výsledky na snímku \obrref{obr:prikladyReseni}.}
\begin{figure}[H]
    \begingroup
    \makeatletter
    \renewcommand\thesubfigure{\thefigure~--~\@nameuse{subfiglabel@\alph{subfigure}}}
    \newcommand{\subfiglabel@a}{vlevo}
    \newcommand{\subfiglabel@b}{vpravo}
    \captionsetup[subfigure]{labelformat=simple, labelsep=colon}
    \renewcommand\p@subfigure{}
    \makeatother
    \begin{subfigure}{0.45\textwidth}
        \centering
        \setlength{\fboxsep}{0pt}
        \fbox{\includegraphics[scale=0.2]{assets/images/PrikladZadani.png}}
        \caption{Dvě zadání příkladu, žáci soutěží v jeho spočítání \jaObr}
        \label{obr:prikladyZadani}
    \end{subfigure}\hfill
    \begin{subfigure}{0.45\textwidth}
        \centering
        \setlength{\fboxsep}{0pt}
        \fbox{\includegraphics[scale=0.2]{assets/images/PrikladHotovo.png}}
        \caption{Ukázka postupu a řešení příkladu \jaObr}
        \label{obr:prikladyReseni}
    \end{subfigure}
    \endgroup
\end{figure}
\newpage
\subsection{Vizuální stránka prezentací}
{Pro vizuální stránku prezentací jsem zvolil tmavě červené detaily doplněné ilustračními obrázky vozidel, ve kterých se daný typ motoru používá. Na \obrref{obr:tatra813} jsou popsány jednotlivé části tvorby takového obrázku.}
\begin{figure}[H]
    \begin{subfigure}{0.5\textwidth}
        \centering
        \setlength{\fboxsep}{0pt}
        \fbox{\includegraphics[scale=.11]{assets/images/tatra813_1.png}}
        \caption{První hrubý model}
    \end{subfigure}
    \begin{subfigure}{0.5\linewidth}
        \centering
        \setlength{\fboxsep}{0pt}
        \fbox{\includegraphics[scale=.11]{assets/images/tatra813_5.png}}
        \caption{Hotový model}
    \end{subfigure}
    \begin{subfigure}{0.5\linewidth}
        \centering
        \setlength{\fboxsep}{0pt}
        \fbox{\includegraphics[scale=.11]{assets/images/tatra813_6.png}}
        \caption{Model s texturamy}
    \end{subfigure}
    \begin{subfigure}{0.5\linewidth}
        \centering
        \setlength{\fboxsep}{0pt}
        \fbox{\includegraphics[scale=.22]{assets/images/tepelneMotoryObecneTatra.png}}
        \caption{Hotový snímek z prezentace}
    \end{subfigure}
    \caption{Proces tvorby ilustračního obrázku nákladního automobilu Tatra 813 \jaObr}
    \label{obr:tatra813}
\end{figure}
{Dále jsem vytvořil obrázky, na kterých jde názorně vidět princip fungování daného motoru: pro parní motory jsem vytvořil obrázek tlakového hrnce, pro spalovací motory píst tlačen výbuchem a pro reaktivní motory vyfukující se balónek.}