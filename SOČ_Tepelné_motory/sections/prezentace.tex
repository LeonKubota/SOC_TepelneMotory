\section{Prezentace}\label{sc:prezentace}
{Prezentace jsem tvořil v softwaru \code{Google Slides}, jsou dostupné v přílohách (\ref{pr:prezentaceTepelneMotory}, \ref{pr:prezentaceParniMotory}, \ref{pr:prezentaceSpalovaciMotory} a \ref{pr:prezentaceReaktivniMotory}). Prezentace ,,Tepelné motor'' zabere přibližně jednu vyučovací hodinu, a to i s příklady. Ostatní prezentace trvají přibližně 20 minut.}\odst
{Cílem prezentací bylo účinně popsat látku tepelných motorů, toho jsem docílil stručnými body. Historie v prezentacích není příliš zastoupena, zaměřil jsem se na fungování a využití daných motorů. K vysvětlení fungování velmi pomohly animace.}

\subsection{Vizuální stránka prezentací}
{Pro vizuální stránku prezentací jsem zvolil tmavě červené detaily doplněné ilustračními obrázky vozidel, ve kterých se daný typ motoru používá. Na \obrref{obr:tatra813} jsou popsány jednotlivé části tvorby takového obrázku.}

\begin{figure}[H]
    \begin{subfigure}{0.5\textwidth}
        \centering
        \includegraphics[scale=.11]{assets/images/tatra813_1.png}\\
        \caption{První hrubý model}
    \end{subfigure}
    \begin{subfigure}{0.5\linewidth}
        \centering
        \includegraphics[scale=.11]{assets/images/tatra813_5.png}\\
        \caption{Hotový model}
    \end{subfigure}
    \begin{subfigure}{0.5\linewidth}
        \centering
        \includegraphics[scale=.11]{assets/images/tatra813_6.png}\\
        \caption{Model s texturamy}
    \end{subfigure}
    \begin{subfigure}{0.5\linewidth}
        \centering
        \includegraphics[scale=.22]{assets/images/tepelneMotoryObecneTatra.png}\\
        \caption{Hotový snímek z prezentace}
    \end{subfigure}
    \caption{Proces tvorby ilustračního obrázku nákladního automobilu Tatra 813 \jaObr}
    \label{obr:tatra813}
\end{figure}

{Dále jsem vytvořil obrázky, na kterých jde názorně vidět princip fungování daného motoru: pro parní motory jsem vytvořil obrázek tlakového hrnce, pro spalovací motory píst tlačen výbuchem a pro reaktivní motory vyfukující se balónek.}