%TC:ignore

\pagestyle{empty}
{\centering
\fontsize{18}{0}\textbf{STŘEDOŠKOLSKÁ ODBORNÁ ČINNOST}\\
\vspace{0.25cm}
\fontsize{14}{0}\textbf{Obor č. 12: Tvorba učebních pomůcek, didaktická technologie}\\
\vspace{8cm}
\fontsize{20}{0}\textbf{3D modely a výukové pomůcky na téma tepelné motory}\\
\vfill{}

\begin{tabularx}{1\textwidth} { 
    >{\raggedright\arraybackslash}X 
    >{\centering\arraybackslash}X 
    >{\raggedleft\arraybackslash}X}
    \fontsize{16}{0}\textbf{Hlavní město Praha} & 
    \fontsize{16}{0}\textbf{Leon Kubota} & 
    \fontsize{16}{0}\textbf{Praha, 2025}
\end{tabularx}}

\newpage
{\centering
\fontsize{18}{0}\textbf{STŘEDOŠKOLSKÁ ODBORNÁ ČINNOST}\\
\vspace{0.25cm}
\fontsize{14}{0}\textbf{Obor č. 12: Tvorba učebních pomůcek, didaktická technologie}\\
\vspace{7.5cm}
\fontsize{20}{0}\textbf{3D modely a výukové pomůcky na téma tepelné motory}\\
\vspace{1.5cm}
\fontsize{20}{0}\textbf{3D Models and Educational Aids on the Topic of Heat Engines}\\
\vspace{0.25cm}
\vfill{}
\begin{minipage}{0.8\textwidth}
      \begin{flushleft}
        \fontsize{16}{20}\selectfont{\textbf{Autor:} Leon Kubota}\\
        \fontsize{16}{20}\selectfont{\textbf{Škola:} Gymnázium, Praha 6, Arabská 14}\\
        \fontsize{16}{20}\selectfont{\textbf{Kraj:} Hlavní město Praha}\\
        \fontsize{16}{20}\selectfont{\textbf{Konzultant:} Mgr. Jana Urzová, Ph.D.}\\
        \fontsize{12}{20}\selectfont{Praha, 2025}\\
    \end{flushleft}
\end{minipage}%
\begin{minipage}{0.2\textwidth}
    \begin{figure}[H]
        \includegraphics[scale = 0.4]{assets/images/logo.png}
    \end{figure}
\end{minipage}
}

\newpage

\section*{Prohlášení}
{Prohlašuji, že jsem svou práci SOČ vypracoval samostatně a použil jsem pouze prameny a literaturu uvedené v seznamu bibliografických záznamů.}\odst
{Prohlašuji, že tištěná verze a elektronická verze soutěžní práce SOČ jsou shodné.}\odst
{Nemám závažný důvod proti zpřístupňování této práce v souladu se zákonem č. 121/2000 Sb., o právu autorském, o právech souvisejících s právem autorským a o změně některých zákonů (autorský zákon) ve znění pozdějších předpisů.}\par
\vspace{2cm}
\noindent{V}
\noindent\rule{3cm}{0.4pt}
\noindent{dne}
\noindent\rule{3cm}{0.4pt}
\hfill
\noindent{Podpis:}
\noindent\rule{5cm}{0.4pt}

\newpage

\section*{Poděkování}
{Rád bych poděkoval Mgr. Janě Urzové, Ph.D. za vedení práce, dobré rady a možnost praktického vyzkoušení mých materiálů. Dále bych chtěl poděkovat  Mgr. Janu Tuzarovi za vedení původní ročníkové práce. Za zapůjčení mikrofonu jsem vděčný Gymnáziu Arabská. Děkuji také mé rodině a přátelům za poskytnutí zpětné vazby k materiálům, které jsem jim příliš často ukazoval.}

\newpage

\section*{Anotace}
{Tato práce se zabývá tvorbou učebních pomůcek pro výuku tepelných motorů. Mezi tyto pomůcky patří zejména animované video obsahující 3D animace parních, spalovacích a reaktivních motorů. Modely těchto motorů jsou založeny na skutečných konstrukcích. V práci jsou také obsaženy prezentace určené pro základní školy a obsáhlejší prezentace pro 2. ročník středních škol. Znalosti žáků mohou učitelé vyzkoušet pomocí obsažených pracovních listů. Materiály byly vyzkoušeny ve výuce studentů druhého ročníku školy Gymnázium, Praha 6, Arabská 14.}

\section*{Klíčová slova}
{Tepelné motory; výukové video; 3D modely; 3D animace; učební pomůcky}

\section*{Abstract}
{This thesis is about the creation of educational tools for teaching about heat engines. Primarly, these tools include an animated video containing 3D animations of steam, combustion and reaction engines. The 3D models of these engines are based on real-life designs. The work also contains presentations made for primary schools and more advanced presentations for the second year of high schools. Teachers may assess the knowledge of students with the use of the included worksheets. The materials were piloted in a class of the second year students at Gymnázium,Prague 6, Arabská 14.}

\section*{Keywords}
{Heat engines; educational video; 3D models; 3D animation; educational aids}

\newpage

\tableofcontents

%TC:endignore

\newpage
\pagestyle{plain}

\section{Úvod}
{Tepelné motory jsou nedílnou součástí moderního světa: od sekaček a automobilů po obrovské jaderné elektrárny a vesmírné rakety. Narazíme na ně na každém rohu, prakticky ještě mnohem častěji. Proto výuku o nich považuji za důležitou. Touto prací navazuji na můj ročníkový projekt ,,Tepelné motory”. Po jeho dokončení jsem se rozhodl pokračovat do soutěže SOČ a vytvořit ještě lepší učební pomůcky.}\odst
{V první kapitole je popsán ideální plyn, ten je pro tepelné motory velmi důležitý. Dále jsou v ní popsány základní informace o tepelných motorech. V následujících kapitolách jsou stručně popsány tepelné motory a to parní, spalovací a reaktivní. Krátce se zabývají historií, fungováním a využitím dané skupiny motorů. Na konci každé z těchto kapitol jsou uvedeny nejdůležitější rovnice týkající se těchto motorů. Tyto kapitoly slouží jako podklad pro praktickou část.}\odst
{Většina existujících výukových materiálů o tepelných motorech je určena pro žáky, které toto téma již zajímá. Většinou se jedná o videa (např. \href{https://www.youtube.com/watch?v=k9DhdvbmRiw}{1}, \href{https://www.youtube.com/watch?v=dR1pyp3q9Ko}{2}, \href{https://www.youtube.com/watch?v=eP8nWqcWmAc}{3} a \href{https://www.youtube.com/watch?v=Fpbg1jUh36M}{4}), která popisují fungování konkrétního motoru. Prezentace na toto téma (např. \href{https://zs-nucice.cz/UserFiles/File/eu_new_361-400/VY_32_INOVACE_374.pdf}{5}, \href{https://view.officeapps.live.com/op/view.aspx?src=http://dumy.cz/nahled/73077}{6} nebo \href{https://slideplayer.cz/slide/12677857/}{7}) jsou často zastaralé. Nalézt kvalitní pracovní listy (např. \href{https://view.officeapps.live.com/op/view.aspx?src=http://dumy.cz/nahled/117673}{8} nebo \href{https://www.soshlinky.cz/documents/uploads/71\%20Motory.xlsx}{9}) na téma tepelných motorů je obtížné. 3D animace (např. \href{https://www.youtube.com/watch?v=kWRRHRWuduk}{10}, \href{https://www.youtube.com/watch?v=ZQvfHyfgBtA&t=26s}{11}, \href{https://www.youtube.com/watch?v=MUxP3PCDRTE}{12} a \href{https://www.youtube.com/watch?v=Iiu3UyxLEHk}{13}) samozřejmě existují; ačkoliv některé z nich jsou velice kvalitní, získat kompletní sbírku animací v jednotném stylu se jeví jako nemožné.}\odst
{Mým cílem bylo vytvořit výukové materály, které jsou názorné, přehledné a snadno dostupné. Vytvořil jsem devět cyklických 3D animací tepelných motorů, které jsem využil ve čtyř prezentacích na toto téma. Pro vyzkoušení znalostí žáků jsem vytvořil dva pracovní listy, jeden pro žáky základních a druhý pro žáky středních škol. Poslední výukovou pomůckou je osmiminutové animované video, které zajímavou formou popíše fungování jednotlivých tepelných motorů. Na závěr jsem své pomůcky vyzkoušel ve výuce žáků druhého ročníku školy Gymnázium, Praha 6, Arabská 14.}\odst
{Tepelné motory jsou zajímavé, ale také velice rozsáhlé téma, které nelze kompletně vysvětlit v několika hodinách fyziky. Mým cílem bylo vytvořit pomůcky, které žákům přiblíží základy těchto strojů. Zároveň bych v žácích rád povzbudil vlastní ambici pro další samostatné studium tepelných motorů či jiných fyzikálních a inženýrských disciplín.}