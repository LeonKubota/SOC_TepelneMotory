%TC:ignore

\pagestyle{empty}
{\centering
\fontsize{18}{0}\textbf{STŘEDOŠKOLSKÁ ODBORNÁ ČINNOST}\\
\vspace{0.25cm}
\fontsize{14}{0}\textbf{Obor č. 12: Tvorba učebních pomůcek, didaktická technologie}\\
\vspace{8cm}
\fontsize{20}{0}\textbf{3D modely a výukové pomůcky na téma tepelné motory}\\
\vfill{}

\begin{tabularx}{1\textwidth} { 
    >{\raggedright\arraybackslash}X 
    >{\centering\arraybackslash}X 
    >{\raggedleft\arraybackslash}X}
    \fontsize{16}{0}\textbf{Hlavní město Praha} & 
    \fontsize{16}{0}\textbf{Leon Kubota} & 
    \fontsize{16}{0}\textbf{Praha, 2025}
\end{tabularx}}

\newpage
{\centering
\fontsize{18}{0}\textbf{STŘEDOŠKOLSKÁ ODBORNÁ ČINNOST}\\
\vspace{0.25cm}
\fontsize{14}{0}\textbf{Obor č. 12: Tvorba učebních pomůcek, didaktická technologie}\\
\vspace{7.5cm}
\fontsize{20}{0}\textbf{3D modely a výukové pomůcky na téma tepelné motory}\\
\vspace{1.5cm}
\fontsize{20}{0}\textbf{3D Models and Educational Aids on the Topic of Heat Engines}\\
\vspace{0.25cm}
\vfill{}
\begin{minipage}{0.8\textwidth}
      \begin{flushleft}
        \fontsize{16}{20}\selectfont{\textbf{Autor:} Leon Kubota}\\
        \fontsize{16}{20}\selectfont{\textbf{Škola:} Gymnázium, Praha 6, Arabská 14}\\
        \fontsize{16}{20}\selectfont{\textbf{Kraj:} Hlavní město Praha}\\
        \fontsize{16}{20}\selectfont{\textbf{Konzultant:} Mgr. Jana Urzová, Ph.D.}\\
        \fontsize{12}{20}\selectfont{Praha, 2025}\\
    \end{flushleft}
\end{minipage}%
\begin{minipage}{0.2\textwidth}
    \begin{figure}[H]
        \includegraphics[scale = 0.4]{assets/images/logo.png}
    \end{figure}
\end{minipage}
}

\newpage

\section*{Prohlášení}
{Prohlašuji, že jsem svou práci SOČ vypracoval samostatně a použil jsem pouze prameny a literaturu uvedené v seznamu bibliografických záznamů.}\odst
{Prohlašuji, že tištěná verze a elektronická verze soutěžní práce SOČ jsou shodné.}\odst
{Nemám závažný důvod proti zpřístupňování této práce v souladu se zákonem č. 121/2000 Sb., o právu autorském, o právech souvisejících s právem autorským a o změně některých zákonů (autorský zákon) ve znění pozdějších předpisů.}\par
\vspace{2cm}
\noindent{V}
\noindent\rule{3cm}{0.4pt}
\noindent{dne}
\noindent\rule{3cm}{0.4pt}
\hfill
\noindent{Podpis:}
\noindent\rule{5cm}{0.4pt}

\newpage

\section*{Poděkování}
{Rád bych poděkoval Mgr. Janě Urzové za vedení práce, dobré rady a možnost praktického vyzkoušení mých materiálů. Dále bych chtěl poděkovat  Mgr. Janu Tuzarovi za vedení původní ročníkové práce. Za zapůjčení mikrofonu jsem vděčný Gymnáziu Arabská. Děkuji také mé rodině a přátelům za poskytnutí zpětné vazby k materiálům, které jsem jim příliš často ukazoval.}

\newpage

\section*{Anotace}
{Tato práce se zabývá tvorbou učebních pomůcek pro výuku tepelných motorů. Mezi tyto pomůcky patří zejména animované video obsahující 3D animace parních, spalovacích a reaktivních motorů. Modely těchto motorů jsou založeny na skutečných konstrukcích. V práci jsou také obsaženy prezentace určené pro základní školy a obsáhlejší prezentace pro 2. ročník středních škol. Znalosti žáků mohou učitelé vyzkoušet pomocí obsažených pracovních listů. Materiály byli vyzkoušeny na studentech druhého ročníku školy Gymnázium Arabská.}

\section*{Klíčová slova}
{Tepelné motory; výukové video; 3D modely; 3D animace; učební pomůcky}

\section*{Abstract}
{This thesis is about the creation of educational tools for teaching about heat engines. Primarly, these tools include an animated video containing 3D animations of steam, combustion and reaction engines. The 3D models of these engines are based on real-life designs. The work also contains presentations made for primary schools and more advanced presentations for the second year of high schools. Teachers may assess the knowledge of students with the use of the included worksheets. The materials were piloted in a class of the second year students at Gymnázium Arabská in Prague.}

\section*{Keywords}
{Heat engines; educational video; 3D models; 3D animation; educational aids}

\newpage

\tableofcontents

%TC:endignore

\newpage
\pagestyle{plain}

\section{Úvod}
{Tepelné motory jsou nedílnou součástí moderního světa: od sekaček a automobilů po obrovské jaderné elektrárny a vesmírné rakety. Narazíme na ně na každém rohu, prakticky ještě mnohem častěji. Proto výuku o nich považuji za důležitou. Touto prací navazuji na můj ročníkový projekt ,,Tepelné motory”. Po jeho dokončení jsem se rozhodl pokračovat do soutěže SOČ a vytvořit ještě lepší učební pomůcky.}\odst
{V první kapitole je popsán ideální plyn, ten je pro tepelné motory velmi důležitý. Dále jsou v ní popsány základní informace o tepelných motorech. V následujících kapitolách jsou stručně popsány tepelné motory a to parní, spalovací a reaktivní. Krátce se zabývají historií, fungováním a využitím dané skupiny motorů. Na konci každé z těchto kapitol jsou uvedeny nejdůležitější rovnice týkající se těchto motorů. Tyto kapitoly slouží jako podklad pro praktickou část.}\odst
{Pomocí softwaru \code{Blender}, který používám již pátým rokem, jsem vytvořil 3D modely a ani\-ma\-ce různých tepelných motorů. Ty jsem využil v několikaminutovém animovaném videu, které zábavnou formou vykládá látku tepelných motorů. Dále práce obsahuje prezentace pro žáky základních i středních škol. Pro vyzkoušení znalostí žáků jsem vytvořil několik pracovních listů. Tyto pomůcky jsem vyzkoušel na studentech druhého ročníku školy Gymnázium Arabská.}\odst
{Tepelné motory jsou zajímavé ale také velice obsáhlé téma, které není možné vysvětlit v několika hodinách fyziky. Mým cílem bylo vytvořit pomůcky, které žákovy přiblíží základy těchto strojů a současně zaujmou natolik, že se někteří z nich začnou zajímat o tepelné motory ve svém volném čase.}

\section*{\textcolor{red}{Konkurence}}
\textcolor{red}{Dočasné zdroje co zakomponuju do úvodu.}
\subsubsection*{\textcolor{red}{Animace}}
\href{https://en.wikipedia.org/wiki/Steam_engine#/media/File:Steam_engine_in_action.gif}{\textcolor{red}{Parní stroj;}}
\href{https://www.tlv.com/sites/default/files/tlv_assets/ja/steam_story/images/0611syuruitoyouto/pafs-generator-turbine_EN.gif}{\textcolor{red}{Parní turbína;}}
\href{URL}{text}
\href{https://en.wikipedia.org/wiki/Internal_combustion_engine#/media/File:4StrokeEngine_Ortho_3D_Small.gif}{\textcolor{red}{Čtyřdobý zážehový motor;}}
\subsubsection*{\textcolor{red}{Videa}}
\href{https://www.youtube.com/watch?v=dR1pyp3q9Ko}{\textcolor{red}{Fyzika na dálku - 8. ročník - Tepelné motory}}\\
\href{https://www.youtube.com/watch?v=k9DhdvbmRiw}{\textcolor{red}{Tepelné motory - fyzika 8 ZŠ}}
\subsubsection*{\textcolor{red}{Prezentace}}
\href{https://zs-nucice.cz/UserFiles/File/eu_new_361-400/VY_32_INOVACE_374.pdf}{\textcolor{red}{Tepelné motory -- úvod -- prezentace}}\\
\href{https://view.officeapps.live.com/op/view.aspx?src=http://dumy.cz/nahled/73077}{\textcolor{red}{Tepelné motory - prezentace}}
\subsubsection*{\textcolor{red}{Pracovní listy}}
\href{https://www.google.com/url?sa=i&url=https\%3A\%2F\%2Fslideplayer.cz\%2Fslide\%2F12677857\%2F&psig=AOvVaw0fGM8NbBvM9WlZNO6TqO0o&ust=1741690312947000&source=images&cd=vfe&opi=89978449&ved=0CAMQjB1qFwoTCMCD2Zes_4sDFQAAAAAdAAAAABBP}{\textcolor{red}{Tepelné motory - opakování}}\\
\href{https://view.officeapps.live.com/op/view.aspx?src=http://dumy.cz/nahled/117673}{\textcolor{red}{Spalovací motory - pracovní list}}\\
\href{https://www.soshlinky.cz/documents/uploads/71\%20Motory.xlsx}{\textcolor{red}{Motory - pracovní list}}