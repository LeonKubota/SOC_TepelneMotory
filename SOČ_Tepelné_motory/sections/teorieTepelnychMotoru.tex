\section{Teorie tepelných motorů}
{Tepelný motor je stroj, který přeměňuje tepelnou energii na mechanickou práci. Tuto energii mu dodáváme plynnou pracovní látkou.}

\subsection{Ideální plyn}
{Pro porozumění tepelným motorům je nezbytné znát ideální plyn. Zákon ideálních plynů popisuje vztahy mezi tlakem, teplotou a objemem. Ve skutečnosti je tvořen třemi zákony: izochorickým Charlesovým zákonem \rvref{rv:Charles}, izotermickým Boylovým zákonem \rvref{rv:Boyle} a izobarickým Gay-Lussacovým zákonem \rvref{rv:GayLussac}.}
\cite{SP:IdealGasBehaviour}

\vspace{0.8cm}
\begin{subequations}
    \begin{minipage}{0.3\textwidth}
        \begin{equation}\label{rv:Charles}
            \frac{p}{T}=konst.
        \end{equation}
    \end{minipage}
    \hfill
    \begin{minipage}{0.3\textwidth}
        \begin{equation}\label{rv:Boyle}
            pV=konst.
        \end{equation}
    \end{minipage}
    \hfill
    \begin{minipage}{0.3\textwidth}
        \begin{equation}\label{rv:GayLussac}
            \frac{V}{T}=konst.
        \end{equation}
    \end{minipage}
    \vspace{0.8cm}

{\(p\) je tlak [Pa]}\\
{\(V\) je objem [\mmm]}\\
{\(T\) je teplota [K]}\\
\podst

{Spojením těchto tří zákonů získáme zákon ideálního plynu, ten lze zapsat stavovou rovnicí ideálního plynu \rvref{rv:stavovaRovnice}.}
\cite{SP:IdealGasBehaviour}

    \begin{equation}\label{rv:stavovaRovnice}
        pV=nRT
    \end{equation}
    
\end{subequations}
{\(n\) je látkové množství [mol]}\\
{\(R\) je univerzální plynová konstanta (8,31 \jmolk)[\jmolk]}\\
\podst

{Ideální plyn se však velmi liší od skutečného plynu, velikost molekul považujeme za zanedbatelnou a stejnou u všech molekul, předpokládáme náhodný pohyb dle Newtonových zákonů a zanedbáváme ztráty energie při kolizích. Ze stavové rovnice však plyne velmi důležitý poznatek: při zahřátí se plyn rozpíná. Toho využívají tepelné motory pro konání práce.}
\cite{SP:IdealGasBehaviour}

\subsection{Účinnost a práce}\label{sc:SkutecnaUcinnost}
{Účinnost tepelných motorů je omezena druhým termodynamickým zákonem. Účinnost tepelného motoru je podíl užitečné mechanické práce a dodaného tepla, viz rovnice \rvref{rv:ucinnost1}.}
\cite{NCEPU:ThermalEfficiencyForHeatEngines}

\begin{subequations}

    \begin{equation}\label{rv:ucinnost1}
        \eta=\frac{W}{Q_d}
    \end{equation}

{\(\eta\) je účinnost [-]}\\
{\(W\) je práce [J]}\\
{\(Q_d\) je dodané teplo [J]}\\
\podst

{Práci můžeme vypočítat pomocí rovnice \rvref{rv:prace}.}
\cite{NCEPU:ThermalEfficiencyForHeatEngines}

    \begin{equation}\label{rv:prace}
        W=Q_d-Q_o
    \end{equation}

{\(Q_o\) je odevzdané teplo [J]}\\
\podst

\end{subequations}