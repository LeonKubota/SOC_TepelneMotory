\section{Použití pomůcek v praxi}\label{sc:pouzitiVPraxi}
{Pracovní list pro základní školy a prezentaci ,,Tepelné motory'' jsem vyzkoušel na žácích druhého ročníku školy Gymnázum, Praha 6, Arabská 14. Učil jsem dvě hodiny, nejprve jsem učil třídu 2.B. Přednesl jsem prezentaci a poté rozdal pracovní papíry jako domácí úkol. Ve druhé hodině jsem učil třídu 2.A, rozdal jsem pracovní listy před hodinou, aby je žáci vyplňovali současně s prezentací a více se soustředili. Termín odevzdání byl o několik hodin později.}\odst
{Mé materiály samozřejmě nebyly perfektní. Prezentace potřebovala drobné úpravy, její následnost byla místy neohrabaná a nelogická, tyto problémy jsem opravil upravením pořadí snímků a změnou či přidáním obrázků.}\odst
{Žáci měli problémy s identifikací svíčky, přidal jsem tedy obrázek svíčky k vysvětlení zážehových motorů, aby na něj učitel mohl upozornit. Dále bylo složité vysvětlit počet pracovních dob za otáčku hřídele. Přidal jsem popisný text ke vzorci výkonu spalovacích motorů. Upravené snímky jsou na \obrref{obr:svickaSlide} a \obrref{obr:vzorceSlide}.}
\begin{figure}[H]
    \begingroup
    \makeatletter
    \renewcommand\thesubfigure{\thefigure~--~\@nameuse{subfiglabel@\alph{subfigure}}}
    \newcommand{\subfiglabel@a}{vlevo}
    \newcommand{\subfiglabel@b}{vpravo}
    \captionsetup[subfigure]{labelformat=simple, labelsep=colon}
    \renewcommand\p@subfigure{}
    \makeatother
    \begin{subfigure}{0.47\textwidth}
        \centering
        \setlength{\fboxsep}{0pt}
        \fbox{\includegraphics[scale=0.2]{assets/images/SvickaPrezentace.png}}
        \caption{Obrázek svíčky u vysvětlení\\čtyřdobých zážehových motorů \jaObr}
        \label{obr:svickaSlide}
    \end{subfigure}\hfill
    \begin{subfigure}{0.47\textwidth}
        \centering
        \setlength{\fboxsep}{0pt}
        \fbox{\includegraphics[scale=0.2]{assets/images/PracovniDoby1.png}}
        \caption{Přidání vysvětlení počtu dob za otáčku \jaObr}
        \label{obr:vzorceSlide}
    \end{subfigure}
    \endgroup
\end{figure}
{Také jsem vytvořil nový obrázek pro znázornění akce a reakce, jelikož starý obrázek židle (viz \obrref{obr:zidlePrezentace}) tlačené hasicími přístroji byl nevyhovující. Nový obrázek \obrref{obr:balonekPrezentace} odlétajícího nafukujícího balónku je mnohem názornější a pochopitelnější.}
\begin{figure}[H]
    \begingroup
    \makeatletter
    \renewcommand\thesubfigure{\thefigure~--~\@nameuse{subfiglabel@\alph{subfigure}}}
    \newcommand{\subfiglabel@a}{vlevo}
    \newcommand{\subfiglabel@b}{vpravo}
    \captionsetup[subfigure]{labelformat=simple, labelsep=colon}
    \renewcommand\p@subfigure{}
    \makeatother
    \begin{subfigure}{0.47\textwidth}
        \centering
        \setlength{\fboxsep}{0pt}
        \fbox{\includegraphics[scale=0.2]{assets/images/zidlePrezentace.png}}
        \caption{Obrázek židle s hasicími přístroji u vysvětlení zákona akce a reakce \jaObr}
        \label{obr:zidlePrezentace}
    \end{subfigure}\hfill
    \begin{subfigure}{0.47\textwidth}
        \centering
        \setlength{\fboxsep}{0pt}
        \fbox{\includegraphics[scale=0.2]{assets/images/balonekPrezentace.png}}
        \caption{Obrázek nafukovacího balónku u vysvětlení zákona akce a reakce \jaObr}
        \label{obr:balonekPrezentace}
    \end{subfigure}
    \endgroup
\end{figure}
{Dále jsem musel opravit chyby v příkladech. Jelikož jsem příklady mnohokrát předělával, výsledky nebyly vždy správné. V hotové verzi je toto samozřejmě opraveno. Některým žákům dělaly příklady potíže, především kvůli jejich problémům s převody jednotek. Příklady však s drobnou asistencí žáci vyřešili správně.}
\subsection{Úprava pracovních listů}
{Jak již bylo zmíněno, pracovní listy jsem zadal jako dobrovolný domácí úkol, termín odevzdání byl stanoven na tentýž den. Žáci pracovali ve dvojicích. Výsledky jsou shrnuty v \tabref{tab:vysledkyPracovnichListu}.}
\rowcolors{2}{gray!20}{white}
\renewcommand{\arraystretch}{1.1}
\begin{table}[H]
    \centering
    \begin{tabular}{|p{3.3cm}|p{2cm}|p{2cm}|p{2cm}|}
        \hline
        \rowcolor{black!60}
        \textcolor{white}{Cvičení} & \textcolor{white}{Celkem} &\textcolor{white}{Třída 2.B} & \textcolor{white}{Třída 2.A} \\
        \hline
        Celkem & \hfill 76,5 \% & \hfill 73,1 \% & \hfill 80,0 \% \\
        \hline
        Doplnění \hfill ~--~ cv.1 & \hfill 77,3 \% & \hfill 76,9 \% & \hfill 77,7 \% \\
        \hline
        Doplnění \hfill ~--~ cv.2 & \hfill 97,3 \% & \hfill 97,4 \% & \hfill 97,2 \% \\
        \hline
        Počítání \hfill ~--~ cv.3 & \hfill 97,3 \% & \hfill  94,9 \% &\hfill  100 \% \\
        \hline
        Doplnění \hfill ~--~ cv.4 & \hfill 84,0 \% & \hfill  69,2 \% &\hfill  100 \% \\
        \hline
        Doplnění \hfill ~--~ cv.5 & \hfill 68,0 \% & \hfill 46,2 \% & \hfill 91,7 \% \\
        \hline
        Zakreslení \hfill ~--~ cv.6 & \hfill 72,0 \% & \hfill 76,9 \% & \hfill 66,7 \% \\
        \hline
        Přiřazení \hfill ~--~ cv.7 & \hfill 62,0 \% & \hfill 38,4 \% & \hfill 87,5 \% \\
        \hline
        Doplnění \hfill ~--~ cv.8 & \hfill 58,7 \% & \hfill 64,1 \% & \hfill 52,8 \% \\
        \hline
        Příklad \hfill ~--~ cv.9 & \hfill 46,7 \% & \hfill 56,4 \% & \hfill 36,1 \% \\
        \hline
        Příklad \hfill ~--~ cv.10 & \hfill 85,3 \% & \hfill 84,6 \% & \hfill 86,1 \% \\
        \hline
    \end{tabular}
    \caption{Hodnocení pracovních listů \jaTab}
    \label{tab:vysledkyPracovnichListu}
\end{table}
{Zde přikládám dva vyplněné pracovní listy. \obrref{obr:prikladNeuspesny} patří mezi méně úspěšné, naopak \obrref{obr:prikladUspesny} je jedním z nejúspějšnějších.}
\begin{figure}[H]
    \begingroup
    \makeatletter
    \renewcommand\thesubfigure{\thefigure~--~\@nameuse{subfiglabel@\alph{subfigure}}}
    \newcommand{\subfiglabel@a}{nahoře}
    \newcommand{\subfiglabel@b}{dole}
    \captionsetup[subfigure]{labelformat=simple, labelsep=colon}
    \renewcommand\p@subfigure{}
    \makeatother
    \begin{subfigure}{\textwidth}
        \centering
        \fbox{\includegraphics[scale=0.15, page=3, clip]{assets/images/ScanPracovniListy.pdf}}\hspace{1cm}
        \fbox{\includegraphics[scale=0.15, page=4, clip]{assets/images/ScanPracovniListy.pdf}}
        \caption{Neúspěšné řešení pracovního listu \jaObr}
        \label{obr:prikladNeuspesny}
    \end{subfigure}\odst
    \begin{subfigure}{\textwidth}
        \centering
        \fbox{\includegraphics[scale=0.15, page=2, clip]{assets/images/ScanPracovniListy.pdf}}\hspace{1cm}
        \fbox{\includegraphics[scale=0.15, page=1, clip]{assets/images/ScanPracovniListy.pdf}}
        \caption{Úspěšné řešení pracovního listu \jaObr}
        \label{obr:prikladUspesny}
    \end{subfigure}
    \endgroup
\end{figure}
{Naskenované pracovní listky jsou dostupné v přílohách zde: \ref{pr:vyplnenePracovniListy}.}
\newpage
{Zkušenosti z výuky jsem využil ke zdokonalení mých materiálů. Snížil jsem počet bodů za jednoduchá cvičení 2 a 3. Ve cvičení 4 žáci doplňovali hodnoty do tabulky, někteří žáci (převážně ze třídy 2.B) toto cvičení přeskočili, pravděpodobně bylo příliš dlouhé. Ve finální verzi jsem ho zkrátil. Cvičení 5 a 6 navazují na cvičení 4, což je jasně řečeno v zadání. Žáci zadání často nečetli a tak jsem se rozhodl upravit rozvržení pracovního papíru abych zdůraznil jejich návaznost. Původní rozvržení je znázorněno na \obrref{obr:puvodniRozvrzeni}, upravené rozvržení na \obrref{obr:upraveneRozvrzeni}.}
\begin{figure}[H]
    \begingroup
    \makeatletter
    \renewcommand\thesubfigure{\thefigure~--~\@nameuse{subfiglabel@\alph{subfigure}}}
    \newcommand{\subfiglabel@a}{vlevo}
    \newcommand{\subfiglabel@b}{vpravo}
    \captionsetup[subfigure]{labelformat=simple, labelsep=colon}
    \renewcommand\p@subfigure{}
    \makeatother
    \begin{subfigure}{0.45\textwidth}
        \centering
        \begin{tikzpicture}[scale=0.23]
            \draw[thick] (0,0) rectangle (21, 29.7);
            \draw[thick](0,27.5) to (21,27.5) (0,22) to (21,22) (0,18) to (21,18) (11,18) to (11,0) (0,9) to (11,9) (11,14.5) to (21,14.5) (11,11) to (21,11);
            \node[right] at (0,28.6) {Jméno a hodnocení};
            \node[right] at (0,26.5) {Cvičení 1};
            \node[right] at (0,21) {Cvičení 2};
            \node[right] at (0,17) {Cvičení 3};
            \node[right] at (0,8) {Cvičení 4};
            \node[right] at (11,17) {Cvičení 5};
            \node[right] at (11,13.5) {Cvičení 6};
            \node[right] at (11,10) {Cvičení 7};
        \end{tikzpicture}
        \caption{Původní rozvržení první strany pracovního listu \jaDiag}
        \label{obr:puvodniRozvrzeni}
    \end{subfigure}\hfill
    \begin{subfigure}{0.45\textwidth}
        \centering
        \begin{tikzpicture}[scale=0.23]
            \draw[thick] (0,0) rectangle (21, 29.7);
            \draw[thick](0,27.5) to (21,27.5) (0,22) to (21,22) (0,18) to (21,18) (10.5,18) to (10.5,11) (0,11) to (10.5,11) (10.5,14.5) to (21,14.5) (10.5,11) to (21,11) (0, 10.75) to (21, 10.75) (11,10.75) to (11,0);
            \node[right] at (0,28.6) {Jméno a hodnocení};
            \node[right] at (0,26.5) {Cvičení 1};
            \node[right] at (0,21) {Cvičení 2};
            \node[right] at (0,17) {Cvičení 4};
            \node[right] at (0,9.5) {Cvičení 3};
            \node[right] at (11,17) {Cvičení 5};
            \node[right] at (11,13.5) {Cvičení 6};
            \node[right] at (11,9.5) {Cvičení 7};
        \end{tikzpicture}
        \caption{Upravené rozvržení první strany pracovního listu \jaDiag}
        \label{obr:upraveneRozvrzeni}
    \end{subfigure}
    \endgroup
\end{figure}
{Dále jsem upravil obrázky pro lepší tisk a změnil jsem obrázek ze cvičení 8, jelikož měli žáci potíže rozeznat některé malé součástky. Nakonec jsem vyměnil pořadí posledních dvou cvičení, aby byl jednodušší příklad první.}
\subsection{Vyhodnocení praktického použití pomůcek}
{Při výuce se ukázalo mnoho chyb prezentace, pracovních listů i mého samotného výstupu. Tyto zkušenosti jsem využil ke zdokonalení svých výukových pomůcek, a to nejen těch, které jsem přímo v hodinách využil.}\odst
{Prezentaci jsem upravil nejméně, přidal jsem několik obrázků a lehce upravil text. Pracovní listy pro základní školy jsem zdokonalil a zpřehlednil; principy, které jsem se naučil, jsem aplikoval i na pracovní listy pro střední školy. Nejvíce jsem změnil svůj výklad, ten byl ve druhé hodině výrazně lepší.}\odst
{Z těchto dvou hodin jsem se naučil mnoho, a doufám, že žáci také. V budoucnosti bych rád vyzkoušel další materiály na různorodějších třídách, například i na základní škole.}
\newpage
\subsection{Zpětná vazba}
{Přibližně týden po výuce jsem žákům rozdal papíry na zpětnou vazbu (viz \obrref{obr:papirZpetnaVazba}). Ptal jsem se na sedm uzavřených otázek, ve kterých žáci hodnotili jednotlivé části výuky. Shrnutí těchto hodnocení naleznete v \tabref{tab:zpetnaVazba}.}\\
\begin{minipage}{0.35\textwidth}
    \begin{figure}[H]
        \centering
        \setlength{\fboxsep}{0pt}
        \fbox{\includegraphics[scale=0.5]{assets/images/ZpetnaVazba.pdf}}
        \caption{Papír pro vyplnění zpětné vazby \jaObr}
        \label{obr:papirZpetnaVazba}
    \end{figure}
\end{minipage}\hfill
\begin{minipage}{0.6\textwidth}
\rowcolors{2}{gray!20}{white}
    \renewcommand{\arraystretch}{1.25}
    \begin{table}[H]
        \begin{tabular}{|p{2.5cm}|p{1.7cm}|p{1.7cm}|p{1.7cm}|}
            \hline
            \rowcolor{black!60}
            \textcolor{white}{Název části} & \textcolor{white}{Celkem} & \textcolor{white}{Třída 2.B} & \textcolor{white}{Třída 2.A}\\
            \hline
            Porozumění & \hfill 50,73 \% & \hfill 52,55 \% & \hfill 49,26 \% \\
            \hline
            Zábavnost & \hfill 82,75 \% & \hfill 90,10 \% & \hfill 76,54 \% \\
            \hline
            Výstup & \hfill 77,24 \% & \hfill 76,61 \% & \hfill 77,78 \% \\
            \hline
            Prezentace & \hfill 83,44 \% & \hfill 81,39 \% & \hfill 85,19 \% \\
            \hline
            Animace & \hfill 93,28 \% & \hfill 91,48 \% & \hfill 94,81 \% \\
            \hline
            Příklady & \hfill 73,78 \% & \hfill 80,00 \% & \hfill 71,48 \% \\
            \hline
            Pracovní listy & \hfill 74,08 \% & \hfill 75,74 \% & \hfill 72,67 \% \\
            \hline
        \end{tabular}
        \caption{Hodnocení pracovních listů \jaTab}
        \label{tab:zpetnaVazba}
    \end{table}
\end{minipage}\odst
{Žáky hodina bavila, přestože mnozí z nich uvedli, že tématu příliš nerozumí. Výsledky pracovních listů však dopadly obstojně, průměrný žák dostal dvojku (76,5 \%). V pracovním listu byl dále prostor na uvedění porozumění látce, průměrně žáci označili 3,6 z 5. Tato neshoda byla pravděpodobně způsobena mým výkladem, kdy jsem některé informace dostatečně nevysvětlil. Nejkladněji byly hodnoceny animace.}\odst
{Dále jsem se zeptal na dvě otevřené otázky. Z celkových padesáti žáků si 18 z nich stěžovalo na~příliš rychlý výklad. Dále si žáci zkritizovali můj výstup a dotazník, který nebyl příliš intuitivní. Žáci hodnotili souboj v počítání u tabule kladně.}\odst