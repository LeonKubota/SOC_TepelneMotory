\section{Použití pomůcek v praxi}
{Pracovní list pro základní školy a prezentaci ,,Tepelné motory'' jsem vyzkoušel na žácích druhého ročníku Gymnázia Arabská. Učil jsem dvě hodiny, první hodinu jsem učil třídu 2.B, nejprve jsem ukázal prezentaci a poté rozdal pracovní papíry jako domácí úkol. Ve druhé hodině jsem učil třídu \hl{X}, rozdal jse pracovní listy, aby je žáci vyplňovali současně s prezentací a více se soustředili, termín odevzdání byl o několik hodin později.}\odst
{Mé materiály samozřejmě nebyly perfektní. Prezentace potřebovala drobné úpravy, její následnost byla místy neohrabaná a nelogická, tyto problémy jsem opravil upravením pořadí snímků a změnou či přidáním obrázků.}\odst
{Pracovní listy jsem zadal jako dobrovolný domácí úkol, termín na odevzdání byl stanoven na tentýž den. Žáci pracovali ve dvojicích. Výsledky jsou shrnuty v \tabref{tab:vysledkyPracovnichListu}.}
\begin{table}[h]
    \centering
    \begin{tabular}{|c|c|c|c|}
        \hline
        Cvičení & úspěšnost třídy 2.B & úspěšnost třídy \hl{X} & celková úspěšnost\\
        \hline
        Celkem & 73,1 \% & 80 \% & 76,5 \%\\
        Cvičení 1 & 76,9 \% & 77,7 \% & 77,3 \%\\
        Cvičení 2 & 97,4 \% & 97,2 \% & 97,3 \%\\
        Cvičení 3 & 94,9 \% & 100 \% & 97,3 \%\\
        Cvičení 4 & 69,2 \% & 100 \% & 84,0 \%\\
        Cvičení 5 & 46,2 \% & 91,7 \% & 68,0 \%\\
        Cvičení 6 & 76,9 \% & 66,7 \% & 72,0 \%\\
        Cvičení 7 & 38,4 \% & 87,5 \% & 62,0\%\\
        Cvičení 8 & 64,1 \% & 52,8 \% & 58,7 \%\\
        Cvičení 9 & 56,4 \% & 36,1 \% & 46,7 \%\\
        Cvičení 10 & 84,6 \% & 86,1 \% & 85,3 \%\\
        \hline
    \end{tabular}
    \caption{Hodnocení pracovních listů}
    \label{tab:vysledkyPracovnichListu}
\end{table}