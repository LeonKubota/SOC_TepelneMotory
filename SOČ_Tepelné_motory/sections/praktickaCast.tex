\addtocontents{toc}{\newpage}
\section{Úvod do praktické části}
{V této kapitole je popsána tvorba učebních pomůcek, mým cílem bylo vytvořit materiály, které učitelé mohou snadno využít při výuce tepelných motorů a které budou studenty bavit. Toho jsem se pokusil docílit použitím 3D grafiky a animace. Věřím, že žák, kterého výuka baví, si~více zapamatuje a více se naučí.}\odst
{Vzhledem k mým zkušenostem s open source softwarem \code{Blender} pro 3D tvorbu jsem se rozhodl využít právě tento program. \code{Blender} nabízí mnoho nástrojů v okruhu 3D modelování, animace a renderování. Nejprve jsem vytvořil 3D animace jednotlivých tepelných motorů, ty~jsou samostatně dostupné ke stažení v příloze, slouží však především jako podklad pro další výukové materiály, nikoliv jako samostatné výukové pomůcky.}\odst
{Vytvořil jsem celkem 4 prezentace, jedna je určena pro obecné seznámení s tepelnými motory, zbylé tři podrobněji popisují každý druh těchto strojů. Pro snadné sdílení jsem využil služby \code{Google Sheets}, díky které mají učitelé a žáci snadný přístup k prezentacím odkudkoliv. Tyto prezentace obsahují již zmíněné animace a také mnou vytvořené ilustrační obrázky.}\odst
{Další výukovou pomůckou, kterou jsem vytvořil, je výukové video. To popisuje všechny tepelné motory zmíněné v prezentacích pomocí názorné a dynamické 3D animace s hlasovým komentářem. Opět jsem využil software \code{Blender} spolu s programem \code{Audacity} pro nahrání hlasu a~zvukové knihovny \code{Soundly}. Toto video může učitel promítat o hodině či zaslat žákům pro samostatné shlédnutí. Žák na něj může žák sám narazit na platformě \code{Youtube}.}\odst
{Poslední výukovou pomůckou, kterou jsem vytvořil, jsou pracovní listy. Ty jsem vytvořil pomocí sázecího softwaru \LaTeX. První z nich je určen pro žáky základních škol, druhý pro žáky druhého stupně středních škol a gymnázií. Součástí jsou samozřejmě pracovní listy se správnými odpověďmi a návrh na hodnocení. Pracovní listy mohou sloužit pro procvičení či jako písemné práce.}\odst
{Dohromady by tyto pomůcky měly pokrýt výuku jedné až tří vyučovacích hodin v závislosti na způsobu vyučování. Mým cílem bylo nejen vytvořit obsahově hodnotné materiály, ale také díky zajímavé grafické stránce zaujmout žákovu pozornost a vzbudit v něm zájem nejen o tepelné motory, ale o celou fyziku. V následujících kapitolách popíši tvorbu a návrh na použití mých materiálů. Na závěr jsem vyzkoušel své materiály ve výuce žáků druhého ročníku školy Gymnázium, Praha 6, Arabská 14.}