\section{Video}
{Jedním z nejdůležitějších výstupů mé práce je krátké výukové video. V této kapitole popíši a~odůvodním svůj postup při jeho tvorbě. Video je dostupné na platformě \code{YouTube} (viz odkaz: \href{https://www.youtube.com/watch?v=vvIy9x1V8KU}{https://www.youtube.com/watch?v=vvIy9x1V8KU}) a také v plné kvalitě v přílohách zde: \ref{pr:videoTepelneMotory}.}
\subsection{Předprodukce}
{Proces tvorby videa začal již před začátkem samotné 3D tvorby, nejprve jsem si stanovil cíle a~vytvořil přibližný plán videa, který jsem poté rozšířil do scénáře.}
\subsubsection{Cíl videa}
{Nejprve jsem si musel rozmyslet, čeho se vlastně snažím dosáhnout. Cílem mého videa je zábavně a dynamicky vysvětlit žákům látku tepelných motorů, video by zároveň nemělo být příliš dlouhé a mělo by být lehce vstřebatelné a názorné.}\odst
{Abych dosáhl tohoto cíle, musel jsem vytvořit krátký, ale také dostatečně obsáhlý scénář, pomocí kterého lze účinně vysvětlit látku tepelných motorů. Samotný mluvený výklad by mohl působit nudně a nezajímavě, proto jsem pro zpestření využil 3D animaci. Snažil jsem se, aby bylo video srozumitelné a zároveň žáky zaujalo.}
\subsubsection{Plánování a scénář}
{Video je celé synchronizované s hlasovým komentářem, toto načasování by mohla rozhodit i~drobná změna intonace či tempa výkladu. 3D animace je velmi složitý a časově náročný proces, a proto abych předešel opakování práce, vytvořil jsem nejprve detailní scénář, kterého jsem se později musel pevně držet.}\odst
{Na tomto videu jsem s nikým nespolupracoval, můj scénář tedy nemusel být psán ve formátu skutečných scénářů pro film či reklamu. Důležité bylo, abych mu dokázal porozumět já sám. Postačil jednoduchý textový dokument s barevným zvýrazněním, k jeho vytvoření jsem použil textový \textit{editor} \code{Google docs}.}
\newpage
\subsection{Produkce}
{V této kapitole je popsána samotná tvorba videa, ta probíhala téměř výhradně v prostředí softwaru \code{Blender}. Nejprve jsem nahrál hlasový doprovod, poté jsem vytvořil modely, které jsem následně rozhýbal a obarvil. Nakonec jsem vše vyrenderoval a připravil na postprodukci.}
\subsubsection{Hlasový doprovod}
{Hlasový doprovod je základem celého videa, řídí se dle něj veškeré vizuály. Logicky byl tedy první částí, na které jsem začal pracovat.}\odst
{Zvuk jsem nahrával se zapůjčeným mikrofonem RØDE VideoMic Rycote. Tento mikrofon nelze přímo připojit k počítači, proto jsem jako mezičlánek využil kameru JVS GC-PX100 BE. Vše jsem umístil na stativ, abych mikrofon umístil do správné pozice. Hotová sestava je na \obrref{obr:nahravaciSestava}.}
\begin{figure}[H]
    \centering
    \includegraphics[scale=0.08]{assets/images/MikrofonovaSestava.png}
    \caption{Nahrávací sestava \jaFoto}
    \label{obr:nahravaciSestava}
\end{figure}
{Dohromady jsem nahrál přibližně hodinové video, ze kterého jsem ponechal pouze zvukovou stopu. Z té jsem odebral části, kde bylo ticho či kde jsem udělal chybu při čtení scénáře. Nakonec zbylo pouhých třiadvacet minut použitelných pro účely videa. Již takto brzo v procesu tvorby bylo patrné, že video bude dlouhé přibližně osm minut.}\odst
{Jelikož jsem nahrával v domácím prostředí, ve zvukové stopě šel slyšet šum. Tento šum jsem odebral pomocí softwaru \code{Audacity}, konkrétně jsem využil funkci \code{noise gate}.}\odst
{Naopak pro část, kde odpočítávám start rakety, jsem potřeboval zvuk degradovat. Využil jsem opět software \code{Audacity}. Přidal jsem šum, fázový posun a ozvěnu, omezil jsem vysoké a nízké frekvence a na závěr jsem snížil vzorkovací frekvenci na 8000.}\odst
{Výsledkem této části je velice kvalitní zvuková stopa, dle které budu časovat animaci videa.}
\newpage
\subsubsection{Modelování}
{Mimo 3D modely samotných motorů, jejichž tvorbu jsem vysvětlil v kapitole \scref{sc:animace}, jsem vytvořil i mnoho modelů pro dovysvětlení či vizuální zpestření. Většinou jde o motory vozidel. V této kapitole vysvětlím 3D modelování ještě podrobněji.}\odst
{Tvořit modely lze dvěma způsoby: destruktivní a nedestruktivní. Při destruktivním modelování měníme geometrii objektu přímo, posouváme body ve 3D prostoru a tím upravujeme tvar. Výhodou tohoto přístupu je jednoduchost a rychlost, model však nelze zpětně upravit. Nedestruktivní modelování spočívá v budování systémů, které vytvoří geometrii. Můžeme se proto kdykoliv vrátit a upravit jakoukoliv hodnotu. Tento způsob je však velmi zdlouhavý a pro některé modely příliš nepraktický. Oba tyto způsoby mají své místo, pro většinu modelů jsem však zvolil destruktivní či kombinovaný přístup.}\odst
{Při tvorbě modelu velice pomůže předloha, ať už v podobě fotografií či technických plánků. Na \obrref{obr:predlohaFotky} je soubor referenčních obrázků pro tvorbu startovní plošiny pro raketu \textit{Saturn V} organizovaný programem \code{PureRef}, \ref{obr:predlohaVykres} je technický výkres této plošiny.}
\begin{figure}[H]
    \centering
    \begingroup
    \makeatletter
    \renewcommand\thesubfigure{\thefigure~--~\@nameuse{subfiglabel@\alph{subfigure}}}
    \newcommand{\subfiglabel@a}{vlevo}
    \newcommand{\subfiglabel@b}{vpravo}
    \captionsetup[subfigure]{labelformat=simple, labelsep=colon}
    \renewcommand\p@subfigure{}
    \makeatother
    \begin{subfigure}{.7\textwidth}
        \centering
        \includegraphics[scale=.15]{assets/images/PureRef.png}
        \caption{Referenční obrázky pro tvorbu startovací plošiny, \\vytvořeno programem \code{PureRef}}
        \label{obr:predlohaFotky}
    \end{subfigure}%
    \begin{subfigure}{.3\textwidth}
        \centering
        \includegraphics[scale=.2]{assets/images/technickyVykresStartovaciPlosina.jpg}
        \caption{Technický výkres startovací plošiny}
        \label{obr:predlohaVykres}
    \end{subfigure}
    \endgroup
\end{figure}
{Po shromáždění obrázků začneme proces modelování, ten vysvětlím na modelu startovací plošiny. Nejprve si představíme, který základní tvar je nejbližší tvořenému objektu; v tomto případě je to krychle. Tuto krychli upravíme do přibližného tvaru startovací plošiny, pracujeme ve skutečném měřítku, aby byl objekt po přidání do hlavní scény správně velký v porovnání s ostatními objekty. K této krychli přidáme různé detaily, různé přístroje vytvoříme z menších krychlí a potrubí vytvoříme z válců.}
\newpage
{Abych si usnadnil práci, využil jsem osové souměrnosti plošiny a vytvořil jsem tak pouze polovinu modelu. Také díky scénáři vím, že bude tato plošina daleko od kamery a jakým úhlem bude otočena. Nemusím tedy ztrácet čas modelováním detailů, které nepůjdou vidět. Hotový model je na \obrref{obr:plosinaModel}, model ve videu je na \obrref{obr:plosinaModelZezadu}.}
\begin{figure}[H]
    \centering
    \begingroup
    \makeatletter
    \renewcommand\thesubfigure{\thefigure~--~\@nameuse{subfiglabel@\alph{subfigure}}}
    \newcommand{\subfiglabel@a}{vlevo}
    \newcommand{\subfiglabel@b}{vpravo}
    \captionsetup[subfigure]{labelformat=simple, labelsep=colon}
    \renewcommand\p@subfigure{}
    \makeatother
    \begin{subfigure}{.38\textwidth}
        \centering
        \includegraphics[scale=.2]{assets/images/raketaPlatformaModel.png}
        \caption{Model startovací plošiny \jaObr}
        \label{obr:plosinaModel}
    \end{subfigure}
    \hfill
    \begin{subfigure}{.58\textwidth}
        \centering
        \includegraphics[scale=.1301]{assets/images/raketaPlatformaModelZezadu.png}
        \caption{Model startovací plošiny ve videu \jaObr}
        \label{obr:plosinaModelZezadu}
    \end{subfigure}
    \endgroup
\end{figure}
{Model je dokončen, jeho kvalita je samozřejmě subjektivní, můžeme však hodnotit několik objektivních kritérií:}\odst 
{Měli bychom se snažit o ,,správnou'' topologii. Obecně platí, že by objekt měl být tvořen přibližně stejně velkými čtyřúhelníky. Důležitější však je, aby nevznikaly žádné chyby při výpočtu stínů a aby se s modelem lehce pracovalo. Topologie startovací plošiny bezzávadná.}\odst
{Dále by měl model použít co nejméně bodů za zachování vizuální věrnosti předloze, velké množství bodů by mohlo vést k zahlcení pořítače (například naplnění paměti). Tato startovací plošina je tvořena 195 168 body. Celé mé video obsahuje celkem 8,5 milionů bodů.}\odst
{Nejdůležitější však je, zdali model vypadá a působí přesvědčivě jako předmět, který představuje.}
\newpage
\subsubsection{Materiály}
{Materiály určují, jak se světlo chová při interakci s objektem.}\odst
{Lze je tvořit dvěma způsoby: můžeme použít \textit{textury} (obrázky) a pokrýt jimi náš model. Tento způsob velice rychle docílí realistických výsledků, není však přizpůsobivý danému využití. Procedurální způsob naopak nabízí plnou kontrolu, tvoří ho jednoduché operace, které dohromady říkají \code{Blenderu}, jak se má světlo chovat při interakci s objektem.}\odst
{Jelikož cílem tohoto projektu není věrná replikace skutečnosti a procedurální materiály jsou mi bližší, rozhodl jsem se je využít i pro tento projekt. Procedurální materiál se skládá z \textit{nodů} (uzly, které upravují parametry), které jsou do sebe vzájemně zapojeny. Existuje přes 80 \textit{nodů}, zde představím nejdůležitější z nich:}
\begin{itemize}
    \item \code{Principled BSDF}\par
        {Tento \textit{node} je srdcem většiny materiálů, patří do kategorie \textit{shaderů}, ty přímu uvádí, jak se má světlo chovat. Umožňuje nastavit například barvu, kovovost, drsnost, průhlednost či normály (ty vytváří iluzi trojrozměrného povrchu bez přidání geometrie).}
    \item \code{Principled Volume}\par
        {Pokud jde o materiál plynu, využijeme tento \textit{shader}. Můžeme nastavit hustotu, barvu, anizotropii lomu světla či záření plynu.}
    \item \code{Noise Texture}\par
        {Tento uzel patří mezi generující \textit{nody}. Generuje náhodný šum, můžeme upravit jeho velikost, detail, drsnost a vlnivost.}
    \item \code{Math Node}\par
        {Patří mezi upravující \textit{nody}, upravuje vstupní parametry pomocí různých aritmetických, trigonometrických a logických operací.}
    \item \code{Texture Coordinate}\par
        {Umožňuje přístup k datům geometrie; můžeme získat například souřadnice, normály či UV mapu (ta se používá pro mapování 2D obrázků na 3D povrch).}
\end{itemize}
{Materiály se tvoří v \textit{shader editoru}, v tomto okně přidáváme \textit{nody}, viz \obrref{obr:shaderEditor}.}
\begin{figure}[H]
    \centering
    \includegraphics[scale=0.55]{assets/images/shaderEditor.png}
    \caption{Tvorba materiálu v \textit{shader editoru} \jaObr}
    \label{obr:shaderEditor}
\end{figure}
\newpage
\subsubsection{Animace}
{Modely jsou hotové, ale statické. Pro jejich pohyb je třeba je animovat. Využil jsem klíčovou animaci, která je založena na přidávání klíčových snímků. Klíčový snímek, neboli klíčová hodnota, obsahuje informace o tom, jakou hodnotu má mít daný parametr v daný čas. Můžeme tak nastavit pozici, rotaci a velikost objektů, upravit parametry jejich materiálů a další.}\odst
{Pro detailnější ovládání klíčových snímků jsem využil \code{graph editor}, ten umožňuje přesněji měnit časování a nuance pohybu, ukázka \code{graph editoru} je na \obrref{obr:grafEditor}.}
\begin{figure}[H]
    \centering
    \includegraphics[scale=0.5]{assets/images/graphEditor.png}
    \caption{Zobrazení animace v \code{graph editoru} \jaObr}
    \label{obr:grafEditor}
\end{figure}
{Animace musela být synchronizována s komentářem, ten byl nastaven tak, aby se automaticky přehrával spolu s ní. Dále jsem využil i procedurálních animací pomocí \textit{geometry nodes}. Tento přístup jsem zvolil pro animaci mechanismů, které mohou být vyjádřeny předpisy. Například animace pístů, ojnic, hřídelí, ventilů atd. Tímto způsobem jsem vytvořil i~povrch vody v části ,,Parní motory''.}\odst
{Posledním způsobem animace je simulace, tu jsem využil pro vytvoření fyzikálně založených objektů jako jsou například tuhá tělesa, částice, látky či plyny. Při tvorbě simulací je nutné správně nastavit vstupní hodnoty a poté nechat počítač simulaci vypočítat. Simulace často nevypadá tak, jak bychom si ji představovali, proto upravíme parametry a proces opakujeme, dokud nejsou výsledky uspokojivé. Příklad simulace a parametrů naleznete na \obrref{obr:simulace}.}
\begin{figure}[H]
    \centering
    \begingroup
    \makeatletter
    \renewcommand\thesubfigure{\thefigure~--~\@nameuse{subfiglabel@\alph{subfigure}}}
    \newcommand{\subfiglabel@a}{vlevo}
    \newcommand{\subfiglabel@b}{vpravo}
    \captionsetup[subfigure]{labelformat=simple, labelsep=colon}
    \renewcommand\p@subfigure{}
    \makeatother
    \begin{subfigure}{0.35\textwidth}
        \centering
        \includegraphics[scale=0.3]{assets/images/simulaceParametry.png}
        \caption{Parametry simulace\\\jaObr}
    \end{subfigure}%
    \begin{subfigure}{0.6\textwidth}
        \centering
        \includegraphics[scale=0.6]{assets/images/simulaceVysledek.png}
        \caption{Výsledný simulace \jaObr}
    \end{subfigure}
    \endgroup
    \caption{Proces simulace v softwaru \code{Blender}}
    \label{obr:simulace}
\end{figure}
\newpage
\subsubsection{Skutečné záběry}
{Abych propojil obsah videa se skutečností a tím poukázal na to, proč jsou tepelné motory důležité, rozhodl jsem se do videa přidat skutečné záběry. Záběry jsou pořízeny za jízdy z automobilu pomocí mobilního telefonu.}\odst
{Příklad skutečného záběru naleznete na \obrref{obr:skutecnyZaber}.}
\begin{figure}[H]
    \centering
    \includegraphics[width=\textwidth]{assets/images/skutecnyZaber.png}
    \caption{Ukázka skutečného záběru z automobilu \jaFoto}
    \label{obr:skutecnyZaber}
\end{figure}
{Tyto záběry jsem využil v závěru videa.}
\newpage
\subsubsection{Renderování}\label{sc:renderovani}
{Trojrozměrná scéna je pouze sbírka nejrůznějších hodnot jako jsou pozice objektů, jejich geometrie, textury, síla světel a nesmírně mnoho dalších. Tato data se pomocí procesu zvaného renderování převedou na obrázky. Počítač simuluje odrazy paprsků světla a jejich dopady na kameru, poté zaznamená barvu a intenzitu do hodnoty pixelu, na který světlo dopadlo. Konkrétně jsem využil \textit{path tracing} \textit{render engine} \code{Cycles}, který věrně mimikuje skutečné světlo.}\odst
{Pro správné renderování bylo zapotřebí nastavit hodnoty, se kterými bude počítač pracovat. Pokud bych je nastavil příliš mírně, byl by výsledný obraz zrnitý nebo by obsahoval různé artefakty; naopak kdybych je nastavil příliš přísně, renderování by trvalo velice dlouho. Na základě vlastních zkušeností a experimentace jsem zvolil tyto hodnoty (vypsány jen nejdůležitější): }
\begin{itemize}
    \item {\code{Resolution} -- 1920x1080 (HD)}\par
        {Rozlišení výsledného videa, 1920x1080 je v současnosti jedno z nejrozšířenějších rozlišení. Pokud bych zvolil 2560x1440 či vyšší, čas renderování by se velice zvýšil.}
    \item {\code{Frame Rate} -- 24}\par
        {Počet snímků za sekundu videa, 24 je nejnižší standard který se používá.}
    \item {\code{Max Samples} -- 1024}\par
        {Nejvyšší počet paprsků, které dopadnou na jeden pixel. Výchozí hodnota je 4096, ta je však příliš vysoká a pro účely animace nedosažitelná.}
    \item {\code{Persistent Data} -- Zapnuto}\par
        {Zachovává data v operační paměti, razantně zrychlí renderování na úkor zvýšeného využívání operační paměti (desítky GB).}
    \item {\code{View Transform} -- \code{AgX}}\par
        {Způsob, kterým se vyhodnotí barva pixelu. Hlavní předností \code{AgX} je vyblednutí světlých objektů, stejně jako u skutečných kamer.}
\end{itemize}
{Dobu renderování však nejvíce ovlivňuje použitý \textit{hardware}. Toto video jsem renderoval na velice výkonném pracovním počítači opatřeným grafickou kartou \textit{Nvidia RTX 4080}, procesorem \textit{AMD 9 7900X} a 32 GB operační paměti.}\odst
{Produktem renderování v tomto případě není video, ale tisíce obrázků formátu \code{exr}. Tento způsob ukládání přináší mnoho výhod: render může být přerušen, části videa mohou být zpětně upraveny bez nutnosti opakovaného renderování a je zachována vysoká kvalita.}
\newpage
{Do metadat obrázků jsem ukládal čas a paměť potřebnou k vykreslení snímku, pomocí vlastního programu napsaného v jazyce \code{Python} jsem vytvořil grafy, ty naleznete na \obrref{obr:grafRender}. Celé video se renderovalo přes 120 hodin.}
\begin{figure}[H]
    \centering
    \begin{subfigure}{\textwidth}
        \centering
        \includegraphics[width=\textwidth]{assets/images/grafUvod.png}
        \caption{Graf času a paměti potřebné k renderování části ,,Úvod'' \jaGraf}\vspace{0.25cm}
    \end{subfigure}
    \begin{subfigure}{\textwidth}
        \includegraphics[width=\textwidth]{assets/images/grafParniMotory.png}
        \caption{Graf času a paměti potřebné k renderování části ,,Parní motory'' \jaGraf}\vspace{0.25cm}
    \end{subfigure}
    \begin{subfigure}{\textwidth}
        \includegraphics[width=\textwidth, height=3cm]{assets/images/grafSpalovaciMotory.png}
        \caption{Graf času a paměti potřebné k renderování části ,,Spalovací motory'' \jaGraf}\vspace{0.25cm}
    \end{subfigure}
    \begin{subfigure}{\textwidth}
        \includegraphics[width=\textwidth, height=3cm]{assets/images/grafReaktivniMotory.png}
        \caption{Graf času a paměti potřebné k renderování části ,,Reaktivní motory'' \jaGraf}\vspace{0.25cm}
    \end{subfigure}
    \begin{subfigure}{\textwidth}
        \includegraphics[width=\textwidth, height=3cm]{assets/images/grafShrnuti.png}
        \caption{Graf času a paměti potřebné k renderování části ,,Shrnutí'' \jaGraf}
    \end{subfigure}
    \caption{Grafy času a paměti potřebné k renderování}
    \label{obr:grafRender}
\end{figure}
{Tyto grafy jsou užitečné pro určení výpočetně náročných částí. Obecně je nejnáročnější renderovat plyny a modely s velkým množstvím bodů.}
\newpage
\subsection{Postprodukce}
Po vyrenderování ještě není hotovo. Je třeba převést video ze souborů \code{exr} na konvenční video soubor, seřadit video tak, aby správně navazovalo, přidat zvuk, exportovat video a zveřejnit ho.
\subsubsection{Střih a zvuk}
{Vyrenderované sekvence snímků jsem přidal do časové osy \code{Blenderu}, na konec videa jsem přidal skutečné záběry. V těchto skutečných záběrech jsem kvůli soukromí řidičů rozmazal poznávací značky vozidel.}\odst
{Z knihovny zvuků \code{Soundly} jsem do osy přidal stovky zvuků, které doplňují vizuál videa. Osu po přidání zvuků vidíte na \obrref{obr:osaZvuk}.}
\begin{figure}[H]
    \centering
    \includegraphics[width=\textwidth, height=0.3\textwidth]{assets/images/osaZvuk.png}
    \caption{Osa videa po přidání zvuků \jaObr}
    \label{obr:osaZvuk}
\end{figure}
{Většinou tyto zvuky patří mezi nezajímavé \textit{whooshes} (hvízdnutí, pohyb větru), dopady, posuny a podobně. Přesto dodávají videu nutné zvukové pozadí tak, aby upoutaly pozornost.}
\subsubsection{Export a distribuce}
{Na závěr jsem video exportoval ve formátu \code{FFmpeg} a kontejneru \code{MPEG-4}. Pro maximální kompatibilitu jsem využil video kodek \code{H.264} a audio kodek \code{AAC}. Pro maximální kvalitu jsem se rozhodl pro export se stálým datovým tokem.}\odst
{Výsledné video jsem poté nahrál na platformu \code{Youtube}, přidal jsem náhledový obrázek, název a~popisek. Video je dostupné z odkazu: \href{https://www.youtube.com/watch?v=vvIy9x1V8KU}{https://www.youtube.com/watch?v=vvIy9x1V8KU}. Také je dostupné ke stažení v původní kvalitě v přílohách zde: \ref{pr:videoTepelneMotory}.}