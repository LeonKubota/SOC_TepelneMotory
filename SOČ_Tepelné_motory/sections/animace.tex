\section{Animace}\label{sc:animace}
{Pro tvorbu animací jsem využil software \code{Blender}, a to kvůli mým předešlým zkušenostem. Tvorbu animací popíši na příkladu čtyřdobého vznětového motoru 1.9 TDI, ostatní animace jsem tvořil obdobně.}\odst
{Nejprve jsem za pomoci technických výkresů vytvořil geometrii jednotlivých součástek daného motoru, v tomto případě nejprve pístu. Tvorba z výkresu (na \obrref{obr:pistVykresy}) je standardním postupem, je však nutno používat i jiné zdroje a domýšlet si třetí rozměr dané součástky.\\ Na \obrref{obr:pistHotovy} je hotový model pístu.}

\begin{figure}[H]
    \centering
    \begingroup
    \makeatletter
    \renewcommand\thesubfigure{\thefigure~--~\@nameuse{subfiglabel@\alph{subfigure}}}
    \newcommand{\subfiglabel@a}{vlevo}
    \newcommand{\subfiglabel@b}{vpravo}
    \captionsetup[subfigure]{labelformat=simple, labelsep=colon}
    \renewcommand\p@subfigure{}
    \makeatother
    \begin{subfigure}{.5\textwidth}
        \centering
        \includegraphics[scale=.19]{assets/images/pistBok.png}
        \caption{Tvorba pístu motoru 1.9 TDI s použitím technických výkresů \jaObr}
        \label{obr:pistVykresy}
    \end{subfigure}%
    \begin{subfigure}{.5\textwidth}
        \centering
        \includegraphics[scale=.2]{assets/images/pist.png}
        \caption{Hotový píst \jaObr}
        \label{obr:pistHotovy}
    \end{subfigure}
    \endgroup
\end{figure}

{Po vytvoření všech součástek motoru je nutno přidat materiály, rozhodl jsem se je tvořit procedurálně. Tento postup, vyobrazen na \obrref{obr:NodeUkazka}, je tvořen různě spojenými operacmi, které určují jak se má materiál chovat.}

\begin{figure}[H]
    \centering
    \includegraphics[scale=.25]{assets/images/NodeUkazka.png}
    \caption{Ukázka procedurální tvorby materiálů \jaObr}
    \label{obr:NodeUkazka}
\end{figure}

\newpage

{Na \obrref{obr:predMaterialy} je motor 1.9 TDI před přidáním materiálů, na \obrref{obr:poMaterialy} po přidání materiálů. Plochy, kde byl motor rozříznut, jsem obarvil červeně.}

\begin{figure}[H]
    \centering
    \begingroup
    \makeatletter
    \renewcommand\thesubfigure{\thefigure~--~\@nameuse{subfiglabel@\alph{subfigure}}}
    \newcommand{\subfiglabel@a}{vlevo}
    \newcommand{\subfiglabel@b}{vpravo}
    \captionsetup[subfigure]{labelformat=simple, labelsep=colon}
    \renewcommand\p@subfigure{}
    \makeatother
    \begin{subfigure}[t]{.45\textwidth}
        \centering
        \includegraphics[scale=.2]{assets/images/1.9TDImodelUkazka.png}
        \caption{Motor 1.9 TDI před přidáním materiálů \jaObr}
        \label{obr:predMaterialy}
    \end{subfigure}
    \hfill
    \begin{subfigure}[t]{.45\textwidth}
        \centering
        \includegraphics[scale=.2]{assets/images/1.9TDImodelMaterialy.png}
        \caption{Motor 1.9 TDI po přidání materiálů \jaObr}
        \label{obr:poMaterialy}
    \end{subfigure}
    \endgroup
\end{figure}

{Model motoru vypadá jako hotový, je však statický. Animace jsem přidal procedurálně, k tomu jsem využit \code{geometry nodes}. Ty dokáží nedestruktivně; tedy způsobem, při kterém nedochází k nenávratným změnám, upravovat geometrii modelů. Parní turbíny, proudové a raketové motory byli vcelku jednoduché na naanimování, nejsložitější byla animace čtyřdobých motorů a parního stroje, a to kvůli jejich klikovým mechanismům.}\odst
{Vše se odvíjí od pohybu klikové hřídele, na ni jsou napojeny ojnice a na ně písty. Spodní část ojnice se pohybuje po kružnici s poloměrem vzdálenosti od středu klikové hřídele. Ojnici jsem poté otočil tak, aby se střed vrchní části pohyboval pouze vertikálně. Úhel, o který jsem ji musel otočit jsem zjistil pomocí sinové a kosinové větě. Vše dohromady vypadá v \code{Blenderu} takto: \obrref{obr:geometryNodeUkazka}.}

\begin{figure}[H]
    \centering
    \includegraphics[scale=.35]{assets/images/geometryNodeUkazka.png}
    \caption{Ukázka procedurální animace ojnice motoru 1.9 TDI \jaObr}
    \label{obr:geometryNodeUkazka}
\end{figure}

{Při animaci pístů jsem lehce podváděl, písty se pohybují narozdíl od skutečných motorů po sinusoidě, díky lehkým úpravám je však tento nedostatek nepostřehnutelný.}

\newpage

{Podobným způsobem jsem animoval i otevírání a zavírání ventilů. Nakonec jsem přidal barevná ,,vlákna'', která zobrazují pohyb plynu v motoru (\obrref{obr:1.9TDICary}).}

\begin{figure}[H]
    \centering
    \includegraphics[scale=.35]{assets/images/cary.png}
    \caption{,,Vlákna'' zobrazující pohyb plynu \jaObr}
    \label{obr:1.9TDICary}
\end{figure}

{Posledním krokem je převedení 3D modelu na obrázek, tento proces je hlouběji popsán v kapitole \scref{sc:renderovani}. K tomu jsem použil zabudovaný render engine \code{Cycles}, který věrně mimikuje chování skutečného světla. Produkt tohoto postupu je na \obrref{obr:1.9TDIUkazkaHotovo}.}

\begin{figure}[H]
    \begin{minipage}[b]{0.5\linewidth}
        \centering
        \includegraphics[scale=.11]{assets/images/1.9TDI1.png}\\
        {Zážeh ve čtvrtém válci}
        \vspace{4ex}
    \end{minipage}
    \begin{minipage}[b]{0.5\linewidth}
        \centering
        \includegraphics[scale=.11]{assets/images/1.9TDI2.png}\\
        {Zážeh ve třetím válci}
        \vspace{4ex}
    \end{minipage}
    \begin{minipage}[b]{0.5\linewidth}
        \centering
        \includegraphics[scale=.11]{assets/images/1.9TDI3.png}\\
        {Zážeh v prvním válci}
        \vspace{4ex}
    \end{minipage}
    \begin{minipage}[b]{0.5\linewidth}
        \centering
        \includegraphics[scale=.11]{assets/images/1.9TDI4.png}\\
        {Zážeh ve druhém válci}
        \vspace{4ex}
    \end{minipage}
    \caption{Hotový motor 1.9 TDI zachycen v různých částech svého cyklu \jaObr}
    \label{obr:1.9TDIUkazkaHotovo}
\end{figure}

\newpage

\subsection{Animace parního stroje}
{Jako příklad parního stroje jsem vytvořil ležatý dvojčinný parní stroj s šoupátkovým rozvodem, ten můžete vidět na \obrref{obr:ParniStrojRender}. Tento motor není na rozdíl od ostatních modelů založen na žádném skutečném motoru.}

\begin{figure}[H]
    \centering
    \includegraphics[scale=.2]{assets/images/ParniStrojRender.png}
    \caption{Model parního stroje \jaObr}
    \label{obr:ParniStrojRender}
\end{figure}

{Animace je dostupná ke stažení v přílohách zde: \ref{pr:animaceParniStroj}.}

\newpage

\subsection{Animace parní turbíny}
{Vytvořil jsem 3D model parní turbíny MTD-50 společnosti Doosan Škoda Power. Pro tvorbu modelu jsem využil technického výkresu \obrref{obr:DoosanSkodaPowerMTD50}.}
\cite{SP:ApplicationAspectsOfSteamTurbinesForCombinedHeatAndPowerGeneration}

\begin{figure}[H]
    \centering
    \includegraphics[scale=2]{assets/images/DoosanSkodaPowerMTD50.png}
    \caption{Technické výkresy parní turbíny MTD-50 \cite{SP:ApplicationAspectsOfSteamTurbinesForCombinedHeatAndPowerGeneration}}
    \label{obr:DoosanSkodaPowerMTD50}
\end{figure}

{Hotový model turbíny je na \obrref{obr:ParniTurbinaRender}. Animace je dostupná ke stažení v přílohách zde: \ref{pr:animaceParniTurbina}.}

\begin{figure}[H]
    \centering
    \includegraphics[scale=.2]{assets/images/ParniTurbinaRender.png}
    \caption{Model parní turbíny MDT-50 \jaObr}
    \label{obr:ParniTurbinaRender}
\end{figure}

\newpage

\subsection{Animace čtyřdobého spalovacího motoru}
{Jako příklad čtyřdobého spalovacího motoru jsem vybral motor Škoda 1000 MB. Použil jsem výkresy \obrref{obr:1000MBvykresyPrurez} a \obrref{obr:1000MBvykresyBok}.}
\cite{AUTOMOBIL:Skoda1000MBLegendaSlavi60Let}

\begin{figure}[H]
    \begingroup
    \makeatletter
    \renewcommand\thesubfigure{\thefigure~--~\@nameuse{subfiglabel@\alph{subfigure}}}
    \newcommand{\subfiglabel@a}{vlevo}
    \newcommand{\subfiglabel@b}{vpravo}
    \captionsetup[subfigure]{labelformat=simple, labelsep=colon}
    \renewcommand\p@subfigure{}
    \makeatother
    \centering
    \begin{subfigure}{.5\textwidth}
        \centering
        \includegraphics[scale=.2]{assets/images/Skoda1000MBPredek.jpg}
        \caption{Průřez}
        \label{obr:1000MBvykresyPrurez}
    \end{subfigure}%
    \begin{subfigure}{.5\textwidth}
        \centering
        \includegraphics[scale=.2]{assets/images/Skoda1000MBBok.jpg}
        \caption{Bok}
        \label{obr:1000MBvykresyBok}
    \end{subfigure}
    \endgroup
\end{figure}

{Výsledek mé práce je na \obrref{obr:1000MBRender}, animaci můžete stáhnout v přílohách zde: \ref{pr:animaceCtyrdobyZazehovyMotor}.}

\begin{figure}[H]
    \centering
    \includegraphics[scale=0.6]{assets/images/1000MBRender.png}
    \caption{Model čtyřdobého zážehového motoru Škoda 1000 MB \jaObr}
    \label{obr:1000MBRender}
\end{figure}
\newpage

\subsection{Animace čtyřdobého vznětového motoru}
{Za předlohu pro čtyřdobý vznětový motor jsem vybral motor od společnosti Volkswagen, a to motor VW 1.9 R4 8v TDI. Pro tvorbu 3D modelu jsem využil technické výkresy \obrref{obr:1.9TDIvykresyPrurez} a \obrref{obr:1.9TDIvykresyBok}.}
\cite{VWGAG:RealizingFutureTrendsInDieselEngineDevelopment}

\begin{figure}[H]
    \begingroup
    \makeatletter
    \renewcommand\thesubfigure{\thefigure~--~\@nameuse{subfiglabel@\alph{subfigure}}}
    \newcommand{\subfiglabel@a}{vlevo}
    \newcommand{\subfiglabel@b}{vpravo}
    \captionsetup[subfigure]{labelformat=simple, labelsep=colon}
    \renewcommand\p@subfigure{}
    \makeatother
    \centering
    \begin{subfigure}{.5\textwidth}
        \centering
        \includegraphics[scale=.3]{assets/images/1.9TDICrossSection.png}
        \caption{Průřez}
        \label{obr:1.9TDIvykresyPrurez}
    \end{subfigure}%
    \begin{subfigure}{.5\textwidth}
        \centering
        \includegraphics[scale=.3]{assets/images/1.9TDILongitudinalSection.png}
        \caption{Bok}
        \label{obr:1.9TDIvykresyBok}
    \end{subfigure}
    \endgroup
\end{figure}

{Hotový model motoru 1.9 TDI je na \obrref{obr:1.9TDImodel}, animace je dostupná ke stažení zde: \ref{pr:animaceCtyrdobyVznetovyMotor}.}

\begin{figure}[H]
    \centering
    \includegraphics[scale=.2]{assets/images/1.9TDI3.png}
    \caption{Model motoru 1.9 TDI \jaObr}
    \label{obr:1.9TDImodel}
\end{figure}

\newpage

\subsection{Animace dvoudobého zážehového motoru}
{Jako příklad dvoudobého zážehového motoru jsem vybral motor motocyklu ČZ 175, přesněji jeho sportovní verzi se dvěma svíčkami. Jako podklady pro tvorbu modelu jsem využil výkresy \obrref{obr:CZ175Vykres}.}
\cite{MZ:CZ175}

\begin{figure}[H]
    \centering
    \includegraphics[scale=.3]{assets/images/motor175.jpg}
    \caption{Výkresy motoru ČZ 175}
    \label{obr:CZ175Vykres}
\end{figure}

{Hotový model je na \obrref{obr:CZ175Render}, animaci lze stáhnout zde: \ref{pr:animaceDvoudobyZazehovyMotor}.}

\begin{figure}[H]
    \centering
    \includegraphics[scale=.6]{assets/images/CZ175Render.png}
    \caption{Model dvoudobého zážehového motoru ČZ 175 \jaObr}
    \label{obr:CZ175Render}
\end{figure}

\newpage

\subsection{Animace raketového motoru na tuhé palivo}
{Jako příklad raketového motoru na tuhé palivo jsem vybral Hercules/Bermite Mk. 36 z naváděné rakety AIM-9 Sidewinder. Jako podklady pro modely jsem použil výkres \obrref{obr:AIM9vykres}. Pro tuto animaci jsem se rozhodl změnit barvu pozadí na tmavě šedou, aby lépe vynikl plamen, také jsem zobrazil průřez palivem.}
\cite{TB:AIM9Sidewinder}

\begin{figure}[H]
    \centering
    \includegraphics[scale=.5]{assets/images/aim-9-sidewinder.png}
    \caption{Výkres rakety AIM-9 Sidewinder}
    \label{obr:AIM9vykres}
\end{figure}

{Dokončený model je na \ref{obr:AIM9Render}, animaci můžete stáhnout zde: \ref{pr:animacePevnyRaketovyMotor}.}

\begin{figure}[H]
    \centering
    \includegraphics[scale=0.5]{assets/images/AIM9Render.png}
    \caption{Model raketového motoru na tuhé palivo AIM-9 Sidewinder \jaObr}
    \label{obr:AIM9Render}
\end{figure}

\newpage

\subsection{Animace raketového motoru na kapalné palivo}
{Vytvořil jsem model motoru Rocketdyne J-2. Jako podklad pro tvorbu jsem využil technický výkres \obrref{obr:J2vykres}. Ze stejného důvodu jako u raketového motoru na pevné palivo jsem zvolil tmavě šedé pozadí.}
\cite{HR:F1RocketEngine}

\begin{figure}[H]
    \centering
    \includegraphics[scale=.2]{assets/images/RocketdyneJ2vykres.png}
    \caption{Výkres raketového motoru J-2 (přeloženo)}
    \label{obr:J2vykres}
\end{figure}

{Hotový model naleznete na \obrref{obr:J2Render}, animace je dostupná ke stažení zde: \ref{pr:animaceKapalnyRaketovyMotor}.}

\begin{figure}[H]
    \centering
    \includegraphics[scale=.6]{assets/images/J2Render.png}
    \caption{Model raketového motoru na kapalné palivo J-2 \jaObr}
    \label{obr:J2Render}
\end{figure}

\newpage

\subsection{Animace proudového motoru}
{Vytvořil jsem model malého proudového motoru TJ100 vyráběného v PBS Velký Bíteš. Pro tvorbu modelu jsem využil výkresy \obrref{obr:PBSTJ100Vykres}.}
\cite{PBS:Minijets}

\begin{figure}[H]
    \centering
    \includegraphics[scale=.3]{assets/images/TJ100Vykres.png}
    \caption{Výkresy motoru TJ100}
    \label{obr:PBSTJ100Vykres}
\end{figure}

{Model motoru TJ100 je na \obrref{obr:TJ100Render}. Animace je dostupná ke stažení v přílohách zde: \ref{pr:animaceProudovyMotor}.}

\begin{figure}[H]
    \centering
    \includegraphics[scale=0.5]{assets/images/TJ100Render.png}
    \caption{Model proudového motoru TJ100 \jaObr}
    \label{obr:TJ100Render}
\end{figure}

\newpage

\subsection{Animace dvouproudového motoru}
{Na popsání fungování dvouproudového motoru jsem vybral motor CFM International CFM56, který je jedním z nejrozšířenějších proudových motorů vůbec. Ke tvorbě jsem využil výkresy \obrref{obr:CFM56vykres1} a \obrref{obr:CFM56vykres2}.}
\cite{RG:ReducedOrderModel}\cite{TL:OffDesignPerformancePrediction}

\begin{figure}[H]
    \centering
    \begingroup
    \makeatletter
    \renewcommand\thesubfigure{\thefigure~--~\@nameuse{subfiglabel@\alph{subfigure}}}
    \newcommand{\subfiglabel@a}{vlevo}
    \newcommand{\subfiglabel@b}{vpravo}
    \captionsetup[subfigure]{labelformat=simple, labelsep=colon}
    \renewcommand\p@subfigure{}
    \makeatother
    \begin{subfigure}{.5\textwidth}
        \centering
        \includegraphics[scale=.4]{assets/images/CFM56vykres1.png}
        \caption{Výkres motoru CFM56}
        \label{obr:CFM56vykres1}
    \end{subfigure}%
    \begin{subfigure}{.5\textwidth}
        \centering
        \includegraphics[scale=.3]{assets/images/CFM56vykres2.png}
        \caption{Výkres motoru CFM56}
        \label{obr:CFM56vykres2}
    \end{subfigure}
    \endgroup
\end{figure}

{Hotový model je na \obrref{obr:CFM56Render}, animace je dostupná ke stažení zde: \ref{pr:animaceDvouproudovyMotor}.}

\begin{figure}[H]
    \centering
    \includegraphics[scale=0.6]{assets/images/CFM56Render.png}
    \caption{Model dvouproudového motoru CFM56 \jaObr}
    \label{obr:CFM56Render}
\end{figure}