\section{Animace}\label{sc:animace}
{Pro tvorbu animací jsem využil software \code{Blender}, a to na základě mých předešlých zkušeností. Tvorbu animací popíši na příkladu čtyřdobého vznětového motoru Volkswagen R4 8v 1.9 TDI, ostatní animace jsem tvořil obdobně.}\odst
{Nejprve jsem za pomoci technických výkresů vytvořil geometrii jednotlivých součástek daného motoru, v tomto případě nejprve pístu. Tvorba z výkresu (na \obrref{obr:pistVykresy}) je standardním postupem, je však nutno používat i jiné zdroje a domýšlet si třetí rozměr. Na \obrref{obr:pistHotovy} je hotový model pístu.}
\begin{figure}[H]
    \centering
    \begingroup
    \makeatletter
    \renewcommand\thesubfigure{\thefigure~--~\@nameuse{subfiglabel@\alph{subfigure}}}
    \newcommand{\subfiglabel@a}{vlevo}
    \newcommand{\subfiglabel@b}{vpravo}
    \captionsetup[subfigure]{labelformat=simple, labelsep=colon}
    \renewcommand\p@subfigure{}
    \makeatother
    \begin{subfigure}{.5\textwidth}
        \centering
        \includegraphics[scale=.19]{assets/images/pistBok.png}
        \caption{Tvorba pístu motoru VW R4 8v 1.9 TDI s použitím technických výkresů \jaObr}
        \label{obr:pistVykresy}
    \end{subfigure}%
    \begin{subfigure}{.5\textwidth}
        \centering
        \includegraphics[scale=.2]{assets/images/pist.png}
        \caption{Hotový píst \jaObr}
        \label{obr:pistHotovy}
    \end{subfigure}
    \endgroup
\end{figure}
{Po vytvoření všech součástek motoru je nutno přidat materiály, rozhodl jsem se je tvořit procedurálně. Tento postup, vyobrazen na \obrref{obr:NodeUkazka}, je tvořen různě spojenými operacmi, které určují, jak se má materiál chovat.}
\begin{figure}[H]
    \centering
    \includegraphics[scale=.25]{assets/images/NodeUkazka.png}
    \caption{Ukázka procedurální tvorby materiálů \jaObr}
    \label{obr:NodeUkazka}
\end{figure}

\newpage
{Na \obrref{obr:predMaterialy} je motor 1.9 TDI před přidáním materiálů, na \obrref{obr:poMaterialy} po přidání materiálů. Plochy, kde byl motor rozříznut, jsem obarvil červeně.}
\begin{figure}[H]
    \centering
    \begingroup
    \makeatletter
    \renewcommand\thesubfigure{\thefigure~--~\@nameuse{subfiglabel@\alph{subfigure}}}
    \newcommand{\subfiglabel@a}{vlevo}
    \newcommand{\subfiglabel@b}{vpravo}
    \captionsetup[subfigure]{labelformat=simple, labelsep=colon}
    \renewcommand\p@subfigure{}
    \makeatother
    \begin{subfigure}[t]{.45\textwidth}
        \centering
        \includegraphics[scale=.2]{assets/images/1.9TDImodelUkazka.png}
        \caption{Motor 1.9 TDI před přidáním materiálů \jaObr}
        \label{obr:predMaterialy}
    \end{subfigure}
    \hfill
    \begin{subfigure}[t]{.45\textwidth}
        \centering
        \includegraphics[scale=.2]{assets/images/1.9TDImodelMaterialy.png}
        \caption{Motor 1.9 TDI po přidání materiálů \jaObr}
        \label{obr:poMaterialy}
    \end{subfigure}
    \endgroup
\end{figure}
{Model motoru je dokončený, je však statický. Animace jsem přidal procedurálně, k tomu jsem využit \code{geometry nodes}. Ty dokáží nedestruktivně, tedy způsobem, při kterém nedochází k nevratným změnám, upravovat geometrii modelů. Tvorba animace parní turbíny, proudových a~raketových motorů byla poměrně jednoduchá, nejsložitější byla animace čtyřdobých motorů a~parního stroje, a to kvůli jejich komplexním klikovým mechanismům.}\odst
{Hotová předpis pro procedurální animaci vypadá v \code{Blenderu} takto: \obrref{obr:geometryNodeUkazka}.}
\begin{figure}[H]
    \centering
    \includegraphics[scale=.35]{assets/images/geometryNodeUkazka.png}
    \caption{Ukázka procedurální animace ojnice motoru 1.9 TDI \jaObr}
    \label{obr:geometryNodeUkazka}
\end{figure}
\newpage
{Podobným způsobem jsem animoval i otevírání a zavírání ventilů. Nakonec jsem přidal barevná ,,vlákna'', která zobrazují pohyb plynu v motoru (\obrref{obr:1.9TDICary}).}
\begin{figure}[H]
    \centering
    \includegraphics[scale=.35]{assets/images/cary.png}
    \caption{,,Vlákna'' zobrazující pohyb plynu \jaObr}
    \label{obr:1.9TDICary}
\end{figure}
{Posledním krokem je převedení 3D modelu na obrázek, tento proces je hlouběji popsán v kapitole \scref{sc:renderovani}. K tomu jsem použil zabudovaný render engine \code{Cycles}, který věrně mimikuje chování skutečného světla. Produkt tohoto postupu je na \obrref{obr:1.9TDIUkazkaHotovo}.}
\begin{figure}[H]
    \centering
    \includegraphics[scale=0.2]{assets/images/1.9TDI1.png}
    \caption{Hotový motor Volkswagen R4 8v 1.9 TDI \jaObr}
    \label{obr:1.9TDIUkazkaHotovo}
\end{figure}
\newpage
\subsection{Animace parního stroje}
{Jako příklad parního stroje jsem vytvořil ležatý dvojčinný parní stroj se šoupátkovým rozvodem, ten můžete vidět na \obrref{obr:ParniStrojRender}. Tento motor jako jediný není založen na žádném skutečném motoru.}
\begin{figure}[H]
    \centering
    \includegraphics[scale=.2]{assets/images/ParniStrojRender.png}
    \caption{Model parního stroje \jaObr}
    \label{obr:ParniStrojRender}
\end{figure}
\animprilohy{pr:animaceParniStroj}
\newpage
\subsection{Animace parní turbíny}
{Vytvořil jsem 3D model parní turbíny MTD-50 společnosti Doosan Škoda Power. Pro tvorbu modelu jsem využil technického výkresu na \obrref{obr:DoosanSkodaPowerMTD50}.}
\begin{figure}[H]
    \centering
    \includegraphics[scale=2]{assets/images/DoosanSkodaPowerMTD50.png}
    \caption{Technické výkresy parní turbíny MTD-50 \cite{SP:ApplicationAspectsOfSteamTurbinesForCombinedHeatAndPowerGeneration}}
    \label{obr:DoosanSkodaPowerMTD50}
\end{figure}
{Hotový model turbíny je na \obrref{obr:ParniTurbinaRender}.}
\begin{figure}[H]
    \centering
    \includegraphics[scale=.2]{assets/images/ParniTurbinaRender.png}
    \caption{Model parní turbíny MDT-50 \jaObr}
    \label{obr:ParniTurbinaRender}
\end{figure}
\animprilohy{pr:animaceParniTurbina}
\newpage
\subsection{Animace čtyřdobého spalovacího motoru}
{Jako příklad čtyřdobého spalovacího motoru jsem vybral motor Škoda 1000 MB. Použil jsem výkresy \obrref{obr:1000MBvykresyPrurez} a \obrref{obr:1000MBvykresyBok}.}
\begin{figure}[H]
    \begingroup
    \makeatletter
    \renewcommand\thesubfigure{\thefigure~--~\@nameuse{subfiglabel@\alph{subfigure}}}
    \newcommand{\subfiglabel@a}{vlevo}
    \newcommand{\subfiglabel@b}{vpravo}
    \captionsetup[subfigure]{labelformat=simple, labelsep=colon}
    \renewcommand\p@subfigure{}
    \makeatother
    \centering
    \begin{subfigure}{.5\textwidth}
        \centering
        \includegraphics[scale=.2]{assets/images/Skoda1000MBPredek.jpg}
        \caption{Průřez motoru Škoda 1000 MB \cite{AUTOMOBIL:Skoda1000MBLegendaSlavi60Let}}
        \label{obr:1000MBvykresyPrurez}
    \end{subfigure}\hfill
    \begin{subfigure}{.48\textwidth}
        \centering
        \includegraphics[scale=.2]{assets/images/Skoda1000MBBok.jpg}
        \caption{Bok motoru Škoda 1000 MB \cite{AUTOMOBIL:Skoda1000MBLegendaSlavi60Let}}
        \label{obr:1000MBvykresyBok}
    \end{subfigure}
    \endgroup
\end{figure}
{Výsledek mé práce je na \obrref{obr:1000MBRender}.}
\begin{figure}[H]
    \centering
    \includegraphics[scale=0.6]{assets/images/1000MBRender.png}
    \caption{Model čtyřdobého zážehového motoru Škoda 1000 MB \jaObr}
    \label{obr:1000MBRender}
\end{figure}
\animprilohy{pr:animaceCtyrdobyZazehovyMotor}
\newpage
\subsection{Animace čtyřdobého vznětového motoru}
{Za předlohu pro čtyřdobý vznětový motor jsem vybral motor Volkswagen R4 8v 1.9 TDI. Pro tvorbu 3D modelu jsem využil technické výkresy \obrref{obr:1.9TDIvykresyPrurez} a \obrref{obr:1.9TDIvykresyBok}.}
\begin{figure}[H]
    \begingroup
    \makeatletter
    \renewcommand\thesubfigure{\thefigure~--~\@nameuse{subfiglabel@\alph{subfigure}}}
    \newcommand{\subfiglabel@a}{vlevo}
    \newcommand{\subfiglabel@b}{vpravo}
    \captionsetup[subfigure]{labelformat=simple, labelsep=colon}
    \renewcommand\p@subfigure{}
    \makeatother
    \centering
    \begin{subfigure}{.48\textwidth}
        \centering
        \includegraphics[scale=.3]{assets/images/1.9TDICrossSection.png}
        \caption{Průřez motoru Volkswagen R4 8v 1.9 TDI \cite{VWGAG:RealizingFutureTrendsInDieselEngineDevelopment}}
        \label{obr:1.9TDIvykresyPrurez}
    \end{subfigure}\hfill
    \begin{subfigure}{.48\textwidth}
        \centering
        \includegraphics[scale=.3]{assets/images/1.9TDILongitudinalSection.png}
        \caption{Bok motoru Volkswagen R4 8v 1.9 TDI \cite{VWGAG:RealizingFutureTrendsInDieselEngineDevelopment}}
        \label{obr:1.9TDIvykresyBok}
    \end{subfigure}
    \endgroup
\end{figure}
{Hotový model motoru 1.9 TDI je na \obrref{obr:1.9TDImodel}.}
\begin{figure}[H]
    \centering
    \includegraphics[scale=.2]{assets/images/1.9TDI3.png}
    \caption{Model motoru 1.9 TDI \jaObr}
    \label{obr:1.9TDImodel}
\end{figure}
\animprilohy{pr:animaceCtyrdobyVznetovyMotor}
\newpage
\subsection{Animace dvoudobého zážehového motoru}
{Jako příklad dvoudobého zážehového motoru jsem vybral motor z motocyklu Jawa-ČZ 175. Jako podklady pro tvorbu modelu jsem využil technické výkresy \obrref{obr:CZ175Vykres}.}
\begin{figure}[H]
    \centering
    \includegraphics[scale=.3]{assets/images/motor175.jpg}
    \caption{Výkresy motoru Jawa-ČZ 175 \cite{MZ:CZ175}}
    \label{obr:CZ175Vykres}
\end{figure}
{Hotový model je na \obrref{obr:CZ175Render}.}
\begin{figure}[H]
    \centering
    \includegraphics[scale=.6]{assets/images/CZ175Render.png}
    \caption{Model dvoudobého zážehového motoru Jawa-ČZ 175 \jaObr}
    \label{obr:CZ175Render}
\end{figure}
\animprilohy{pr:animaceDvoudobyZazehovyMotor}
\newpage
\subsection{Animace raketového motoru na tuhé palivo}
{Jako příklad raketového motoru na tuhé palivo jsem vybral Hercules/Bermite Mk. 36 z naváděné rakety AIM-9 Sidewinder. Jako podklad pro model jsem použil technický výkres \obrref{obr:AIM9vykres}. Pro tuto animaci jsem se rozhodl změnit barvu pozadí na tmavě šedou, aby lépe vynikl plamen, také jsem zobrazil průřez palivem.}
\begin{figure}[H]
    \centering
    \includegraphics[scale=.5]{assets/images/aim-9-sidewinder.png}
    \caption{Technický výkres rakety AIM-9 Sidewinder \cite{TB:AIM9Sidewinder}}
    \label{obr:AIM9vykres}
\end{figure}
{Dokončený model je na \obrref{obr:AIM9Render}.}
\begin{figure}[H]
    \centering
    \includegraphics[scale=0.5]{assets/images/AIM9Render.png}
    \caption{Model raketového motoru na tuhé palivo AIM-9 Sidewinder \jaObr}
    \label{obr:AIM9Render}
\end{figure}
\animprilohy{pr:animacePevnyRaketovyMotor}
\newpage
\subsection{Animace raketového motoru na kapalné palivo}
{Vytvořil jsem model motoru Rocketdyne J-2. Jako podklad pro tvorbu jsem využil technický výkres na \obrref{obr:J2vykres}. Ze stejného důvodu jako u raketového motoru na tuhé palivo jsem zvolil tmavě šedé pozadí.}
\begin{figure}[H]
    \centering
    \includegraphics[scale=.2]{assets/images/RocketdyneJ2vykres.png}
    \caption{Výkres raketového motoru Rocketdyne J-2 \cite[přeloženo]{HR:F1RocketEngine}}
    \label{obr:J2vykres}
\end{figure}
{Hotový model naleznete na \obrref{obr:J2Render}.}
\begin{figure}[H]
    \centering
    \includegraphics[scale=.6]{assets/images/J2Render.png}
    \caption{Model raketového motoru na kapalné palivo Rocketdyne J-2 \jaObr}
    \label{obr:J2Render}
\end{figure}
\animprilohy{pr:animaceKapalnyRaketovyMotor}
\newpage
\subsection{Animace jednoproudového motoru}
{Vytvořil jsem model malého jednoproudového motoru PBS Velká Bíteš TJ100. Pro tvorbu modelu jsem využil výkresy \obrref{obr:PBSTJ100Vykres}.}
\begin{figure}[H]
    \centering
    \includegraphics[scale=.3]{assets/images/TJ100Vykres.png}
    \caption{Výkresy jednoproudového motoru PBS Velká Bíteš TJ100 \cite[přeloženo]{PBS:Minijets}}
    \label{obr:PBSTJ100Vykres}
\end{figure}
{Model motoru TJ100 je na \obrref{obr:TJ100Render}.}
\begin{figure}[H]
    \centering
    \includegraphics[scale=0.5]{assets/images/TJ100Render.png}
    \caption{Model jednoproudového motoru PBS Velká Bíteš TJ100 \jaObr}
    \label{obr:TJ100Render}
\end{figure}
\animprilohy{pr:animaceProudovyMotor}
\newpage
\subsection{Animace dvouproudového motoru}
{Na popis fungování dvouproudového motoru jsem vybral motor CFM International CFM56, který je jedním z nejrozšířenějších proudových motorů vůbec. Ke tvorbě jsem využil výkresy \obrref{obr:CFM56vykres1} a \obrref{obr:CFM56vykres2}.}
\begin{figure}[H]
    \centering
    \begingroup
    \makeatletter
    \renewcommand\thesubfigure{\thefigure~--~\@nameuse{subfiglabel@\alph{subfigure}}}
    \newcommand{\subfiglabel@a}{vlevo}
    \newcommand{\subfiglabel@b}{vpravo}
    \captionsetup[subfigure]{labelformat=simple, labelsep=colon}
    \renewcommand\p@subfigure{}
    \makeatother
    \begin{subfigure}{.42\textwidth}
        \centering
        \includegraphics[scale=.4]{assets/images/CFM56vykres1.png}
        \caption{Výkres motoru CFM56 \cite{RG:ReducedOrderModel}}
        \label{obr:CFM56vykres1}
    \end{subfigure}\hfill
    \begin{subfigure}{.55\textwidth}
        \centering
        \includegraphics[scale=.3]{assets/images/CFM56vykres2.png}
        \caption{Výkres motoru CFM56 \cite[upraveno]{TL:OffDesignPerformancePrediction}}
        \label{obr:CFM56vykres2}
    \end{subfigure}
    \endgroup
\end{figure}
{Hotový model je na \obrref{obr:CFM56Render}.}
\begin{figure}[H]
    \centering
    \includegraphics[scale=0.2]{assets/images/CFM56Render.png}
    \caption{Model dvouproudového motoru CFM International CFM56 \jaObr}
    \label{obr:CFM56Render}
\end{figure}
\animprilohy{pr:animaceDvouproudovyMotor}