\section{Pracovní listy}
{Vyrobil jsem pracovní list pro žáky základních škol a dvě verze pro středoškoláky. Tvořil jsem pomocí sázecího softwaru \LaTeX. Pracovní list pro základní školy jsem vyzkoušel na žácích druhého ročníku Gymnázia Arabská (více v kapitole \nameref{sc:pouzitiVPraxi} [\ref{sc:pouzitiVPraxi}]). Pracovní listy by měli být užívány spolu s prezentacemi, které jsem vytvořil v kapitole \scref{sc:prezentace}.}\odst
{Pracovní listy jsou dostupné v přílohách zde: \ref{pr:pracovniListy}.}
\subsection{Návrh na hodnocení pracovních listů}
{Učitel se při použití mých pracovních listů může řídit tímto návrhem, nebo může hodnotit dle vlastního úsudku. V obou případech mu práci usnadní opravené šablony pracovních listů, ty jsou také v přílohách.}\odst
{V pracovním listu pro základní školy může žák získat nejvýše 23 bodů. První cvičení je velice jednoduché, proto ho doporučuji hodnotit přísně a za každou chybu strhnout bod z maximálních tří. Ve druhém cvičení ztratí žák bod za každý řádek, toto cvičení žáci bez problémů zvládají. Ve třetím cvičení doplňují žáci hodnoty do tabulky, za každý řádek s chybou ztratí bod. Cvičení 4 a 5 jsou za jeden bod, v odpovědi by se měly vyskytnout čísla motorů a odůvodnění. Ve cvičení 6 získá žák bod za nakreslení šipek a následně za správné popsání. Sedmé cvičení je velice jednoduché, za každou chybnou odpověď ztratí žák bod, takto doporučuji hodnotit i cvičení osm. Cvičení 9 a 10 jsou příklady, žák získá bod za správný převod, za dosazení do rovnic a za správný výsledek. Průměrně žáci získali 20 bodů (viz kapitola \nameref{sc:pouzitiVPraxi} [\ref{sc:pouzitiVPraxi}]).}\odst
{Narozdíl od pracovních listů pro základní školy jsem pracovní listy pro střední školy nevyzkoušel. V prvních čtyř cvičeních ztratí žák bod za každou špatnou odpověď, ve cvičení 5 za každý nesprávný řádek. Ve cvičení 6 ztratí žák bod za každou chybu. Cvičení 7 a 8 odpovídají cvičením 7 a 3 z pracovního sešitu pro základní školy, cvičení 9 je kombinací cvičení 4 a 5. Cvičení 10, 11 a 12 jsou příklady. Žák dostane 2 body za převod jednotek, dosazení do rovnic a výsledek, může se při tom dopustit drobné chyby a ztratí pouze jeden bod. Celkový počet bodů je 40.}