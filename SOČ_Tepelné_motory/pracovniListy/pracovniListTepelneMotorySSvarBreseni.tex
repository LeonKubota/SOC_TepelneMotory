\documentclass[../main.tex]{subfiles}

\begin{document}
\newgeometry{paperwidth=210mm, paperheight=297mm, margin=1cm}
\pagestyle{empty}

%název
\textbf{Varianta B}
\hfill
\textbf{Jméno:}
\tecky{6cm}
\hspace{0.5cm}
\textbf{Hodnocení:}
\tecky{0.5cm}
{/}
{40}
\vspace{0.25cm}
\hrule
\vspace{0.25cm}

\begin{enumerate}[label={\textbf{\arabic*.}}]
\begin{minipage}{0.45\textwidth}
    \item \textbf{Zakroužkujte správné odpovědi:}
        \begin{enumerate}[label={\alph*)}, itemsep=0pt, topsep=0.15cm]
            \item {Výkon je nepřímo úměrný frekvenci cyklů}
            \item[\textcolor{red}{b)}] \textcolor{red}{{Výkon je přímo úměrný frekvenci cyklů}}
            \item[\textcolor{red}{c)}] \textcolor{red}{{Výkon je přímo úměrný práci za cyklus}}
            \item {Výkon je nepřímo úměrný práci za cyklus}
            \item {Žádná z odpovědí není správná}
        \end{enumerate}
    \body{2}

    \item \textbf{Zakroužkujte správné odpovědi:}
        \begin{enumerate}[label={\alph*)}, itemsep=0pt, topsep=0.15cm]
            \item {Účinnost se značí \(\upmu\)}
            \item {Účinnost může být vyšší než 100 \%}
            \item[\textcolor{red}{c)}] \textcolor{red}{{Účinnost se rovná \sfrac{W}{Q\textsubscript{d}}}}
            \item {Účinnost může být maximálně 273,15 kE}
            \item {Žádná z odpovědí není správná}
        \end{enumerate}
    \body{2}

    \item \textbf{Zakroužkujte správné odpovědi:}
        \begin{enumerate}[label={\alph*)}, itemsep=0pt, topsep=0.15cm]
            \item[\textcolor{red}{a)}] \textcolor{red}{{Letadla využívají reaktivní motory}}
            \item[\textcolor{red}{b)}] \textcolor{red}{{V automobilech jsou spalovací motory}}
            \item {Reaktivní motory jsou nejrozšířenější}
            \item[\textcolor{red}{c)}] \textcolor{red}{{Parní motory využíváme v elektrárnách}}
            \item {Žádná z odpovědí není správná}
        \end{enumerate}
    \body{2}

    \item \textbf{Zakroužkujte správné odpovědi:}
        \begin{enumerate}[label={\alph*)}, itemsep=0pt, topsep=0.15cm]
            \item[\textcolor{red}{a)}] \textcolor{red}{{Kelvin je jednotkou teploty}}
            \item {Kelvin je jednotkou tepla}
            \item[\textcolor{red}{c)}] \textcolor{red}{{Odevzdané teplo značíme Q\textsubscript{o}}}
            \item {Teplo dodané značíme T\textsubscript{d}}
            \item {Žádná z odpovědí není správná}
        \end{enumerate}
    \body{2}
    
    \item \textbf{Jak dělíme tepelné motory?}
        \begin{enumerate}[label={\alph*)}, itemsep=0pt, topsep=0.15cm]
            \item \nadtecky{parní}\tecky{2cm} na \nadtecky{parní}\tecky{1cm} stroje a \nadtecky{parní}\tecky{1cm} turbíny
            \item \nadtecky{spalovací}\tecky{2cm} na zážehové a vznětové
            \item \nadtecky{reaktivní}\tecky{2cm} na proudové a raketové
            \item \nadtecky{spalovací}\tecky{2cm} na čtyřdobé a \nadtecky{dvoudobé}\tecky{2cm}
        \end{enumerate}
    \body{2}
\end{minipage}
\hfill
\begin{minipage}[t]{0.4\textwidth}
    \vspace{-12.4cm}

    \item \textbf{Zakreslete do obrázku části proudového motoru}
    
    \vspace{1.25cm}
    \begin{center}
    \includegraphics[scale=0.1]{assets/images/ProudovyMotor.png}
    \begin{tikzpicture}[overlay, remember picture, scale=2/3, shift=({-5.25,-0.5})]

        %kompresor
        \draw[<->, thick] (-3.12,6.5) to (0.09,6.5);
        \draw[dotted, thick,red] (-3.12,6.5) to (-3.12,1.2);
        \draw[dotted, thick,red] (0.09,6.5) to (0.09,2.6);
        \node[anchor=west, above,align=left] at (-2,7) {\nadteckyN{kompresor}\tecky{1.5cm}};
        \draw[thick, ->] (-2,7) to (-1.5,6.6);

        %spalovací komora
        \draw[<->, thick] (0.09,6.5) to (1.32,6.5);
        \node[anchor=west, above, align=left] at (0,8) {\nadteckyN{spalovací komora}\tecky{3cm}};
        \draw[thick, ->] (-0.5,8) to (0.75,6.6);

        %turbína
        \draw[<->, thick] (1.32,6.5) to (2.28,6.5);
        \draw[dotted, thick,red] (1.32,6.5) to (1.32,2.3);
        \draw[dotted, thick,red] (2.28,6.5) to (2.28,2.2);
        \node[anchor=west, above,align=left] at (2,7) {\nadteckyN{turbína}\tecky{1.5cm}};
        \draw[thick, ->] (2,7) to (1.75,6.6);

        %tryska
        \draw[<->, thick] (2.28,6.5) to (4.8,6.5);
        \draw[dotted, thick,red] (4.8,6.5) to (4.8,2.7);
        \node[anchor=west, above,align=left] at (4.1,8) {\nadteckyN{tryska}\tecky{1.5cm}};
        \draw[thick, ->] (4.2,8) to (3.45,6.6);
    \end{tikzpicture}
    \end{center}
    \body{2}

    \item \textbf{Uveďte příklad využití daného motoru:}
    \begin{enumerate}[label={\arabic*.}, itemsep=0pt, topsep=0.15cm]
        \item {Parní stroj}\hfill\nadtecky{Parní lokomotiva}\tecky{3.75cm}
        \item {Raketový motor}\hfill\nadtecky{Raketa}\tecky{3.75cm}
        \item {Proudový motor}\hfill\nadtecky{Letadlo}\tecky{3.75cm}
        \item {Dvoudobý motor}\hfill\nadtecky{Motorová pila}\tecky{3.7cm}
        \item {Čtyřdobý motor}\hfill\nadtecky{Automobil}\tecky{3.75cm}
        \item {Parní turbína}\hfill\nadtecky{Tepelná elektrárna}\tecky{3.75cm}
    \end{enumerate}
    \body{2}
    
    \item \textbf{Doplňte do tabulky hodnoty}
        \begin{center}
            \renewcommand{\arraystretch}{1.25}
            \begin{tabular}{|c|c|c|c|c|} \hline 
                &Q\textsubscript{d} & Q\textsubscript{o} & W & \(\upeta\) \\ \hline  
                1.& 100 J & 50 J &\nadteckyN{50 J}\tecky{1cm} & \nadteckyN{50 \%}\tecky{1cm} \\ \hline  
                3.&300 J & 300 J & \nadteckyN{0 J}\tecky{1cm} & \nadteckyN{0 \%}\tecky{1cm} \\ \hline  
                5.&1200 J & \nadtecky{0 J}\tecky{1cm} & 1200 J & \nadteckyN{100 \%}\tecky{1cm} \\ \hline  
                6.& \nadteckyN{700 J}\tecky{1cm} & 630 J & 70 J & \nadteckyN{10 \%}\tecky{1cm} \\ \hline
            \end{tabular}
        \end{center}
    \body{6}

    \item \textbf{Který motor z tabulky nelze sestrojit? Který lze sestrojit, ale je k ničemu?}\vspace{0.2cm}\\
        \nadtecky{3. motor nelze sestrojit, má účinnost 100 \%, }\tecky{7.75cm}
        \nadtecky{motor č. 2 je k ničemu, nevykoná žádnou práci.}\tecky{7.75cm}
    \body{2}

\end{minipage}

\end{enumerate}

\newpage
\textbf{Varianta B}
\hfill
\textbf{Jméno:}
\tecky{6cm}
\hspace{0.5cm}
\textbf{Hodnocení:}
\tecky{0.5cm}
{/}
{40}
\vspace{0.25cm}
\hrule
\vspace{0.25cm}

\begin{enumerate}[label={\textbf{\arabic*.}}, resume]

    \item \textbf{Vypočítejte práci, kterou vykoná parní stroj, který dostane páru o teplotě 148 °C a vypustí ji o 25 K chladnější. Jeho účinnost je 9 \% a spotřebuje 50 g páry. Měrná tepelná kapacita páry je 1840 J/kg\(\cdot\)K.}
    \vspace{-0.75cm}
    \begin{flushright}
        \begin{minipage}{0.16\textwidth}
            \begin{tcolorbox}[colframe=black, colback=white, boxrule=0.6pt]
                {$\eta=W/Q_d$}
            \end{tcolorbox}
        \end{minipage}
        \begin{minipage}{0.18\textwidth}
            \begin{tcolorbox}[colframe=black, colback=white, boxrule=0.6pt]
                {$Q=m\cdot{c}\cdot{\Delta{T}}$}
            \end{tcolorbox}
        \end{minipage}
    \end{flushright}
    \vspace{-1.25cm}
    \begin{minipage}{0.5\textwidth}
        \begin{center}
            \textcolor{red}{
            \begin{tabular}{l c l}
                    \(W\) & = & ? J\\
                    \(\Delta{T}\) & = & 25 K\\
                    \(\eta\)& = & 9 \% = 0,09 \\
                    \(m\) & = & 50g = 0,05 kg\\
                    \(c\) & = & 1840 \mertepkap 
            \end{tabular}
            }
        \end{center}
        \redeq{W=Q_d\cdot{\eta}}
        \redeq{Q_d=m\cdot{c}\cdot{\Delta{T}}}
    \end{minipage}
    \begin{minipage}{0.5\textwidth}
        \vspace{1.25cm}
        \redeq{W=(m\cdot{c}\cdot{\Delta{T}})\cdot{\eta}}
        \redeq{W=(0,05\cdot{1840}\cdot{25})\cdot{0,09}}
        \redeq{W=207}
    \end{minipage}\odst
    \textcolor{red}{Práce vykonaná parním strojem je 207 J.}
    \body{6}
    \item \textbf{Čtyřdobý spalovací šestiválec vykoná jedním pístem každou pracovní dobu 800 J energie. Jaký je jeho výkon při 7200 otáčkách za minutu?}\\
    \vspace{-0.75cm}
    \begin{flushright}
        \begin{minipage}{0.16\textwidth}
            \begin{tcolorbox}[colframe=black, colback=white, boxrule=0.6pt]
                {$P=W\cdot{ot}$}
            \end{tcolorbox}
        \end{minipage}
    \end{flushright}
    \vspace{-1cm}
    \begin{minipage}{0.5\textwidth}
            \begin{center}
                \textcolor{red}{
                \begin{tabular}{l c l}
                     \(P\) & = & ? W\\
                     \(W\) & = & 800 J\\
                     \(ot\) & = & 7200 \otmin = 120 \ots
                \end{tabular}
                }
            \end{center}
            \redeq{P={W}\cdot{f}}
        \end{minipage}
        \begin{minipage}{0.5\textwidth}
            \vspace{0.5cm}
            \redeq{P=6\cdot{W}\cdot{\frac{ot}{2}}}
            \redeq{P=6\cdot{800}\cdot{\frac{120}{2}}}
            \redeq{P=288000}
        \end{minipage}
    \textcolor{red}{Výkon tohoto motoru je 288000 W neboli 288 kW.}\odst
    \begin{minipage}{0.7\textwidth}
        \small{\textbf{Nápověda: }uvědomte si kolik pracovních dob se odehraje za otáčku (motor je čtyřdobý). Nezapomeňte započítat počet válců.}
    \end{minipage}
    \begin{minipage}{0.25\textwidth}
        \body{6}
    \end{minipage}
    \item \textbf{Raketa s tahem 45 kN má 5875 kg paliva. Kolik minut poletí, pokud má výstupní rychlost 955 \ms a o 14 kPa vyšší výstupní tlak než prostředí? Poloměr trysky je 20 cm.}
    \vspace{-0.25cm}
    \begin{flushright}
        \begin{minipage}{0.29\textwidth}
            \begin{tcolorbox}[colframe=black, colback=white, boxrule=0.6pt]
                {$\vec{F}=v\cdot{\dot{m}-(p_1-p_2)\cdot{S}}$}
            \end{tcolorbox}
        \end{minipage}
        \begin{minipage}{0.14\textwidth}
            \begin{tcolorbox}[colframe=black, colback=white, boxrule=0.6pt]
                {$S=\pi\cdot{r^2}$}
            \end{tcolorbox}
        \end{minipage}
    \end{flushright}
    \vspace{-1.5cm}
    \begin{minipage}{0.5\textwidth}
        \begin{center}
            \textcolor{red}{
            \begin{tabular}{l c l}
                \(t\) & = & ? s\\
                \(\dot{m}\) & = & ? \kgs\\
                \(\vec{F}\) & = & 45 kN = 45000 N\\
                \(m_{palivo}\) & = & 5875 kg\\
                \(v\) & = & 955 \ms\\
                \(\Delta{p}\) & = & 14 kPa = 14000 Pa\\
                \(r\) & = & 20 cm = 0,2 m
            \end{tabular}
            }
        \end{center}
        \redeq{t=\frac{m_{palivo}}{\dot{m}}}
        \redeq{F=\dot{m}\cdot{v}+\Delta{p}\cdot{S}}\odst
        \textcolor{red}{Raketa poletí přibližně 130 sekund.}
    \end{minipage}
    \begin{minipage}{0.5\textwidth}
        \vspace{1.5cm}
        \redeq{\vec{F}=\dot{m}\cdot{v}+\Delta{p}\cdot{\pi\cdot{r^2}}}
        \redeq{\dot{m}=\frac{\vec{F}-\Delta{p}\cdot{\pi\cdot{r^2}}}{v}}
        \redeq{t=\frac{m}{\frac{\vec{F}-\Delta{p}\cdot{\pi\cdot{r^2}}}{v}}}
        \redeq{t=\frac{5875}{\frac{45000-(14000\cdot{\pi\cdot{0,2^2}})}{955}}}
        \redeq{t=129,7533\approx130}
    \end{minipage}
    \vfill
    \body{6}
\end{enumerate}
\end{document}