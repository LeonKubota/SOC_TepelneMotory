\documentclass[../main.tex]{subfiles}

\begin{document}
\newgeometry{paperwidth=210mm, paperheight=297mm, margin=1cm}
\pagestyle{empty}

%název
{\Large\textbf{Tepelné motory}}
\hfill
\textbf{Jméno:}
\nadtecky{Řešení}\tecky{5cm}
\hspace{0.25cm}
\textbf{Hodnocení:}
\tecky{0.5cm}
{/}
%Maximální počet bodů
{23}
\vspace{0.25cm}
\hrule

\begin{enumerate}[label={\textbf{\arabic*.}}]
    \item \textbf{Doplňte slova do textu, vždy doplňte jedno slovo.}\vspace{0.25cm}\\ 
    {Tepelné motory jsou stroje, které převádí \nadtecky{tepelnou}\cara{3cm} energii na \nadtecky{mechanickou}\cara{4cm} práci. \\
    Dělíme je na \nadtecky{parní}\cara{3cm}, \nadtecky{spalovací}\cara{3cm} a \nadtecky{reaktivní}\cara{3cm} motory. Parní motory dělíme na \\
    parní stroje a parní \nadtecky{turbíny}\cara{3cm}, které využíváme například v tepelných \nadtecky{elektrárnách}\cara{3cm}.\\
    \nadtecky{Spalovací}\cara{3cm} motory dělíme na \nadtecky{dvoudobé}\cara{3cm} a \nadtecky{čtyřdobé}\cara{3cm} dle počtu dob, nebo na \\
    zážehové a \nadtecky{vznětové}\cara{3cm} dle způsobu zapálení paliva. Motory, které fungují na základě zákona\\
    akce a \nadtecky{reakce}\cara{3cm} se jmenují \nadtecky{reaktivní}\cara{3cm} motory, dělíme je na \nadtecky{proudové}\cara{3cm} a raketové.}
    \body{3}
    \vspace{-0.5cm}
    \item \textbf{Jaké jsou doby ve čtyřdobém spalovacím motoru? Stručně je popište ve správném pořadí.}
    \begin{enumerate}[label={\arabic*.}]
        \item {Sání:}\hspace{1.75cm}\nadtecky{Píst jede dolů, sací ventil je otevřen a válec se plní směsí.}\tecky{14cm} .
        \item \nadtecky{Komprese}\tecky{2cm}:\hspace{0.5cm}\raisebox{0.2ex}{\rlap{{Píst jede směrem nahoru, oba ventily jsou zavřeny. Tlak a teplota rostou}}}\tecky{14cm} .
        \item \nadtecky{Výbuch}\tecky{2cm}:\hspace{0.5cm}\nadtecky{Směs se zapálí, plyn se rozpíná a koná práci. Ventily jsou zavřeny, píst jede dolů.}\tecky{14cm} .
        \item \nadtecky{Výfuk}\tecky{2cm}:\hspace{0.5cm}\nadtecky{Píst jede směrem nahoru a tlačí ven spaliny výfukovým ventilem.}\tecky{14cm} .
    \end{enumerate}
    \vspace{-0.5cm}
    \body{2}
    \hrule
    \begin{minipage}{0.45\textwidth}
        \vspace{0.25cm}
        \item \textbf{Doplňte tabulku:}
        \begin{center}
        \renewcommand{\arraystretch}{1.25}
        \begin{tabular}{|c|c|c|c|c|} \hline 
            &Q\textsubscript{d} & Q\textsubscript{o} & W & \(\upeta\) \\ \hline  
            1.& \nadteckyN{100 J}\tecky{1cm} & \nadteckyN{50 J}\tecky{1cm} & 50 J & 50 \% \\ \hline  
            2.& \nadteckyN{60 J}\tecky{1cm} & 45 J & \nadteckyN{15 J}\tecky{1cm} & 25 \% \\ \hline 
            3.&1200 J & 0 J & \nadteckyN{1200 J}\tecky{1cm} & \nadteckyN{100 \%}\tecky{1cm} \\ \hline   
            4.&300 J & 300 J & \nadteckyN{0 J}\tecky{1cm} & \nadteckyN{0 \%}\tecky{1cm} \\ \hline  
        \end{tabular}
        \end{center}
        \body{3}
    \end{minipage}\hfill
    \begin{minipage}[t]{0.45\textwidth}
        \vspace{-2.35cm}
        \item \textbf{Který motor ze \underline{cvičení 4} nelze sestrojit? Proč?}\\
        \nadtecky{3. motor nelze sestrojit, má účinnost 100 \%.}\tecky{8cm}
        \nadtecky{Účinnost tepelného motoru musí být nižší.}\tecky{8cm}
        \body{1}
        \item \textbf{Který motor ze \underline{cvičení 4} je k ničemu? Proč?}\\
        \nadtecky{4. motor je k ničemu, nevykoná žádnou práci,}\tecky{8cm}
        \nadtecky{protože jeho účinnost je 0 \%.}\tecky{8cm}
        \body{1}
    \end{minipage}
    \vspace{0.25cm}
    \hrule
    \begin{minipage}{0.45\textwidth}
        \vspace{0.25cm}
        \item \textbf{Zakreslete akci a reakci do obrázku:}
        \begin{center}
        \begin{tikzpicture} 
            \node[opacity=0.7]{\includegraphics[scale=0.12]{assets/images/Raketa.png}};
            \draw[ultra thick, ->, red] (-0.75,-1.6) to (-0.67,-1.6+1.5);
            \draw[ultra thick, ->, red] (-0.75,-1.6) to (-0.83,-1.6-1.5);
            \draw[ultra thick, red, fill = red] (-0.75,-1.6) circle (0.075);
            \node[red, left] at (-0.8, -1) {\textbf{Reakce}};
            \node[red, left] at (-0.8, -2) {\textbf{Akce}};
        \end{tikzpicture}
        \end{center}
        \body{2}
    \end{minipage}\hfill
    \begin{minipage}{0.45\textwidth}
        \vspace{-1.5cm}
        \item \textbf{K čemu motor pravděpodobně využijeme?}
        \begin{tcolorbox}[colframe=black, colback=white, boxrule=0.6pt]
            {motorová pila \(\bullet\) letadlo \(\bullet\) automobil \(\bullet\) \\ raketoplán \(\bullet\) elektrárna \(\bullet\) parní lokomotiva}
        \end{tcolorbox}
        \begin{enumerate}[label={\arabic*.}]
            \item {Raketový motor}\hfill\nadtecky{raketoplán}\tecky{4cm}
            \item {Čtyřdobý motor}\hfill\nadtecky{automobil}\tecky{4cm}
            \item {Parní stroj}\hfill\nadtecky{parní lokomotiva}\tecky{4cm}
            \item {Parní turbína}\hfill\nadtecky{elektrárna}\tecky{4cm}
            \item {Dvoudobý motor}\hfill\nadtecky{motorová pila}\tecky{4cm}
            \item {Proudový motor}\hfill\nadtecky{letadlo}\tecky{4cm}
        \end{enumerate}
        \body{2}
    \end{minipage}
    \hfill
    \newpage\pagestyle{empty}
\end{enumerate}
    {\Large\textbf{Tepelné motory}}
    \hfill
    \textbf{Jméno:}
    \tecky{3cm}
    \hspace{0.25cm}
    \textbf{Hodnocení:}
    \tecky{0.5cm}
    {/}
    %Maximální počet bodů
    {23}
    \vspace{0.25cm}
    \hrule
\begin{enumerate}[label={\textbf{\arabic*.}}, resume]

    \item \textbf{Co jsou součástky na obrázku? Z jakého motoru asi jsou?}
        \begin{enumerate}[label={\arabic*.}, wide=0pt]
        \item \nadtecky{kliková hřídel}\tecky{5.3cm}
        \item \nadtecky{svíčka}\tecky{5.3cm}
        \item \nadtecky{ventil}\tecky{5.3cm}
        \item \nadtecky{píst}\tecky{5.3cm}
        \item \nadtecky{ojnice}\tecky{5.3cm}
        \item[Druh motoru:]\nadtecky{čtyřdobý zážehový}\tecky{3cm}
    \end{enumerate}
    \begin{tikzpicture}[overlay, remember picture]
        \node at (13cm, 2cm) {\includegraphics[scale=0.175]{assets/images/Soucastky.png}};
    \end{tikzpicture}
    \body{3}

    \item \textbf{Vypočítejte práci parního stroje, jestliže přijme 8 kJ tepla a odevzdá 7,5 kJ? Jaký je jeho výkon pokud každou sekundu vykoná cyklus čtyřikrát?}
    \vspace{-0.75cm}
    
    \begin{flushright}
        \begin{minipage}{0.32\textwidth}
            \begin{tcolorbox}[colframe=black, colback=white, boxrule=0.6pt]
                {\(W=Q_d-Q_o\)\hspace{0.25cm}\textbf{a}\hspace{0.25cm}  \(P=f\cdot{W}\)}
            \end{tcolorbox}
        \end{minipage}
    \end{flushright}

    \begin{minipage}[]{0.5\textwidth}
        \begin{center}
            \textcolor{red}{
            \begin{tabular}{l l}
                W & ? J\\
                P & ? W\\
                Q\textsubscript{d} & 8 kJ = 8000 J\\
                Q\textsubscript{o} & 7,5 kJ = 7500 J\\
                f & 4 Hz
            \end{tabular}
            }
        \end{center}
        \redeq{W=Q_d-Q_o}
        \redeq{W=8000-7500}
    \end{minipage}%
    \begin{minipage}{0.5\textwidth}
        \redeq{W=500}
        \redeq{P=f\cdot{W}}
        \redeq{P=4\cdot{500}}
        \redeq{P=2000}
    \end{minipage}

    \nadtecky{Práce jednoho cyklu je 500 J a výkon je 2000 W.}\dotfill
    
    \body{3}

    \item \textbf{Letadlo letí rychlostí 720 km/h, každou sekundu nasaje 195 kg vzduchu a spálí 5 kg paliva, tato směs uniká z motoru rychlostí 330 m/s. Jaký je tah?}
    \vspace{-0.75cm}
    \begin{flushright}
        \begin{minipage}{0.25\textwidth}
            \begin{tcolorbox}[colframe=black, colback=white, boxrule=0.6pt]
                {\(\vec{F}=\dot{m}_{1}\cdot{v_1}-\dot{m}_{0}\cdot{v_0}\)}
            \end{tcolorbox}
        \end{minipage}
    \end{flushright}

    \begin{minipage}{0.5\textwidth}
        \begin{center}
            \textcolor{red}{
            \begin{tabular}{l l}
                {F} & ? N \\
                v\textsubscript{0} & 720 \kms = 200 \ms\\
                \dottext{m}\textsubscript{0} & 195 \kgs\\
                \dottext{m}\textsubscript{palivo} & 5 \kgs\\
                \dottext{m}\textsubscript{1} & ? \kgs\\
                v\textsubscript{1} & 330 \ms
            \end{tabular}
            }
        \end{center}
        \redeq{\vec{F}=\dot{m}_{1}\cdot{v_1}-\dot{m}_{0}\cdot{v_0}}  
    \end{minipage}%
    \begin{minipage}{0.5\textwidth}
        \redeq{\dot{m}_{1}=\dot{m}_{0}+\dot{m}_{palivo}}
        \redeq{\vec{F}=(\dot{m}_{0}+\dot{m}_{palivo})\cdot{v_1}-\dot{m}_{0}\cdot{v_0}}
        \redeq{\vec{F}=(195+5)\cdot{330}-195\cdot{200}}
        \redeq{\vec{F}=27000}
    \end{minipage}

    \nadtecky{Tah motoru je 27000 N neboli 27 kN.}\dotfill

    \body{3}
    \hrule
\end{enumerate}
\begin{minipage}{0.5\textwidth}
    \textbf{BONUS: Jak dobře rozumíte tepelným motorům?}
\end{minipage}
\hfill
\begin{minipage}{0.4\textwidth}
    \begin{tcolorbox}[colframe=black, colback=white, boxrule=0.6pt]
        \Large{\frownie}
        \hfill
        \tikz[baseline, yshift=0.2cm]{
            \draw (0,0) -- (5,0);
            \foreach \x in {0,1,...,5} {
                \draw (\x,0.1) -- (\x,-0.1);
            }
        }
        \hfill
        \Large{\smiley} 
    \end{tcolorbox}
\end{minipage}
\restoregeometry
\end{document}