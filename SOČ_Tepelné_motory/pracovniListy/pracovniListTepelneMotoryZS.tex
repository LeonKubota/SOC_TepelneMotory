\documentclass[../main.tex]{subfiles}

\begin{document}
\newgeometry{paperwidth=210mm, paperheight=297mm, margin=1cm}
\pagestyle{empty}

%název
{\Large\textbf{Tepelné motory}}
\hfill
\textbf{Jméno:}
\tecky{5cm}
\hspace{0.25cm}
\textbf{Hodnocení:}
\tecky{0.5cm}
{/}
%Maximální počet bodů
{23}
\vspace{0.25cm}
\hrule

\begin{enumerate}[label={\textbf{\arabic*.}}]
    \item \textbf{Doplňte slova do textu, vždy doplňte jedno slovo.}\vspace{0.25cm}\\ 
    {Tepelné motory jsou stroje, které převádí \cara{3cm} energii na \cara{4cm} práci. \\
    Dělíme je na \cara{3cm}, \cara{3cm} a \cara{3cm} motory. Parní motory dělíme na \\
    parní stroje a parní \cara{3cm}, které využíváme například v tepelných \cara{3cm}.\\
    \cara{3cm} motory dělíme na \cara{3cm} a \cara{3cm} dle počtu dob, nebo na \\
    zážehové a \cara{3cm} dle způsobu zapálení paliva. Motory, které fungují na základě zákona\\
    akce a \cara{3cm} se jmenují \cara{3cm} motory, dělíme je na \cara{3cm} a raketové.}
    \body{3}
    \vspace{-0.5cm}
    \item \textbf{Jaké jsou doby ve čtyřdobém spalovacím motoru? Stručně je popište ve správném pořadí.}
    \begin{enumerate}[label={\arabic*.}]
        \item {Sání:}\hspace{1.75cm}\tecky{14cm} .
        \item \tecky{2cm}:\hspace{0.5cm}{Píst jede směrem nahoru, oba ventily jsou zavřeny. Tlak a teplota rostou} .
        \item \tecky{2cm}:\hspace{0.5cm}\tecky{14cm} .
        \item \tecky{2cm}:\hspace{0.5cm}\tecky{14cm} .
    \end{enumerate}
    \vspace{-0.5cm}
    \body{2}
    \hrule
    \begin{minipage}{0.45\textwidth}
        \vspace{0.25cm}
        \item \textbf{Doplňte tabulku:}
        \begin{center}
        \renewcommand{\arraystretch}{1.25}
        \begin{tabular}{|c|c|c|c|c|} \hline 
            &Q\textsubscript{d} & Q\textsubscript{o} & W & \(\upeta\) \\ \hline  
            1.& \tecky{1cm} & \tecky{1cm} & 50 J & 50 \% \\ \hline  
            2.&\tecky{1cm} & 45 J & \tecky{1cm} & 25 \% \\ \hline 
            3.&1200 J & 0 J & \tecky{1cm} & \tecky{1cm} \\ \hline   
            4.&300 J & 300 J & \tecky{1cm} & \tecky{1cm} \\ \hline  
        \end{tabular}
        \end{center}
        \body{3}
    \end{minipage}\hfill
    \begin{minipage}[t]{0.45\textwidth}
        \vspace{-2.35cm}
        \item \textbf{Který motor ze \underline{cvičení 4} nelze sestrojit? Proč?}\\
        \tecky{8cm}
        \tecky{8cm}
        \body{1}
        \item \textbf{Který motor ze \underline{cvičení 4} je k ničemu? Proč?}\\
        \tecky{8cm}
        \tecky{8cm}
        \body{1}
    \end{minipage}
    \vspace{0.25cm}
    \hrule
    \begin{minipage}{0.45\textwidth}
        \vspace{0.25cm}
        \item \textbf{Zakreslete akci a reakci do obrázku:}
        \begin{center}
        \begin{tikzpicture}
            \node[opacity=0.7]{\includegraphics[scale=0.12]{assets/images/Raketa.png}};
        \end{tikzpicture}
        \end{center}
        \body{2}
    \end{minipage}\hfill
    \begin{minipage}{0.45\textwidth}
        \vspace{-1.5cm}
        \item \textbf{K čemu motor pravděpodobně využijeme?}
        \begin{tcolorbox}[colframe=black, colback=white, boxrule=0.6pt]
            {motorová pila \(\bullet\) letadlo \(\bullet\) automobil \(\bullet\) \\ raketoplán \(\bullet\) elektrárna \(\bullet\) parní lokomotiva}
        \end{tcolorbox}
        \begin{enumerate}[label={\arabic*.}]
            \item {Raketový motor}\hfill\tecky{4cm}
            \item {Čtyřdobý motor}\hfill\tecky{4cm}
            \item {Parní stroj}\hfill\tecky{4cm}
            \item {Parní turbína}\hfill\tecky{4cm}
            \item {Dvoudobý motor}\hfill\tecky{4cm}
            \item {Proudový motor}\hfill\tecky{4cm}
        \end{enumerate}
        \body{2}
    \end{minipage}
    \hfill
    
    \newpage\pagestyle{empty}
\end{enumerate}
    {\Large\textbf{Tepelné motory}}
    \hfill
    \textbf{Jméno:}
    \tecky{3cm}
    \hspace{0.25cm}
    \textbf{Hodnocení:}
    \tecky{0.5cm}
    {/}
    %Maximální počet bodů
    {23}
    \vspace{0.25cm}
    \hrule
\begin{enumerate}[label={\textbf{\arabic*.}}, resume]

    \item \textbf{Co jsou součástky na obrázku? Z jakého motoru asi jsou?}
        \begin{enumerate}[label={\arabic*.}, wide=0pt]
            \item \tecky{5.3cm}
            \item \tecky{5.3cm}
            \item \tecky{5.3cm}
            \item \tecky{5.3cm}
            \item \tecky{5.3cm}
            \item[Druh motoru:]\tecky{3cm}
        \end{enumerate}
        \begin{tikzpicture}[overlay, remember picture]
            \node at (12cm, 2cm) {\includegraphics[scale=0.2]{assets/images/Soucastky.png}};
        \end{tikzpicture}
    \body{3}
    
    \item \textbf{Vypočítejte práci parního stroje, jestliže přijme 8 kJ tepla a odevzdá 7,5 kJ? Jaký je jeho výkon pokud každou sekundu vykoná cyklus čtyřikrát?}
    \vspace{-0.75cm}
    \begin{flushright}
        \begin{minipage}{0.32\textwidth}
            \begin{tcolorbox}[colframe=black, colback=white, boxrule=0.6pt]
                {\(W=Q_d-Q_o\)\hspace{0.25cm}\textbf{a}\hspace{0.25cm}  \(P=f\cdot{W}\)}
            \end{tcolorbox}
        \end{minipage}
    \end{flushright}
    \vspace{6cm}
    \body{3}

    \item \textbf{Letadlo letí rychlostí 720 km/h, každou sekundu nasaje 195 kg vzduchu a spálí 5 kg paliva, tato směs uniká z motoru rychlostí 330 m/s. Jaký je tah?}
    \vspace{-0.75cm}
    \begin{flushright}
        \begin{minipage}{0.25\textwidth}
            \begin{tcolorbox}[colframe=black, colback=white, boxrule=0.6pt]
                {\(\vec{F}=\dot{m}_{1}\cdot{v_1}-\dot{m}_{0}\cdot{v_0}\)}
            \end{tcolorbox}
        \end{minipage}
    \end{flushright}
    \vspace{6cm}
    \body{3}

    \hrule
\end{enumerate}
\begin{minipage}{0.5\textwidth}
    \textbf{BONUS: Jak dobře rozumíte tepelným motorům?}
\end{minipage}
\hfill
\begin{minipage}{0.4\textwidth}
    \begin{tcolorbox}[colframe=black, colback=white, boxrule=0.6pt]
        \Large{\frownie}
        \hfill
        \tikz[baseline, yshift=0.2cm]{
            \draw (0,0) -- (5,0);
            \foreach \x in {0,1,...,5} {
                \draw (\x,0.1) -- (\x,-0.1);
            }
        }
        \hfill
        \Large{\smiley} 
    \end{tcolorbox}
\end{minipage}
\restoregeometry
\end{document}