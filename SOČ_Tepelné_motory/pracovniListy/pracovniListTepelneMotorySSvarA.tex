\documentclass[../main.tex]{subfiles}

\begin{document}
\newgeometry{paperwidth=210mm, paperheight=297mm, margin=1cm}
\pagestyle{empty}

%název
\textbf{Varianta A}
\hfill
\textbf{Jméno:}
\tecky{6cm}
\hspace{0.5cm}
\textbf{Hodnocení:}
\tecky{0.5cm}
{/}
{40}
\vspace{0.25cm}
\hrule
\vspace{0.25cm}

\begin{enumerate}[label={\textbf{\arabic*.}}]
\begin{minipage}{0.45\textwidth}
    \item \textbf{Zakroužkujte správné odpovědi:}
        \begin{enumerate}[label={\alph*)}, itemsep=0pt, topsep=0.15cm]
            \item {Výkon je nepřímo úměrný práci za cyklus}
            \item {Výkon je přímo úměrný práci za cyklus}
            \item {Výkon je nepřímo úměrný frekvenci cyklů}
            \item {Výkon je přímo úměrný frekvenci cyklů}
            \item {Žádná z odpovědí není správná}
        \end{enumerate}
    \body{2}

    \item \textbf{Zakroužkujte správné odpovědi:}
        \begin{enumerate}[label={\alph*)}, itemsep=0pt, topsep=0.15cm]
            \item {Účinnost může být maximálně 273,15 kE}
            \item {Účinnost se rovná \sfrac{W}{Q\textsubscript{d}}}
            \item {Účinnost může být vyšší než 100 \%}
            \item {Účinnost se značí \(\upmu\)}
            \item {Žádná z odpovědí není správná}
        \end{enumerate}
    \body{2}

    \item \textbf{Zakroužkujte správné odpovědi:}
        \begin{enumerate}[label={\alph*)}, itemsep=0pt, topsep=0.15cm]
            \item {Parní motory využíváme v elektrárnách}
            \item {Reaktivní motory jsou nejrozšířenější}
            \item {V automobilech jsou spalovací motory}
            \item {Letadla využívají reaktivní motory}
            \item {Žádná z odpovědí není správná}
        \end{enumerate}
    \body{2}

    \item \textbf{Zakroužkujte správné odpovědi:}
        \begin{enumerate}[label={\alph*)}, itemsep=0pt, topsep=0.15cm]
            \item {Teplo dodané značíme T\textsubscript{d}}
            \item {Odevzdané teplo značíme Q\textsubscript{o}}
            \item {Kelvin je jednotkou teploty}
            \item {Kelvin je jednotkou tepla}
            \item {Žádná z odpovědí není správná}
        \end{enumerate}
    \body{2}
    
    \item \textbf{Jak dělíme tepelné motory?}
        \begin{enumerate}[label={\alph*)}, itemsep=0pt, topsep=0.15cm]
            \item \tecky{2cm} na čtyřdobé a \tecky{2cm}
            \item \tecky{2cm} na proudové a raketové
            \item \tecky{2cm} na \tecky{1cm} stroje a \tecky{1cm} turbíny
            \item \tecky{2cm} na zážehové a vznětové
        \end{enumerate}
    \body{2}
\end{minipage}
\hfill
\begin{minipage}[t]{0.4\textwidth}
    \vspace{-12.75cm}

    \item \textbf{Zakreslete do obrázku sílu akce a reakce}
        \begin{tikzpicture} 
            \node[opacity=0.7]{\includegraphics[scale=0.12]{assets/images/Raketa.png}};
        \end{tikzpicture}
    \body{2}

    \item \textbf{Uveďte příklad využití daného motoru:}
    \begin{enumerate}[label={\arabic*.}, itemsep=0pt, topsep=0.15cm]
        \item {Čtyřdobý motor}\hfill\tecky{3.75cm}
        \item {Parní turbína}\hfill\tecky{3.75cm}
        \item {Dvoudobý motor}\hfill\tecky{3.7cm}
        \item {Raketový motor}\hfill\tecky{3.75cm}
        \item {Parní stroj}\hfill\tecky{3.75cm}
        \item {Proudový motor}\hfill\tecky{3.75cm}
    \end{enumerate}
    \body{2}
    
    \item \textbf{Doplňte do tabulky hodnoty}
        \begin{center}
            \renewcommand{\arraystretch}{1.25}
            \begin{tabular}{|c|c|c|c|c|} \hline 
                &Q\textsubscript{d} & Q\textsubscript{o} & W & \(\upeta\) \\ \hline  
                1.& \tecky{1cm} & \tecky{1cm} & 50 J & 50 \% \\ \hline  
                2.&1200 J & 0 J & \tecky{1cm} & \tecky{1cm} \\ \hline  
                3.&300 J & 300 J & \tecky{1cm} & \tecky{1cm} \\ \hline  
                4.&700 J & \tecky{1cm} & \tecky{1cm} & 10 \% \\ \hline
            \end{tabular}
        \end{center}
    \body{6}

    \item \textbf{Který motor z tabulky nelze sestrojit? Který lze sestrojit, ale je k ničemu?}\vspace{0.2cm}\\
        \tecky{7.75cm}
        \tecky{7.75cm}
    \body{2}

\end{minipage}

\end{enumerate}

\newpage
\textbf{Varianta A}
\hfill
\textbf{Jméno:}
\tecky{6cm}
\hspace{0.5cm}
\textbf{Hodnocení:}
\tecky{0.5cm}
{/}
{40}
\vspace{0.25cm}
\hrule
\vspace{0.25cm}

\begin{enumerate}[label={\textbf{\arabic*.}}, resume]

    \item \textbf{Vypočítejte termodynamickou účinnost parního stroje, který dostane páru o teplotě 227 °C a vypustí ji o 50 K chladnější. Vykoná 368 J práce za jednu otáčku a spotřebuje 50 g páry. Měrná tepelná kapacita páry je 1840 J/kg\(\cdot\)K.}
    \vspace{-0.75cm}
    \begin{flushright}
        \begin{minipage}{0.16\textwidth}
            \begin{tcolorbox}[colframe=black, colback=white, boxrule=0.6pt]
                {$\eta=W/Q_d$}
            \end{tcolorbox}
        \end{minipage}
        \begin{minipage}{0.18\textwidth}
            \begin{tcolorbox}[colframe=black, colback=white, boxrule=0.6pt]
                {$Q=m\cdot{c}\cdot{\Delta{T}}$}
            \end{tcolorbox}
        \end{minipage}
    \end{flushright}
    \vspace{4cm}
    \body{6}
    \item \textbf{Čtyřdobý spalovací čtyřválec vykoná jedním pístem každou pracovní dobu 750 J energie. Jaký je jeho výkon při 4800 otáčkách za minutu?}
    \vspace{-0.75cm}
    \begin{flushright}
        \begin{minipage}{0.16\textwidth}
            \begin{tcolorbox}[colframe=black, colback=white, boxrule=0.6pt]
                {$P=W\cdot{ot}$}
            \end{tcolorbox}
        \end{minipage}
    \end{flushright}
    \vspace{4cm}
    \begin{minipage}{0.7\textwidth}
        \small{\textbf{Nápověda: }uvědomte si kolik pracovních dob se odehraje za otáčku (motor je čtyřdobý). Nezapomeňte započítat počet válců.}
    \end{minipage}
    \hfill
    \begin{minipage}{0.25\textwidth}
        \body{6}
    \end{minipage}

    \item \textbf{Vypočítejte, jaký poloměr výstupní trysky potřebuje raketový motor, který má tah 2500 kN při výstupové rychlosti 2830 \ms, hmotnostním průtoku 745 \kgs a výstupový tlak o 20 kPa vyšší než tlak prostředí. Zaokrouhlete na dvě desetinná místa.}
    \vspace{-0.75cm}
    \begin{flushright}
        \begin{minipage}{0.29\textwidth}
            \begin{tcolorbox}[colframe=black, colback=white, boxrule=0.6pt]
                {$\vec{F}=v\cdot{\dot{m}-(p_1-p_2)\cdot{S}}$}
            \end{tcolorbox}
        \end{minipage}
        \begin{minipage}{0.14\textwidth}
            \begin{tcolorbox}[colframe=black, colback=white, boxrule=0.6pt]
                {$S=\pi\cdot{r^2}$}
            \end{tcolorbox}
        \end{minipage}
    \end{flushright}
    \vfill
    \body{6}
    
\end{enumerate}
%}
\end{document}