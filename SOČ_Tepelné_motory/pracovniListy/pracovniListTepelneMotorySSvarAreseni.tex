\documentclass[../main.tex]{subfiles}

\begin{document}
\newgeometry{paperwidth=210mm, paperheight=297mm, margin=1cm}
\pagestyle{empty}

%název
\textbf{Varianta A}
\hfill
\textbf{Jméno:}
\tecky{6cm}
\hspace{0.5cm}
\textbf{Hodnocení:}
\tecky{0.5cm}
{/}
{40}
\vspace{0.25cm}
\hrule
\vspace{0.25cm}

\begin{enumerate}[label={\textbf{\arabic*.}}]
\begin{minipage}{0.45\textwidth}
    \item \textbf{Zakroužkujte správné odpovědi:}
        \begin{enumerate}[label={\alph*)}, itemsep=0pt, topsep=0.15cm]
            \item {Výkon je nepřímo úměrný práci za cyklus}
            \item[\textcolor{red}{b)}] \textcolor{red}{{Výkon je přímo úměrný práci za cyklus}}
            \item {Výkon je nepřímo úměrný frekvenci cyklů}
            \item[\textcolor{red}{c)}] \textcolor{red}{{Výkon je přímo úměrný frekvenci cyklů}}
            \item {Žádná z odpovědí není správná}
        \end{enumerate}
    \body{2}

    \item \textbf{Zakroužkujte správné odpovědi:}
        \begin{enumerate}[label={\alph*)}, itemsep=0pt, topsep=0.15cm]
            \item {Účinnost může být maximálně 273,15 kE}
            \item[\textcolor{red}{b)}] \textcolor{red}{{Účinnost se rovná \sfrac{W}{Q\textsubscript{d}}}}
            \item {Účinnost může být vyšší než 100 \%}
            \item {Účinnost se značí \(\upmu\)}
            \item {Žádná z odpovědí není správná}
        \end{enumerate}
    \body{2}

    \item \textbf{Zakroužkujte správné odpovědi:}
        \begin{enumerate}[label={\alph*)}, itemsep=0pt, topsep=0.15cm]
            \item[\textcolor{red}{a)}] \textcolor{red}{{Parní motory využíváme v elektrárnách}}
            \item {Reaktivní motory jsou nejrozšířenější}
            \item[\textcolor{red}{c)}] \textcolor{red}{{V automobilech jsou spalovací motory}}
            \item[\textcolor{red}{d)}] \textcolor{red}{{Letadla využívají reaktivní motory}}
            \item {Žádná z odpovědí není správná}
        \end{enumerate}
    \body{2}

    \item \textbf{Zakroužkujte správné odpovědi:}
        \begin{enumerate}[label={\alph*)}, itemsep=0pt, topsep=0.15cm]
            \item {Teplo dodané značíme T\textsubscript{d}}
            \item[\textcolor{red}{b)}] \textcolor{red}{{Odevzdané teplo značíme Q\textsubscript{o}}}
            \item[\textcolor{red}{c)}] \textcolor{red}{{Kelvin je jednotkou teploty}}
            \item {Kelvin je jednotkou tepla}
            \item {Žádná z odpovědí není správná}
        \end{enumerate}
    \body{2}
    
    \item \textbf{Jak dělíme tepelné motory?}
        \begin{enumerate}[label={\alph*)}, itemsep=0pt, topsep=0.15cm]
            \item \nadtecky{spalovací}\tecky{2cm} na čtyřdobé a \nadtecky{dvoudobé}\tecky{2cm}
            \item \nadtecky{reaktivní}\tecky{2cm} na proudové a raketové
            \item \nadtecky{parní}\tecky{2cm} na \nadtecky{parní}\tecky{1cm} stroje a \nadtecky{parní}\tecky{1cm} turbíny
            \item \nadtecky{spalovací}\tecky{2cm} na zážehové a vznětové
        \end{enumerate}
    \body{2}
\end{minipage}
\hfill
\begin{minipage}[t]{0.4\textwidth}
    \vspace{-12.75cm}

    \item \textbf{Zakreslete do obrázku sílu akce a reakce}
        \begin{tikzpicture} 
            \node[opacity=0.7]{\includegraphics[scale=0.12]{assets/images/Raketa.png}};
            \draw[ultra thick, ->, red] (-0.75,-1.6) to (-0.67,-1.6+1.5);
            \draw[ultra thick, ->, red] (-0.75,-1.6) to (-0.83,-1.6-1.5);
            \draw[ultra thick, red, fill = red] (-0.75,-1.6) circle (0.075);
            \node[red, left] at (-0.8, -1) {\textbf{Reakce}};
            \node[red, left] at (-0.8, -2) {\textbf{Akce}};
        \end{tikzpicture}
    \body{2}

    \item \textbf{Uveďte příklad využití daného motoru:}
    \begin{enumerate}[label={\arabic*.}, itemsep=0pt, topsep=0.15cm]
        \item {Čtyřdobý motor}\hfill\nadtecky{Automobil}\tecky{3.75cm}
        \item {Parní turbína}\hfill\nadtecky{Tepelná elektrárna}\tecky{3.75cm}
        \item {Dvoudobý motor}\hfill\nadtecky{Motorová pila}\tecky{3.7cm}
        \item {Raketový motor}\hfill\nadtecky{Raketa}\tecky{3.75cm}
        \item {Parní stroj}\hfill\nadtecky{Parní lokomotiva}\tecky{3.75cm}
        \item {Proudový motor}\hfill\nadtecky{Letadlo}\tecky{3.75cm}
    \end{enumerate}
    \body{2}
    
    \item \textbf{Doplňte do tabulky hodnoty}
        \begin{center}
            \renewcommand{\arraystretch}{1.25}
            \begin{tabular}{|c|c|c|c|c|} \hline 
                &Q\textsubscript{d} & Q\textsubscript{o} & W & \(\upeta\) \\ \hline  
                1.& \nadteckyN{100 J}\tecky{1cm} & \nadteckyN{50 J}\tecky{1cm} & 50 J & 50 \% \\ \hline  
                2.&1200 J & 0 J & \nadteckyN{1200 J}\tecky{1cm} & \nadteckyN{100 \%}\tecky{1cm} \\ \hline
                3.&300 J & 300 J & \nadteckyN{0 J}\tecky{1cm} & \nadteckyN{0 \%}\tecky{1cm} \\ \hline   
                4.&700 J & \nadteckyN{70 J}\tecky{1cm} & \nadteckyN{630 J}\tecky{1cm} & 10 \% \\ \hline
            \end{tabular}
        \end{center}
    \body{6}

    \item \textbf{Který motor z tabulky nelze sestrojit? Který lze sestrojit, ale je k ničemu?}\vspace{0.2cm}\\
        \nadtecky{2. motor nelze sestrojit, má účinnost 100 \%, }\tecky{7.75cm}
        \nadtecky{motor č. 3 je k ničemu, nevykoná žádnou práci.}\tecky{7.75cm}
    \body{2}

\end{minipage}

\end{enumerate}

\newpage
\textbf{Varianta A}
\hfill
\textbf{Jméno:}
\tecky{6cm}
\hspace{0.5cm}
\textbf{Hodnocení:}
\tecky{0.5cm}
{/}
{40}
\vspace{0.25cm}
\hrule
\vspace{0.25cm}

\begin{enumerate}[label={\textbf{\arabic*.}}, resume]

    \item \textbf{Vypočítejte termodynamickou účinnost parního stroje, který dostane páru o teplotě 227 °C a vypustí ji o 50 K chladnější. Vykoná 368 J práce za jednu otáčku a spotřebuje 50 g páry. Měrná tepelná kapacita páry je 1840 J/kg\(\cdot\)K.}
    \vspace{-0.75cm}
    \begin{flushright}
        \begin{minipage}{0.16\textwidth}
            \begin{tcolorbox}[colframe=black, colback=white, boxrule=0.6pt]
                {$\eta=W/Q_d$}
            \end{tcolorbox}
        \end{minipage}
        \begin{minipage}{0.18\textwidth}
            \begin{tcolorbox}[colframe=black, colback=white, boxrule=0.6pt]
                {$Q=m\cdot{c}\cdot{\Delta{T}}$}
            \end{tcolorbox}
        \end{minipage}
    \end{flushright}
    \vspace{-1cm}
    \begin{minipage}{0.5\textwidth}
        \begin{center}
            \textcolor{red}{
            \begin{tabular}{l c l}
                    \(\eta\)& = & ? \\
                    \(\Delta{T}\) & = & 50 K\\
                    \(W\) & = & 368 J\\
                    \(m\) & = & 50g = 0,05 kg\\
                    \(c\) & = & 1840 \mertepkap 
            \end{tabular}
            }
        \end{center}
        \redeq{\eta=\frac{W}{Q_d}}
        \redeq{Q_d=m\cdot{c}\cdot{\Delta{T}}}
    \end{minipage}
    \begin{minipage}{0.5\textwidth}
        \vspace{1cm}
        \redeq{\eta=\frac{W}{m\cdot{c}\cdot{\Delta{T}}}}
        \redeq{\eta=\frac{368}{0,05\cdot{1840}\cdot{50}}}
        \redeq{\eta=0,08}
    \end{minipage}
    \textcolor{red}{Termodynamická účinnost parního stroje je 8 \%.}
    \body{6}
    \item \textbf{Čtyřdobý spalovací čtyřválec vykoná jedním pístem každou pracovní dobu 750 J energie. Jaký je jeho výkon při 4800 otáčkách za minutu?}
    \vspace{-0.75cm}
    \begin{flushright}
        \begin{minipage}{0.16\textwidth}
            \begin{tcolorbox}[colframe=black, colback=white, boxrule=0.6pt]
                {$P=W\cdot{ot}$}
            \end{tcolorbox}
        \end{minipage}
    \end{flushright}
    \vspace{-1cm}
    \begin{minipage}{0.5\textwidth}
        \begin{center}
            \textcolor{red}{
            \begin{tabular}{l c l}
                    \(P\) & = & ? W\\
                    \(W\) & = & 750 J\\
                    \(ot\) & = & 4800 \otmin = 80 \ots
            \end{tabular}
            }
        \end{center}
        \redeq{P={W}\cdot{ot}}
    \end{minipage}
    \begin{minipage}{0.5\textwidth}
        \vspace{0.75cm}
        \redeq{P=4\cdot{W}\cdot{\frac{ot}{2}}}
        \redeq{P=4\cdot{750}\cdot{\frac{80}{2}}}
        \redeq{P=120000}
    \end{minipage}
    \textcolor{red}{Výkon tohoto motoru je 120000 W neboli 120 kW.}\odst
    \begin{minipage}{0.7\textwidth}
        \small{\textbf{Nápověda: }uvědomte si kolik pracovních dob se odehraje za otáčku (motor je čtyřdobý). Nezapomeňte započítat počet válců.}
    \end{minipage}
    \hfill
    \begin{minipage}{0.25\textwidth}
        \body{6}
    \end{minipage}

    \item \textbf{Vypočítejte, jaký poloměr trysky potřebuje raketový motor, který má tah 2500 kN při výstupové rychlosti 2830 \ms, hmotnostním průtoku 745 \kgs a výstupový tlak o 20 kPa vyšší než tlak prostředí. Zaokrouhlete na dvě desetinná místa.}
    \vspace{-0.75cm}
    \begin{flushright}
        \begin{minipage}{0.29\textwidth}
            \begin{tcolorbox}[colframe=black, colback=white, boxrule=0.6pt]
                {$\vec{F}=v\cdot{\dot{m}-(p_1-p_2)\cdot{S}}$}
            \end{tcolorbox}
        \end{minipage}
        \begin{minipage}{0.14\textwidth}
            \begin{tcolorbox}[colframe=black, colback=white, boxrule=0.6pt]
                {$S=\pi\cdot{r^2}$}
            \end{tcolorbox}
        \end{minipage}
    \end{flushright}
    \begin{minipage}{0.5\textwidth}
        \begin{center}
            \textcolor{red}{
            \begin{tabular}{l c l}
                \(r\)& = & ? m\\
                \(F\)& = & 2500 kN = 2500000 N\\
                \(v_1\) & = & 2830 \ms\\
                \(\dot{m}\) & = & 745 \kgs\\
                \(\Delta{p}\) & = & 20 kPa = 20000 Pa
            \end{tabular}
            }
        \end{center}
        \redeq{F=\dot{m}\cdot{v_1}+\Delta{p}\cdot{S}}
        \redeq{S=\pi\cdot{r^2}}
    \end{minipage}
    \begin{minipage}{0.5\textwidth}
        \redeq{F=\dot{m}\cdot{v_1}+\Delta{p}\cdot{\pi\cdot{r^2}}}
        \redeq{r=\sqrt{\frac{F-\dot{m}\cdot{v_1}}{\Delta{p}\cdot{\pi}}}}
        \redeq{r=\sqrt{\frac{2500000-745\cdot{2830}}{20000\cdot{\pi}}}}
        \redeq{r=2,4966\approx{2,50}}
    \end{minipage}
    \vfill
    \nadtecky{Poloměr trysky je přibližně 2,50 m.}
    \body{6}
    \newpage
\end{enumerate}
\restoregeometry
\end{document}