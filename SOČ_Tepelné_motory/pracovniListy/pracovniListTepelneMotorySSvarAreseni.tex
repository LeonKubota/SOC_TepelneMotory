\documentclass[../main.tex]{subfiles}

\begin{document}
\newgeometry{paperwidth=210mm, paperheight=297mm, margin=1cm}
\pagestyle{empty}

%název
\textbf{A}
\hspace{1cm}
\textbf{Jméno:}
\tecky{4cm}
\hspace{0.25cm}
\textbf{Příjmení}
\tecky{4cm}
\hfill
\textbf{Hodnocení:}
\tecky{0.5cm}
{/}
{40}
\vspace{0.25cm}
\hrule
\vspace{0.25cm}

\begin{enumerate}[label={\textbf{\arabic*.}}]
\begin{minipage}{0.45\textwidth}
    \item \textbf{Zakroužkujte správné odpovědi:}
        \begin{enumerate}[label={\alph*)}, itemsep=0pt, topsep=0.15cm]
            \item {Výkon je nepřímo úměrný práci za cyklus}
            \item \textcolor{red}{{Výkon je přímo úměrný práci za cyklus}}
            \item {Výkon je nepřímo úměrný frekvenci cyklů}
            \item \textcolor{red}{{Výkon je přímo úměrný frekvenci cyklů}}
            \item {Žádná z odpovědí není správná}
        \end{enumerate}
    \body{2}

    \item \textbf{Zakroužkujte správné odpovědi:}
        \begin{enumerate}[label={\alph*)}, itemsep=0pt, topsep=0.15cm]
            \item {Účinnost může být maximálně 273,15 kE}
            \item \textcolor{red}{{Účinnost se rovná \sfrac{W}{Q\textsubscript{d}}}}
            \item {Účinnost může být vyšší než 100 \%}
            \item {Účinnost se značí \(\upmu\)}
            \item {Žádná z odpovědí není správná}
        \end{enumerate}
    \body{2}

    \item \textbf{Zakroužkujte správné odpovědi:}
        \begin{enumerate}[label={\alph*)}, itemsep=0pt, topsep=0.15cm]
            \item \textcolor{red}{{Parní motory využíváme v elektrárnách}}
            \item {Reaktivní motory jsou nejrozšířenější}
            \item \textcolor{red}{{V automobilech jsou spalovací motory}}
            \item \textcolor{red}{{Letadla využívají reaktivní motory}}
            \item {Žádná z odpovědí není správná}
        \end{enumerate}
    \body{2}

    \item \textbf{Zakroužkujte správné odpovědi:}
        \begin{enumerate}[label={\alph*)}, itemsep=0pt, topsep=0.15cm]
            \item {Teplo dodané značíme T\textsubscript{d}}
            \item \textcolor{red}{{Odevzdané teplo značíme Q\textsubscript{o}}}
            \item \textcolor{red}{{Kelvin je jednotkou teploty}}
            \item {Kelvin je jednotkou tepla}
            \item {Žádná z odpovědí není správná}
        \end{enumerate}
    \body{2}
    
    \item \textbf{Jak dělíme tepelné motory?}
        \begin{enumerate}[label={\alph*)}, itemsep=0pt, topsep=0.15cm]
            \item \nadtecky{spalovací}\tecky{2cm} na čtyřdobé a \nadtecky{dvoudobé}\tecky{2cm}
            \item \nadtecky{reaktivní}\tecky{2cm} na proudové a raketové
            \item \nadtecky{parní}\tecky{2cm} na \nadtecky{parní}\tecky{1cm} stroje a \nadtecky{parní}\tecky{1cm} turbíny
            \item \nadtecky{spalovací}\tecky{2cm} na zážehové a vznětové
        \end{enumerate}
    \body{2}
\end{minipage}
\hfill
\begin{minipage}[t]{0.4\textwidth}
    \vspace{-12.4cm}

    \item \textbf{Zakreslete do obrázku sílu akce a reakce}
        \begin{center}
            \includegraphics[scale=0.11]{../images/RaketaReseni.png}
        \end{center}
    \body{2}

    \item \textbf{Uveďte příklad využití daného motoru:}
    \begin{enumerate}[label={\arabic*.}, itemsep=0pt, topsep=0.15cm]
        \item {Čtyřdobý motor}\hfill\nadtecky{Automobil}\tecky{3.75cm}
        \item {Parní turbína}\hfill\nadtecky{Tepelná elektrárna}\tecky{3.75cm}
        \item {Dvoudobý motor}\hfill\nadtecky{Motorová pila}\tecky{3.75cm}
        \item {Raketový motor}\hfill\nadtecky{Raketa}\tecky{3.75cm}
        \item {Parní stroj}\hfill\nadtecky{Parní lokomotiva}\tecky{3.75cm}
        \item {Proudový motor}\hfill\nadtecky{Letadlo}\tecky{3.75cm}
    \end{enumerate}
    \body{2}
    
    \item \textbf{Doplňte do tabulky hodnoty}
        \begin{center}
            \renewcommand{\arraystretch}{1.25}
            \begin{tabular}{|c|c|c|c|c|} \hline 
                &Q\textsubscript{d} & Q\textsubscript{o} & W & \(\upeta\) \\ \hline  
                1.& \nadteckyN{100 J}\tecky{1cm} & \nadteckyN{50 J}\tecky{1cm} & 50 J & 50 \% \\ \hline  
                2.&600 J & \nadteckyN{210 J}\tecky{1cm} & \nadteckyN{390 J}\tecky{1cm} & 35 \% \\ \hline  
                3.&300 J & 300 J & \nadteckyN{0 J}\tecky{1cm} & \nadteckyN{0 \%}\tecky{1cm} \\ \hline  
                4.&\nadteckyN{8000 J}\tecky{1cm} & \nadteckyN{3600 J}\tecky{1cm} & 4400 J & 45 \% \\ \hline  
                5.&1200 J & 0 J & \nadteckyN{1200 J}\tecky{1cm} & \nadteckyN{100 \%}\tecky{1cm} \\ \hline  
                6.&700 J & \nadteckyN{70 J}\tecky{1cm} & \nadteckyN{630 J}\tecky{1cm} & 10 \% \\ \hline  
            \end{tabular}
        \end{center}
    \body{6}

    \item \textbf{Který motor z tabulky nelze sestrojit? Který lze sestrojit, ale je k ničemu?}\vspace{0.2cm}\\
        \nadtecky{5. motor nelze sestrojit, má účinnost 100 \%, }\tecky{7.75cm}
        \nadtecky{motor č. 3 je k ničemu, nevykoná žádnou práci.}\tecky{7.75cm}
    \body{2}

\end{minipage}

\end{enumerate}

\newpage
\textbf{A}
\hspace{1cm}
\textbf{Jméno:}
\tecky{4cm}
\hspace{0.25cm}
\textbf{Příjmení}
\tecky{4cm}
\hfill
\textbf{Hodnocení:}
\tecky{0.5cm}
{/}
{40}
\vspace{0.25cm}
\hrule
\vspace{0.25cm}

\begin{enumerate}[label={\textbf{\arabic*.}}, resume]

    \item \textbf{Vypočítejte termodynamickou účinnost parního stroje, který dostane páru o teplotě 227 °C a vypustí ji o 50 K chladnější. Vykoná 368 J práce za jednu otáčku a spotřebuje 50 g páry. Měrná tepelná kapacita páry je 1840 J/kg\(\cdot\)K.}

        \begin{minipage}{0.5\textwidth}
            \begin{center}
                \textcolor{red}{
                \begin{tabular}{l l}
                     \(\eta\)& ? \\
                     \(\Delta{T}\) & 50 K\\
                     \(W\) & 368 J\\
                     \(m\) & 50g = 0,05 kg\\
                     \(c\) & 1840 \mertepkap 
                \end{tabular}
                }
            \end{center}

            \redeq{\eta=\frac{W}{Q_d}}
        \end{minipage}
        \begin{minipage}{0.5\textwidth}
            
            \redeq{Q_d=m\cdot{c}\cdot{\Delta{T}}}
            \redeq{\eta=\frac{W}{m\cdot{c}\cdot{\Delta{T}}}}
            \redeq{\eta=\frac{368}{0,05\cdot{1840}\cdot{50}}}
            \redeq{\eta=0,08}
        \end{minipage}

    \vfill
    
    \nadtecky{Termodynamická účinnost parního stroje je 8 \%.}\dotfill
    \body{6}
    
    \item \textbf{Čtyřdobý spalovací čtyřválec vykoná jedním pístem každou pracovní dobu 750 J energie. Jaký je jeho výkon při 4800 otáčkách za minutu?}

    \begin{minipage}{0.5\textwidth}
            \begin{center}
                \textcolor{red}{
                \begin{tabular}{l l}
                     \(P\) & ? W\\
                     \(W\) & 750 J\\
                     \(ot\) & 4800 \otmin = 80 \ots
                \end{tabular}
                }
            \end{center}
            \redeq{P={W}\cdot{f}}
            
        \end{minipage}
        \begin{minipage}{0.5\textwidth}
            \redeq{P=4\cdot{W}\cdot{\frac{ot}{2}}}
            \redeq{P=4\cdot{750}\cdot{\frac{80}{2}}}
            \redeq{P=120000}
        \end{minipage}

    \vfill
    
    \nadtecky{Výkon tohoto motoru je 120000 W neboli 120 kW.}\dotfill\odst
    \begin{minipage}{0.7\textwidth}
        \small{\textbf{Nápověda: }uvědomte si kolik pracovních dob se odehraje za otáčku (motor je čtyřdobý). Nezapomeňte započítat počet válců.}
    \end{minipage}
    \hfill
    \begin{minipage}{0.25\textwidth}
        \body{6}
    \end{minipage}

    \item \textbf{Vypočítejte, jaký poloměr trysky potřebuje raketový motor, který má tah 2500 kN při výstupové rychlosti 2830 \ms, hmotnostním průtoku 870 \kgs a výstupový tlak o 19 kPa vyšší než tlak prostředí. Zaokrouhlete na dvě desetinná místa.}

        \begin{minipage}{0.5\textwidth}
            \begin{center}
                \textcolor{red}{
                \begin{tabular}{l l}
                    \(r\) & ? m\\
                    \(F\)& 2500 kN = 2500000 N\\
                    \(v_1\) & 2830 \ms\\
                    \(\dot{m}\) & 870 \kgs\\
                    \(\Delta{p}\) & 19 kPa = 19000 Pa
                \end{tabular}
                }
            \end{center}
            \redeq{F=\dot{m}\cdot{v_1}+\Delta{p}\cdot{S}}
            \redeq{S=\pi\cdot{r^2}}
        \end{minipage}
        \begin{minipage}{0.5\textwidth}
            \redeq{F=\dot{m}\cdot{v_1}+\Delta{p}\cdot{\pi\cdot{r^2}}}
            \redeq{r=\sqrt{\frac{F-\dot{m}\cdot{v_1}}{\Delta{p}\cdot{\pi}}}}
            \redeq{r=\sqrt{\frac{2500000-870\cdot{2830}}{20000\cdot{\pi}}}}
            \redeq{r=0,7968\approx{0,80}}
        \end{minipage}

    \vfill
    
    \nadtecky{Poloměr trysky je přibližně 0,80 m.}\dotfill
    \body{6}

    \newpage
\end{enumerate}
\restoregeometry
\end{document}